\documentclass[12pt,twoside]{report}
\usepackage[activeacute,spanish]{babel}
\usepackage[width= 150mm,top=25mm,bottom=25mm]{geometry}
\usepackage{graphicx} % Required for inserting images

\usepackage{amssymb}
\usepackage{amsthm}
\usepackage{amsmath}
\usepackage{mathtools}
\usepackage{mathrsfs}
\usepackage{extpfeil}
\usepackage{enumitem}
\usepackage{microtype}
\usepackage[style=mexican]{csquotes} 

\usepackage[style= alphabetic]{biblatex}
\addbibresource{referencias.bib}


\usepackage{parskip}
\setlength{\headheight}{16pt}
\usepackage{fancyhdr}
\pagestyle{fancy}
\fancyhead{}
\fancyhead[RO,LE]{\nouppercase{\leftmark}}

\usepackage{tikz-cd}
\usetikzlibrary{decorations.pathmorphing} 
\usetikzlibrary{babel}
\usetikzlibrary{calc}

\usepackage[hidelinks]{hyperref}

\makeatletter
\renewenvironment{proof}[1][\proofname]{\par
\pushQED{\qed}%
\normalfont \topsep-1\p@\@plus6\p@\relax
\trivlist
\item\relax
{\itshape
#1\@addpunct{.}}\hspace\labelsep\ignorespaces
}{%
\popQED\endtrivlist\@endpefalse
}
\makeatother

\theoremstyle{definition}
\newtheorem{defi}{Definición}[section]
\newtheorem{ejem}[defi]{Ejemplo}
\newtheorem{ejems}[defi]{Ejemplos}

\theoremstyle{plain}
\newtheorem{prop}[defi]{Proposición}
\newtheorem{teo}[defi]{Teorema}
\newtheorem*{teo*}{Teorema}
\newtheorem{coro}[defi]{Corolario}

\newcommand{\N}[0]{\mathbb N}
\newcommand{\Q}[0]{\mathbb Q}
\newcommand{\ff}[3]{#1\colon #2\to #3}
\newcommand{\pare}[1]{\left(#1\right)}

\newcommand{\cat}[1]{\mathsf{#1}}
\newcommand{\id}[1]{1_{#1}}
\newcommand{\homc}[3]{\Hom_\cat #1\left(#2,#3\right)}
\newcommand{\homopp}[3]{\Hom_\opp #1\left(#2,#3\right)}
\newcommand{\homo}[2]{\Hom\left(#1,#2\right)}
\newcommand{\inv}[1]{#1^{-1}}
\newcommand{\invopp}[1]{#1^{\textbf{op}}}
\newcommand{\opp}[1]{\cat #1^{\textbf{op}}}
\newcommand{\fun}[1]{\mathcal{#1}}
\newcommand{\func}[3]{\fun{#1}\colon \cat #2\to \cat #3}
\newcommand{\nat}[3]{#1\colon #2\Rightarrow #3}

\newcommand\chap[1]{%
  \chapter*{#1}%
  \addcontentsline{toc}{chapter}{#1}}

\DeclareMathOperator{\sub}{Sub}
\DeclareMathOperator{\dems}{Dem}
\DeclareMathOperator{\Ob}{Ob}
\DeclareMathOperator{\set}{Set}
\DeclareMathOperator{\grp}{Grp}
\DeclareMathOperator{\topo}{Top}
\DeclareMathOperator{\mat}{Mat_{\mathbb K}}
\DeclareMathOperator{\cero}{\mathbf{0}}
\DeclareMathOperator{\uno}{\mathbf{1}}
\DeclareMathOperator{\dos}{\mathbf{2}}
\DeclareMathOperator{\poset}{Poset}
\DeclareMathOperator{\C}{\cat C}
\DeclareMathOperator{\F}{\fun F}
\DeclareMathOperator{\Hom}{Hom}
\DeclareMathOperator{\cring}{CRing}
\DeclareMathOperator{\mon}{Mon}
\DeclareMathOperator{\Nat}{Nat}
\DeclareMathOperator{\ev}{Ev}

\newcommand{\ob}[1]{\Ob(\cat{#1})}
\newcommand{\obopp}[1]{\Ob(\opp{#1})}


%Comandos en tikz
\newcommand{\df}[4]{\begin{tikzcd}[ampersand replacement=\&]
	{#1} \& {#2}
	\arrow["#3", shift left, from=1-1, to=1-2]
	\arrow["#4"', shift right, from=1-1, to=1-2]
\end{tikzcd}}
\newcommand{\diag}[8]{
\begin{tikzcd} [ampersand replacement=\&]
	#1 \&\& #2 \\
	\\
	#3 \&\& #4
	\arrow["#5", from=1-1, to=1-3]
	\arrow["#7"', from=1-1, to=3-1]
	\arrow["#8", from=1-3, to=3-3]
	\arrow["#6"', from=3-1, to=3-3]
\end{tikzcd}}


\title{Teorema del Punto Fijo de Lawvere}
\author{Alejandro Sánchez Goiz}
\date{\today}

\begin{document}

\maketitle

\tableofcontents


\chapter*{Introducci'on}
\addcontentsline{toc}{chapter}{Introducci'on}
\markboth{Introducci'on}{Introducci'on}

El desarrollo de la Teoría de Conjuntos por Georg Cantor, aunado a la evolución del pensamiento matemático del siglo XIX, dio pie a una clase de razonamientos lógicos que hoy llamamos \textbf{argumentos diagonales}.

Estos argumentos revelan propiedades estructurales profundas mediante mecanismos de autorreferencia. A fin de ilustrar la esencia de este procedimiento, comenzaremos mostrando uno de los ejemplos más conocidos: la demostración de que los números reales son más grandes que los números naturales.


\begin{teo*}[Teorema de la Diagonal de Cantor]
El intervalo real unitario $(0,1)$ no es numerable. Es decir, no existe una biyección $\ff{\phi}{\N}{(0,1)}$.
\end{teo*}

\begin{proof}
Procederemos por reducción al absurdo. Supongamos que el conjunto es numerable y que, por tanto, existe una biyección $\phi$ que nos permite ennumerar los elementos del intervalo como una sucesión $x_1, x_2, x_3, \dots$, donde $x_n = \phi(n)$.

Podemos representar cada número $x_n$ mediante su expansión decimal única:
\begin{align*}
    x_1 &= 0.a_{11} a_{12} a_{13} a_{14} \dots \\
    x_2 &= 0.a_{21} a_{22} a_{23} a_{24} \dots \\
    x_3 &= 0.a_{31} a_{32} a_{33} a_{34} \dots \\
    &\vdotswithin{=} \\
    x_n &= 0.a_{n1} a_{n2} a_{n3} a_{nn} \dots 
\end{align*}
Buscamos encontrar un número real $y \in (0,1)$ que no se encuentre en dicha lista. Para ello, definimos $y = 0.d_1d_2d_3\dots$, donde los dígitos $d_n$ están dados por la siguiente regla:
\[
    d_n = \begin{cases} 
    1 & \text{si } a_{nn} \neq 1 \\
    2 & \text{si } a_{nn} = 1 
    \end{cases}
\]
Observemos que, dada esta regla, para cualquier índice $n \in \N$, el $n$-ésimo dígito de nuestro nuevo número $y$ difiere del $n$-ésimo dígito del número $x_n$ de la lista (es decir, $d_n \neq a_{nn}$). 

Esto implica necesariamente que $y \neq x_n$ para todo $n$. Sin embargo, $y$ es un número real bien definido con expansión decimal de unos y dos, por lo que $y \in (0,1)$. Por tanto, hemos encontrado un número en el intervalo $(0,1)$ que no está en la imagen de $\phi$, contradiciendo la hipótesis de que $\phi$ era sobreyectiva. Por lo tanto, concluimos que el intervalo $(0,1)$ no es numerable.
\end{proof}

Podemos observar en la demostración anterior que, para construir el elemento $y$, recurrimos explícitamente a la diagonal principal de la matriz; es decir, nos fijamos en los términos de la forma $a_{nn}$.

\begin{center}
\tikzset{every picture/.style={line width=0.75pt}}

\begin{tikzpicture}[x=0.75pt,y=0.75pt,yscale=-1,xscale=1]
% Text Node
\draw (215,31.4) node [anchor=north west][inner sep=0.75pt]    {$ \begin{array}{l}
0.\textcolor[rgb]{0.82,0.01,0.11}{a_{11}} a_{12} a_{13} a_{14} a_{15} \cdots a_{1n} \cdots \\
0.a_{21}\textcolor[rgb]{0.82,0.01,0.11}{a_{22}} a_{23} a_{24} a_{25} \cdots a_{2n} \cdots \\
0.a_{31} a_{32}\textcolor[rgb]{0.82,0.01,0.11}{a_{33}} a_{34} a_{35} \cdots a_{3n} \cdots \\
0.a_{41} a_{42} a_{43}\textcolor[rgb]{0.82,0.01,0.11}{a_{44}} a_{45} \cdots a_{4n} \cdots \\
0.a_{51} a_{52} a_{53} a_{54}\textcolor[rgb]{0.82,0.01,0.11}{a_{55}} \cdots a_{5n} \cdots \\
\vdots \\
0.a_{n1} a_{n2} a_{n3} a_{n4} a_{n5} \cdots \textcolor[rgb]{0.82,0.01,0.11}{a_{nn}} \cdots \\
\vdots 
\end{array}$};
\end{tikzpicture}
\end{center}

La clave del argumento reside en el uso de esta información diagonal para definir un objeto que difiere sistemáticamente de todos los elementos listados. Observemos que el éxito de la construcción depende de la capacidad de \enquote{invertir} el valor encontrado en la diagonal. El apelar a esta estructura es lo que conocemos hoy como \textbf{argumento diagonal}.

Este argumento es generalizable y trasciende la geometría de una matriz infinita. En muchos contextos, la diagonal no es tan visible, pero el mecanismo subyacente es el mismo: la \textbf{autorreferencia}. 

Si trasladamos este argumento a la Teoría de Conjuntos intuitiva, sustituyendo la numeración por la relación de pertenencia, el argumento diagonal, en vez de producir un \enquote{nuevo elemento}, revela una contradicción en la estructura misma de la teoría. Esta contradicción fue descubierta por Bertrand Russell en 1901, y es conocida como la \textbf{Paradoja de Russell}. 

\subsection*{La Paradoja de Russell}

La Paradoja de Russell concierne a la ahora llamada \textit{Teoría Intuitiva de Conjuntos}, como fue inicialmente concebida por Georg Cantor. Esta teoría incial se fundamentaba en el principio de que cualquier propiedad lógica define un conjunto. Por ejemplo, la propiedad de ser un número par da pie al conjunto $A=\{2k\,|\, k\in \mathbb{N}\}$. Con esto en mente, para visualizar el problema necesitamos considerar conjuntos que puedan pertenecerse a sí mismos. Por ejemplo, el conjunto definido por la propiedad de \enquote{ser una idea} es, en sí mismo, una idea, y por lo tanto pertenece a sí mismo.

Consideremos el conjunto $R$, definido como la colección de todos los conjuntos que \textbf{no} son elementos de sí mismos:
\[R = \{X \mid X \notin X\}\]
Podemos entonces formular la siguiente pregunta: ¿Se pertenece $R$ a sí mismo?
Ante esto, tenemos dos posibilidades:
\begin{enumerate}
    \item $R \in R$. En este caso, tenemos que debe cumplir la condición que define al conjunto, es decir, tenemos $R \notin R$.
    \item $R \notin R$. En este otro caso, tenemos que cumple la propiedad que define a $R$ por lo que debe pertenecer al conjunto, es decir, $R \in R$.
\end{enumerate}
En ambos casos llegamos a la contradicción:
\[R \in R \iff R \notin R\]
Este resultado evidencia que la teoría intuitiva es inconsistente: no puede ser el caso que cualquier propiedad defina un conjunto. De permitirlo, el edificio lógico de la teoría se derrumbaría.

Ahora bien, hemos afirmado que en el corazón de esta paradoja reside la \textbf{autorreferencia} y que se trata de un argumento diagonal. La autoreferencia está presente para llegar a la Paradoja. Sin embargo, en la formulación anterior no construimos ninguna matriz ni apelamos a la diagonal. ¿Cómo es entonces que este ejemplo cae en un \enquote{argumento diagonal} como el que observamos con Cantor?

Imaginemos, conforme a la teoría intuitiva, que podemos enlistar \enquote{todos los conjuntos existentes}. Organicémoslos tanto vertical como horizontalmente para formar una tabla como la siguiente:

\begin{center}
$ \begin{matrix} 
& A & B & C & D & \dots \\
A & \cdot & \cdot & \cdot & \cdot & \\
B & \cdot & \cdot & \cdot & \cdot & \\
C & \cdot & \cdot & \cdot & \cdot & \\
\vdots & & & & & \ddots
\end{matrix} $
\end{center}

Asignemos valores de $1$ y $0$ de la siguiente forma: pondremos un $1$ si el conjunto de la fila pertenece al conjunto de la columna, y un $0$ en caso contrario. Supongamos, por ejemplo, que $A \notin A$ y $A \in B$. Los valores resultantes serían entonces $0$ y $1$, respectivamente. Nuestra matriz de pertenencia podría entonces lucir así:


\begin{center}
\tikzset{every picture/.style={line width=0.75pt}}

\begin{tikzpicture}[x=0.75pt,y=0.75pt,yscale=-1,xscale=1]
% Text Node
\draw (215,31.4) node [anchor=north west][inner sep=0.75pt]    {$
\begin{array}{rcccccc}
    & A & B & C & D & E & \cdots \\
  A & \mathbf{0} & 0 & 0 & 1 & 0 & \cdots \\
  B & 1 & \mathbf{1} & 0 & 1 & 0 & \cdots \\
  C & 1 & 0 & \mathbf{0} & 1 & 1 & \cdots \\
  D & 0 & 0 & 1 & \mathbf{0} & 1 & \cdots \\
  E & 1 & 0 & 1 & 0 & \mathbf{1} & \cdots \\
  \vdots & \vdots & \vdots & \vdots & \vdots & \vdots & \ddots
\end{array}
$};
\end{tikzpicture}
\end{center}



Al igual que en el teorema anterior, nuestra atención se centra en la \textbf{diagonal principal} que hemos resaltado. Las entradas con valor $0$ en la diagonal corresponden a los conjuntos $X$ tales que $X \notin X$.
La definición de $R$ es precisamente la instrucción de tomar esos casos: $R$ se define \enquote{invirtiendo} la diagonal principal.
\[ X \in R \iff \text{Diagonal}(X, X) = 0 \]
Dado que $R$ es un conjunto dentro de esta teoría, debe ocupar algún lugar en la lista, es decir, la fila $R$. ¿Qué valor tiene $R$ en su propia intersección diagonal? Por construcción, debe tener el valor opuesto al que tiene en la diagonal. Es decir, el valor en $(R, R)$ debe ser $0$ si y solo si es $1$.

Esta visualización confirma que la paradoja es estructuralmente idéntica al argumento de Cantor: el problema surge al intentar introducir dentro de la matriz un objeto definido por la negación de la diagonal de dicha matriz.

Siguiendo estos casos de autorreferencia, el más célebre y con mayor peso filosófico es indudablemente el \textbf{Primer Teorema de Incompletitud de Gödel}. Un planteamiento y demostración rigurosa de este teorema ocuparía demasiado espacio y se saldría de los objetivos de esta introducción. Por ello, nos limitaremos a dar un esbozo que sigue la línea argumental presentada por el propio Gödel en \cite{gdl}, resaltando cómo es, en el fondo, otro argumento diagonal.

\subsection*{El Teorema de Incompletitud de Gödel}


Los sistemas formales que fundamentan la matemática moderna, tales como los Axiomas de Peano o la Teoría de Conjuntos de Zermelo-Fraenkel, poseen la capacidad expresiva suficiente para construir la aritmética básica de los números naturales. La expectativa histórica, liderada por el programa de Hilbert, era que estos sistemas fueran capaces de decidir la verdad o falsedad de cualquier proposición formulada en su lenguaje.

Sin embargo, estos sistemas presentan limitaciones internas fundamentales respecto a lo que pueden demostrar. Gödel probó que, si estos sistemas son \textbf{consistentes} (que no poseen contradicciones), existen  entonces proposiciones en dichos sistemas que son \textit{indecidibles}; es decir, enunciados que el sistema no es capaz de demostrar como verdaderos ni refutar como falsos.

Podría pensarse que la solución radica en añadir estas proposiciones como nuevos axiomas. No obstante, esto no resuelve el problema estructural: en el nuevo sistema ampliado surgirían inevitablemente nuevas proposiciones indecidibles. Este fenómeno de incompletitud no es exclusivo de los dos sistemas mencionados; como veremos en el Capítulo 3, una clase extensiva de sistemas formales sufren de esta limitación.


En el corazón de este problema yace, nuevamente, la autorreferencia. Construimos una proposición que afirma algo sobre su propia indemostrabilidad. Para formalizar esto, Gödel tuvo que idear una manera de que la aritmética \enquote{hablara de sí misma}.

La idea central y genial de Gödel fue trasladar el problema de la autorreferencia al lenguaje de la aritmética. Para ello, ideó un método de codificación (conocido como \textit{numeración de Gödel}) que asigna a cada fórmula $\phi$ del sistema un número natural único, denotado por $\ulcorner \phi \urcorner$. Esto permite que la aritmética \enquote{hable} de sus propias fórmulas.

Gracias a esta codificación, la propiedad metamatemática \enquote{$y$ es una demostración de $x$} se puede expresar como una relación aritmética, llamémosla $\dems(x,y)$. A partir de aquí, podemos definir el concepto de ser demostrable como 

\begin{center}
    
$x$ es demostrable $ \iff \exists y \dems(x,y).$

\end{center}
Nace de aquí la diagonal como en los otros dos ejemplos. Gödel definió una función de sustitución $s(n, m)$, que calcula el código de la fórmula resultante al tomar la fórmula con código $n$ y reemplazar su variable libre por el número $m$, es decir, 
\[ s(\ulcorner \phi(x) \urcorner, m) = \ulcorner \phi(m) \urcorner. \]
Esta función $s$ es la herramienta que nos permite \enquote{caminar por la diagonal}: si evaluamos $s(n, n)$, estamos pidiéndole intrínsecamente a la fórmula número $n$ que hable del número $n$. Es el análogo aritmético de mirar la entrada $(n,n)$ en la matriz del argumento de la diagonal de Cantor o en la matriz que visualizamos de pertencencia para la Paradoja de Russell.

Ahora, consideremos la siguiente fórmula:
\[ P(x) := \neg \dems( s(x, x) ). \]
Sea $k$ el número de Gödel de esta fórmula, es decir, $k = \ulcorner P(x) \urcorner$. Si evaluamos ahora la diagonal en este punto (calculamos $P(k)$), obtenemos una sentencia $G$ tal que:
\[ G \iff \neg \dems( s(k, k) ) \]
Pero por definición de nuestra función de sustitución, $s(k, k)$ es precisamente el código de $G$. Es decir, $\ulcorner G \urcorner = s(k, k)$. Sustituyendo esto, llegamos a:
\[ G \iff \neg \dems( \ulcorner G \urcorner ) \]
Esta sentencia $G$ afirma, esencialmente: \enquote{yo no soy demostrable}.

Si el sistema es consistente, no puede demostrar falsedades, por lo que no puede demostrar $G$ (ya que si lo hiciera, $G$ sería falsa). Pero al no poder demostrarla, lo que $G$ afirma es verdad. Tenemos entonces una sentencia que es verdadera pero no demostrable dentro del sistema.

Para visualizar la naturaleza diagonal de este argumento, podemos recurrir a la misma representación matricial que utilizamos con Cantor y Russell.

Imaginemos una lista enumerada de todas las fórmulas posibles con una variable libre en el lenguaje de la aritmética: $\phi_0(x), \phi_1(x), \phi_2(x), \dots$.
En las columnas, colocamos los números naturales que pueden servir como argumentos para estas fórmulas.

Definamos el valor de la entrada $(n, m)$ en esta matriz como $1$ si el sistema demuestra que la fórmula $\phi_n$ es verdadera para el número $m$ (es decir, si $\vdash \phi_n(m)$), y $0$ en caso contrario.

\begin{center}
\tikzset{every picture/.style={line width=0.75pt}}

\begin{tikzpicture}[x=0.75pt,y=0.75pt,yscale=-1,xscale=1]
% Text Node
\draw (215,31.4) node [anchor=north west][inner sep=0.75pt]    {$
\begin{array}{rcccccc}
    & 0 & 1 & 2 & 3 & 4 & \cdots \\
  \phi_0 & \mathbf{1} & 0 & 1 & 0 & 0 & \cdots \\
  \phi_1 & 0 & \mathbf{0} & 1 & 1 & 0 & \cdots \\
  \phi_2 & 1 & 1 & \mathbf{0} & 0 & 1 & \cdots \\
  \phi_3 & 0 & 0 & 1 & \mathbf{1} & 0 & \cdots \\
  \vdots & \vdots & \vdots & \vdots & \vdots & \mathbf{\ddots} & 
\end{array}
$};
\end{tikzpicture}
\end{center}

Nuevamente, nuestra atención se dirige a la \textbf{diagonal principal}. Esta diagonal representa si es demostrable cada fórmula cuando es aplicada a su propio número de código o dicho de otra forma: la entrada $(n, n)$ en la matriz nos dice si $\vdash \phi_n(n)$.

La sentencia indecidible de Gödel, $G$, se construye esencialmente definiendo una nueva fila (una nueva fórmula) que \enquote{invierte} los valores de esta diagonal. La fórmula $G(x)$ está diseñada para comportarse como la negación de la diagonal:
\[ G(n) \text{ es verdadera} \iff \text{la entrada} (n,n) \text{de la matriz es } 0 \]
Como $G$ es una fórmula válida en el sistema, debe aparecer en algún lugar de la lista de filas, digamos en la fila $k$. La contradicción (o en este caso, la indecidibilidad) surge al preguntar por el valor de la intersección $(k, k)$. La fórmula debe negar su propio valor en la diagonal, llegando así a que es verdadera si y solo si no es demostrable.

A pesar de que los tres ejemplos expuestos difieren sustancialmente en su contenido proposicional —el primero trata sobre cardinalidades infinitas, el segundo sobre inconsistencias en la teoría intuitiva de conjuntos y el tercero sobre las limitaciones deductivas de la aritmética—, todos comparten un método subyacente idéntico en esencia. En cada caso, la motivación principal surge de una noción de \textbf{autorreferencia} que es \enquote{transmitida} a la demostración matemática.

 Es precisamente este rasgo autorreferencial el que ha rodeado de un halo de misticismo y especulación filosófica a los Teoremas de Incompletitud de Gödel y, en general, a muchos resultados obtenidos mediante argumentos diagonales. Surge entonces una interrogante natural: si el germen de la autorreferencia puede incrustarse en el corazón de las matemáticas mediante un mecanismo tan simple, ¿acaso hay una noción más allá de lo científico que las limite?

Esta inquietud fue popular durante gran parte del siglo XX, impulsada por la crisis de los fundamentos. Sin embargo, en 1969, William F. Lawvere publicó unas notas que, silenciosa pero contundentemente, derrumbarían este misticismo. En su artículo \textit{Diagonal Arguments and Cartesian Closed Categories}, Lawvere observó que todos estos argumentos no son sino consecuencias naturales de trabajar en tipos particulares de estructuras categóricas. En sus propias palabras:

\begin{quote}
The original aim of this article was to demystify the incompleteness theorem of Gödel and the truth-definition theory of Tarski by showing that both are consequences of some very simple algebra in the cartesian-closed setting.
\end{quote}

El aporte de Lawvere es fundamental: logró unificar en un solo esquema formal algunos de los resultados de imposibilidad más importantes del siglo pasado. Más adelante, este hallazgo sería conocido como el \textbf{Teorema del Punto Fijo de Lawvere}, uniéndose a la vasta familia de teoremas de punto fijo que abundan en diversas áreas de la matemática. 

No obstante, y a pesar de su trascendencia unificadora, este teorema permanece sorprendentemente poco difundido. Una posible explicación radica en que el lenguaje natural del resultado, la Teoría de Categorías, es una disciplina relativamente joven y, en ocasiones, ajena a la formación matemática tradicional. Por ello, existe un vacío en la literatura divulgativa y educativa respecto a este resultado. El presente trabajo nace de la intención de llenar ese vacío, presentando el Teorema del Punto Fijo de Lawvere de manera autocontenida y accesible, sin asumir conocimientos previos en Teoría de Categorías.

Como el presente trabajo es autocontenido, en el \textbf{Capítulo 1} nos enfocamos en introducir los conocimientos de Teoría de Categorías necesarios. Dado que no se trata de un curso extensivo de Teoría de Categorías, nos limitaremos a los conceptos indispensables para llegar al resultado deseado. El lector interesado en profundizar en esta teoría puede consultar \cite{Rie16}, \cite{Lei16} y \cite{Mac97}.

El \textbf{Capítulo 2} aborda el núcleo del trabajo: el \textbf{Teorema del Punto Fijo de Lawvere}. Presentamos una versión modernizada y una demostración más detallada que la original de Lawvere en \cite{Law06}. Es en este capítulo donde introducimos el concepto de \textit{categoría cartesiana cerrada}, enfatizando su papel crucial como el entorno donde estos argumentos diagonales ocurren. Además, veremos aquí también la contrapositiva del teorema, que es en esencia el resultado que da pie a estos argumentos diagonales.

El \textbf{Capítulo 3} explora las aplicaciones del teorema a través de diversas áreas de la matemática. Este es el capítulo más extenso, pues el interés principal de este trabajo es demostrar la utilidad y extensividad del resultado de Lawvere. 

El \textbf{Capítulo 4} presenta las conclusiones del trabajo.

La bibliogafía usada para esta tesis se podrá encontrar al final de la misma.



\chapter{Teoría de Categorías}
\section{Categor'ias y tipos de morfismos}

En el estudio moderno de las matemáticas, la mayoría de áreas empiezan a través de la teoría de conjuntos. Así, por ejemplo, un grupo es un conjunto con cierta estructura añadida; una topología es un
 conjunto y un subconjunto del conjunto potencia con ciertos axiomas a cumplir; incluso un sistema formal parte de conjuntos de símbolos, axiomas, reglas de formación y de inferencia. El lenguaje 
 matemático es entonces, por lo general, uno que nace de la visión conjuntista. Sin embargo, el teorema del punto fijo de Lawvere sólo puede aplicarse en categorías cartesianas cerradas. 
 Por tanto, todos los ejemplos que veremos en el capítulo 3 serán tratados desde la visión categórica, no la conjuntista. Es necesario entonces introducir los conceptos básicos de la teoría de 
 categorías para poder llegar al teorema deseado y poder visualizar los ejemplos a mostrar desde una nueva percepción, i.e., la categórica. El punto natural para iniciar es con la noci'on de categor'ia,
d'andonos nuestra primera


\begin{defi}[Categoría]
    Una \emph{categoría} $\cat C$ consta de:
    \begin{itemize}
        \item Una colección de objetos, simbolizada por $\ob C$, 
        \item Para cada $A, B$ en $\ob C$, una colección de morfismos $f\colon A\to B$, donde $A$ es denominado el \textbf{dominio} y $B$ el \textbf{codominio}.
        \item Para cada $A$ en $\ob C$, un \textbf{morfismo identidad}, denotado por $\id A$, con dominio y codominio $A$.  
        \item Para cada par de morfismos $f\colon A\to B$, $g\colon B\to C$ un \textbf{morfismo composici'on}, $gf\colon A\to C$, tambi'en denominado el \textbf{compuesto de $f$ con $g$}.

    \end{itemize}

    tales que cumplen los siguientes dos axiomas:

    \begin{itemize}
        \item \textbf{Identidad:} Para cada $f\colon A\to B$, se cumple que $\id A f = f = f \id B$.
        \item \textbf{Asociatividad:} Para cualesquiera $f\colon A\to B$, $g\colon B\to C$, $h\colon C\to D$, tenemos que $h(gf) = (hg)f$.
    \end{itemize}

\end{defi}

    Para denotar objetos usamos intercambiablemente letras may'usculas o letras min'usculas, seg'un sea m'as conveniente en el contexto que se usen. 

    A los morfismos $f\colon A \to B$ tambi'en los denominamos como \textbf{flechas} o \textbf{mapas} entre objetos. A la colección de morfismos entre dos objetos $A$ y $B$ de una categor'ia $\cat C$ 
	la simbolizamos por $\homc{C}{A}{B}$ o bien simplemente $\homo{A}{B}$ si no existe confusi'on sobre la categoría a la que pertenecen. Adicionalmente, los morfismos $f\colon A\to B$ tambi'en los
	 podemos representar como $A \xlongrightarrow{f} B$. 

    En el estudio de la teor'ia de categor'ias nos veremos constantemente usando \emph{diagramas}. Someramente, un diagrama es una representaci'on visual de objetos y flechas entre estos objetos.
	 Pensado como una gr'afica, los objetos actuan como v'ertices y las flechas como aristas. Existe, como veremos a lo largo de este cap'itulo, una ventaja en representar cuestiones de la teor'ia
	  a trav'es de diagramas. Usualmente, nos permitir'a una nueva percepci'on del problema y puede facilitar nuestro entendimiento de 'este. As'i, por ejemplo, la composici'on $gf\colon A\to B$
	 puede ser representado como la afirmaci'on de que el siguiente diagrama conmuta:

% https\colon//q.uiver.app/#q=WzAsMyxbMSwwLCJCIl0sWzAsMSwiQSJdLFsyLDEsIkMiXSxbMSwyLCJnZiIsMl0sWzEsMCwiZiJdLFswLDIsImciXV0=
\[\begin{tikzcd}
	& B \\
	A && C.
	\arrow["g", from=1-2, to=2-3]
	\arrow["f", from=2-1, to=1-2]
	\arrow["gf"', from=2-1, to=2-3]
\end{tikzcd}\]

    Cuando decimos que el diagrama conmuta, nos referimos a que es lo mismo tomar el camino superior que el inferior, es decir, del objeto $A$ pasar al objeto $B$ a trav'es de $f$ y del objeto $B$ 
	pasar a trav'es de $g$ al objeto $C$, que simplemente pasar del objeto $A$ al objeto $C$ a trav'es de $gf$.

    Para ejemplificar mejor, mostramos los axiomas de identidad y asociatividad a trav'es de diagramas. 

    \textbf{Axioma de identidad} 

    Para toda $f\colon A\to B$, el siguiente diagrama es conmutativo 

    % https\colon//q.uiver.app/#q=WzAsNCxbMCwwLCJBIl0sWzIsMCwiQSJdLFsyLDIsIkIiXSxbNCwyLCJCIl0sWzAsMSwiXFxpZCBBIl0sWzEsMiwiZiJdLFsyLDMsIlxcaWQgQiJdLFswLDIsImYiLDJdLFsxLDMsImYiXV0=
\[\begin{tikzcd}
	A && A \\
	\\
	&& B && B.
	\arrow["{\id A}", from=1-1, to=1-3]
	\arrow["f"', from=1-1, to=3-3]
	\arrow["f", from=1-3, to=3-3]
	\arrow["f", from=1-3, to=3-5]
	\arrow["{\id B}", from=3-3, to=3-5]
\end{tikzcd}\]
\newpage
\textbf{Axioma de asociatividad}

Para cualesquiera, $f\colon A\to B$, $g\colon B\to C$ y $C\to D$, el siguiente diagrama es conmutativo. 

% https\colon//q.uiver.app/#q=WzAsNCxbMCwwLCJBIl0sWzMsMCwiQiJdLFszLDIsIkMiXSxbMCwyLCJEIl0sWzAsMSwiZiJdLFsxLDIsImciXSxbMiwzLCJoIl0sWzAsMywiaChnZik9KGhnKWYiLDEseyJzdHlsZSI6eyJib2R5Ijp7
%Im5hbWUiOiJkb3R0ZWQifX19XSxbMCwyLCJnZiIsMSx7ImxhYmVsX3Bvc2l0aW9uIjozMCwic3R5bGUiOnsiYm9keSI6eyJuYW1lIjoiZGFzaGVkIn19fV0sWzEsMywiaGciLDEseyJsYWJlbF9wb3NpdGlvbiI6MzAsInN0eWxlIjp7ImJvZHki
%OnsibmFtZSI6ImRhc2hlZCJ9fX1dXQ==
\[\begin{tikzcd}
	A &&& B \\
	\\
	D &&& C.
	\arrow["f", from=1-1, to=1-4]
	\arrow["{h(gf)=(hg)f}"{description}, dotted, from=1-1, to=3-1]
	\arrow["gf"{description, pos=0.3}, dashed, from=1-1, to=3-4]
	\arrow["hg"{description, pos=0.3}, dashed, from=1-4, to=3-1]
	\arrow["g", from=1-4, to=3-4]
	\arrow["h", from=3-4, to=3-1]
\end{tikzcd}\]

Dos observaciones son importantes en la definici'on de categor'ia. La primera, la visi'on conjuntista habitual nos har'ia pensar que los objetos son en realidad conjuntos y los morfismos, funciones.
 Es importante observar que esta visi'on puede privar a uno de una visi'on m'as amplia que ofrece 'esta teor'ia y que, en lo general, no es cierto que se pueda hacer dicha comparaci'on. 

La segunda observación, se hace menci'on de \emph{colecciones}, no de conjuntos en la definición. Esto es porque en principio las categor'ias que se manejan pueden recolectar una cantidad tan grande
 de objetos y morfismos que no pueden ser conjuntos. Sin embargo, hay casos en que podemos hablar de conjuntos en vez de colecciones para los morfismos de una categor'ia, y la distinci'on es lo
  suficientemente importante para valer una definici'on:

\begin{defi} Sea $\cat C$ una categor'ia. Decimos que es 
\begin{enumerate}
    \item \textbf{Localmente pequeña} si para cada par de objetos $A,B$ en $\ob C$, $\homc{C}{A}{B}$ es un conjunto. 
    \item \textbf{Pequeña} si tanto la colección de sus objetos como la colección de todas sus flechas son conjuntos. 
\end{enumerate}   
\end{defi}
 
Exhibimos algunos ejemplos de categor'ias. 

\begin{ejems} \leavevmode 
    \begin{enumerate}[label=\alph*)]
        \item $\set$ es la categor'ia cuyos objetos son conjuntos y los morfismos son funciones entre conjuntos. 
        \item $\grp$ es la categor'ia cuyos objetos son grupos y los morfismos son homomorfismos entre grupos. Del mismo modo podemos definir categor'ias para grupos abelianos, Ab, para anillos unitarios,
		 Ring, etc.
        \item Un s'olo grupo $G$ puede ser visto como una categor'ia $\cat G$, que consta de un 'unico objeto $*$ y cuyos morfismos son todos los elementos del grupo. As'i, la composici'on est'a dada por la
		 operaci'on del grupo y el morfismo identidad de $\cat G$ es el elemento neutro del grupo. 
        \item $\topo$ es la categor'ia cuyos objetos son espacios topol'ogicos y los morfismos son las funciones continuas entre espacios topol'ogicos. 
        \item $\topo_*$ es la categor'ia cuyos objetos son espacios topol'ogicos con un punto base y los morfismos son funciones continuas que preservan puntos base. 
        \item $\mat$ es la categor'ia cuyos objetos son n'umeros naturales y cuyos morfismos son matrices con entradas elementos de un campo $\mathbb K$, es decir, dados n'umeros naturales $m,n$ un
		 morfismo $A\colon m\to n$ es una matriz de $m\times n$ con entradas en $\mathbb K$. La composici'on est'a dada por multiplicaci'on de matrices, y los morfismos identidad son las matrices identidad. 
        \item Dado un orden parcial $(P, \leq)$ podemos tratar 'este como una categor'ia $\cat P$. Sus objetos son los elementos del orden parcial y dados dos objetos $a,b\in \ob P$, decimos que existe 
		un morfismo $a\to b$ si y solo si $a\leq b$. Por la propiedad reflexiva del orden parcial, $a\leq a$ para todo $a\in \ob P$ por lo que existen morfismos identidad para cada objeto. La propiedad 
		transitiva del orden define la composici'on de morfismos. 
        \item La categoría $\poset$ tiene como objetos 'ordenes parciales $(P,\leq)$ y como morfismos funciones que preservan 'ordenes, es decir, tales que si $(P, \leq_P)$ y $(Q,\leq_Q)$ son 'ordenes
		 parciales, $f\colon (P,\leq_P)\to (Q,\leq_q)$ es una funci'on tal que para todo $p, q\in P$, $fp\leq_Q fq$ si y s'olo si $p\leq_P q$.

    \end{enumerate}
    \end{ejems}

En contraste a la teor'ia de conjuntos, aqu'i no hablamos de elementos. La pertenencia como tal no es propia en el manejo de una categor'ia. Es decir, si consideramos un objeto $X$ en la categor'ia 
$\set$, no podemos hablar de un elemento $a\in X$. Existe, en este sentido, una forma de hablar de elementos y pertenencia a trav'es de morfismos. Para ello consideramos las siguientes definiciones:

\begin{defi}
    Sea $\cat C$ una categoría. Decimos que un objeto es:
    \begin{enumerate}
        \item \textbf{Terminal} (denotado usualmente por $\uno$), si para cualquier objeto $X$ en $\cat C$ existe un \emph{único} morfismo $X \to \uno$.
        \item \textbf{Inicial} (denotado usualmente por $\cero$), si para cualquier objeto $X$ en $\cat C$ existe un \emph{único} morfismo $\cero \to X$.
    \end{enumerate}
\end{defi}


\begin{ejems} \leavevmode 
\begin{enumerate}[label=\alph*)]
    \item En la categoría $\set$, cualquier conjunto unitario es un objeto terminal. Denotamos para esta categoría $\uno = \{\varnothing\}$. Por otro lado, el único objeto inicial es el conjunto
	 vacío $\varnothing$, que denotamos como $\cero$; por vacuidad sólo existe una función con dominio $\varnothing$ y codominio $X$ para cualquier conjunto $X$.

    \item En un orden parcial $(P,\leq)$, un objeto inicial $\cero$ es un elemento tal que para todo $x \in P$, existe una única flecha $\cero \to x$, lo cual es equivalente a decir $\cero \leq x$.
	 Es decir, el objeto inicial es el \textbf{mínimo} del orden. Análogamente, un objeto terminal $\uno$ satisface $x \leq \uno$ para todo $x$, correspondiendo al \textbf{máximo} del orden.
\end{enumerate}
\end{ejems}

Observemos en $\set$ que seleccionar un elemento $a$ de un conjunto $A$ es equivalente a definir una función $a\colon \uno \to A$ cuya imagen es precisamente $a$. Esto nos permite reinterpretar 
la noción conjuntista de \enquote{pertenencia} usando únicamente flechas y objetos terminales, para generalizarla a cualquier categoría con objeto terminal.

\begin{defi}[Elemento Global] \label{elemento global}
    Sea $\cat C$ una categoría con objeto terminal $\uno$. Un \textbf{elemento global} de un objeto $A$ es un morfismo $x\colon \uno \to A$.
\end{defi}

Bajo esta óptica, en la categoría $\set$, la expresión $a \in A$ se traduce al lenguaje categórico como la existencia de un morfismo $a\colon \uno \to A$.

Pensando ahora en los morfismos, hay propiedades interesantes de estos que merecen ser distinguidas. Tenemos las siguientes definiciones. 

\begin{defi}
    Decimos que un morfismo $f\colon A\to B$ es 
    \begin{enumerate}
        \item \textbf{monomorfismo} o \textbf{mono} si para cada par de morfismos $h,g \colon C\to A$,\,  $fg=fh$ implica $g=h$. Es decir, si el siguiente diagrama conmuta 
        % https\colon//q.uiver.app/#q=WzAsMyxbMCwwLCJDIl0sWzIsMCwiQSJdLFs0LDAsIkIiXSxbMCwxLCJnIiwwLHsib2Zmc2V0IjotMn1dLFsxLDIsImYiXSxbMCwxLCJoIiwyLHsib2Zmc2V0IjoyfV1d
\[\begin{tikzcd}
	C && A && B
	\arrow["g", shift left=2, from=1-1, to=1-3]
	\arrow["h"', shift right=2, from=1-1, to=1-3]
	\arrow["f", from=1-3, to=1-5]
\end{tikzcd}\]
entonces, $g=h$.
        \item \textbf{epimorfismo} o $\textbf{epi}$ si para cada par de morfismos $h,g\colon B\to C$,\, $gf=hf$ implica $g=h$. Es decir, si el siguiente diagrama conmuta, 
        % https\colon//q.uiver.app/#q=WzAsMyxbMCwwLCJDIl0sWzIsMCwiQiJdLFs0LDAsIkEiXSxbMSwwLCJnIiwyLHsib2Zmc2V0IjoyfV0sWzIsMSwiZiIsMl0sWzEsMCwiaCIsMCx7Im9mZnNldCI6LTJ9XV0=
\[\begin{tikzcd}
	C && B && A
	\arrow["g"', shift right=2, from=1-3, to=1-1]
	\arrow["h", shift left=2, from=1-3, to=1-1]
	\arrow["f"', from=1-5, to=1-3]
\end{tikzcd}\]
entonces $g=h$.
        \item \textbf{invertible} si existe $g\colon B\to A$ tal que $fg= \id B$ y $gf= \id A$.  Es decir, existe $g$ tal que el siguiente diagrama conmuta 
        % https\colon//q.uiver.app/#q=WzAsNSxbMCwwLCJBIl0sWzIsMCwiQiJdLFswLDIsIkEiXSxbMiwyLCJCIl0sWzIsMywiQiJdLFswLDEsImYiXSxbMCwyLCJcXGlkIEEiXSxbMSwzLCJcXGlkIEIiLDJdLFsyLDMsImYiLDJdLFsxLDIsImciLDFdXQ==
\[\begin{tikzcd}
	A && B \\
	\\
	A && B. \\
	\arrow["f", from=1-1, to=1-3]
	\arrow["{\id A}"', from=1-1, to=3-1]
	\arrow["g"{description}, from=1-3, to=3-1]
	\arrow["{\id B}", from=1-3, to=3-3]
	\arrow["f"', from=3-1, to=3-3]
\end{tikzcd}\]

En este caso decimos que $f$ es un \textbf{isomorfismo} y que $A$ y $B$ son \textbf{isomorfos}.
    \end{enumerate}
\end{defi}
\begin{ejems}  \label{iso} \leavevmode
    
    \begin{enumerate}[label=\alph*)]
        \item En $\set$ los monomorfismos son las funciones inyectivas, los epimorfismos son las funciones suprayectivas y los isomorfismos son las funciones biyectivas. 
        \item En $\topo$ los isomorfismos son los homeomorfismos. 
        \item En un grupo $G$ considerado como categor'ia, todos los morfismos son isomorfismos. Recordemos que un grupo visto como categor'ia consta de un 'unico objeto $*$ y los elementos del
		 grupo act'uan como morfismos, mientras que el elemento neutro actua como el morfismo identidad del objeto $G$. Recordemos que para cada elemento $g$ del grupo $G$, existe un inverso $h$
		  bajo la operaci'on del grupo tal que $g\cdot h = e$, con $\cdot$ la operaci'on del grupo y $e$ el elemento neutro. De esto vemos que cada morfismo tiene un inverso que da el morfismo
		   identidad, por lo que todos son isomorfismos. 
    \end{enumerate}
\end{ejems}

Es crucial aquí hacer una observación. En la teor'ia de conjuntos, una funci'on biyectiva es tanto suprayectiva como inyectiva. Conversamente, si una funci'on es tanto suprayectiva como inyectiva,
 es biyectiva. En teor'ia de categor'ias el an'alogo es cierto para la primera parte: si un morfismo es isomorfismo es tambi'en epimorfismo y monomorfismo. Sin embargo, el converso no es cierto: 
 un morfismo que es tanto epi como mono no implica que sea isomorfismo. Para ver esto, consideremos el siguiente ejemplo.

\begin{ejem} 
	\label{ejemplo orden parcial mono y epi}


    Consideremos nuevamente un orden parcial $(P, \le)$ visto como categoría. Recordemos que una flecha $a\to b$ existe si y solo si $a\le b$. Por tanto, entre dos objetos $a,b$ existe a lo sumo
	 una flecha. De este hecho tenemos que por vacuidad se da que toda flecha es tanto epimorfismo como monomorfismo: las premisas $fg=fh$ y $gf=hf$ implican trivialmente $g=h$ pues entre dos objetos
	 la flecha es única. Por otro lado, si $a\to b$ es un isomorfismo debe primeramente existir $b\to a$. Esto implica $a\le b$ y $b\le a$. Por antisimetr'ia del orden parcial tenemos entonces
	 que $a=b$. Por tanto, los únicos isomorfismos en $(P,\leq)$ son las flechas identidad $a\to a$.
	De esto vemos que en un orden parcial considerado como categoría, toda flecha es epi y mono, pero solo las identidades son isomorfismos. 

\end{ejem}

Basta considerar $P=\{1,2\}$ con el orden usual para ver un ejemplo concreto de lo anterior. En este caso, la flecha $1\to 2$ es tanto epi como mono, pero no es isomorfismo pues no existe flecha $2\to 1$.

Por otro lado, mencion'abamos que el hecho de ser isomorfismo implica ser epi y mono. Demostramos esta afirmaci'on.

\begin{prop} Si $f\colon A\to B$ es un isomorfismo, entonces es tambi'en un epimorfismo y un monomorfismo. 
\end{prop}
\begin{proof}
    Demostremos primero que es un monomorfismo. Sean entonces $g,h\colon C\to A$ morfismos tales que $fg=fh$. Como $f$ es isomorfismo, existe $\inv f\colon B\to A$ tal que $\inv f f = \id A$. Tenemos entonces 
    \[
        g = \id A g = (\inv f f) g=  \inv f(fg) = \inv f (fh) = (\inv f f) h = \id A h = h 
   \]
   por lo tanto $g=h$. Probar que $f$ es epimorfismo es completamente an'alogo a como procedimos. 
\end{proof}
En el uso de la prueba anterior usamos la notaci'on $\inv f$ para hablar de un morfismo inverso para $f$. Esto se debe a que dicho morfismo inverso, de existir, es 'unico. Demostramos esta afirmaci'on. 

\begin{prop}
    Si $f$ es un isomorfismo, entonces existe un 'unico morfismo inverso. 
\end{prop}
\begin{proof}
    Sea $f\colon A\to B$ isomorfismo y $g,h\colon B\to A$ inversas de $f$. Tenemos entonces 
    \[g = g\id B = g(fh) = (gf)h = \id A h = h \]
    Por lo tanto $g=h$.
\end{proof}
Por tanto nuestra notaci'on $\inv f$ esta debidamente justificada. 

Siguiendo la comparación entre monomorfismos con funciones inyectivas y epimorfismos con funciones suprayectivas, es importante mencionar otra distinción que existe entre estas nociones 
en teor'ia de categor'ias. En teor'ia de conjuntos, una funci'on $\ff{f}{A}{B}$ es inyectiva si y solo si existe una funci'on \emph{ inversa izquierda}, es decir, una funci'on $\ff{g}{B}{A}$ tal que $gf = \id A$.
De manera similar, una funci'on $\ff{f}{A}{B}$ es suprayectiva si y solo si existe una funci'on \emph{inversa derecha}, es decir, una funci'on $\ff{h}{B}{A}$ tal que $fh = \id B$. En teor'ia de categor'ias, estas 
 afirmaciones no son ciertas en general: un monomorfismo no necesariamente tiene una  inversa izquierda y un epimorfismo no necesariamente tiene una inversa derecha. El ejemplo $P=\{1,2\}$ 
  que vimos en ~\ref{ejemplo orden parcial mono y epi} sirve tambi'en para ilustrar este hecho. La flecha $1\to 2$ es tanto epi como mono, pero no tiene ni izquierda ni derecha inversa pues no 
  existe flecha $2\to 1$. Sin embargo, si un morfismo tiene una  inversa izquierda, es un monomorfismo; de manera similar, si un morfismo tiene una inversa derecha, es un epimorfismo. Para 
  ver esto, primero damos la siguiente definici'on:

\begin{defi}[Retracci'on y secci'on]
	Sean $x\xlongrightarrow{s}y\xlongrightarrow{r}x$ morfismos en una categor'ia $\cat C$ tal que $rs=\id x$. Decimos que $r$ es una \textbf{retracci'on} o \textbf{inversa izquierda} de $s$ y que $s$ es 
	una \textbf{secci'on} o \textbf{inversa derecha} de $r$.
\end{defi}  

\begin{teo} Sean $s\colon x\to y$ y $r\colon y\to x$ morfismos en una categor'ia $\cat C$ tal que $rs=\id x$. Entonces, $s$ es un monomorfismo y $r$ es un epimorfismo.
	\begin{proof}
		Demostramos primero que $s$ es un monomorfismo. Sean $f,g\colon z\to x$ morfismos tales que $sf=sg$. Tenemos 
		\[
			f = \id x f = (rs)f = r(sf) = r(sg) = (rs)g = \id x g = g
		\]
		Por lo tanto, $f=g$ y $s$ es un monomorfismo. 
		
		La demostraci'on de que $r$ es un epimorfismo es completamente an'aloga.
	\end{proof}

\end{teo}

En virtud de este resultado, decimos que $s$ es un \emph{monomorfismo escindido} y que $r$ es un \emph{epimorfismo escindido}.

Concluimos esta secci'on construyendo una categor'ia a partir de una  $\cat C$. Dada una flecha $a\to b$, podemos pensar en el proceso de invertir la direcci'on de 'esta. En este caso, tendr'iamos $b\to a$. El proceso de \enquote{revertir todas las flechas} produce lo que conocemos como un proceso \emph{dual}. Este procedimiento nos permite definir la siguiente categor'ia: 

\begin{defi} [Categor'ia Opuesta]
    Sea $\cat C$ una categor'ia. La categor'ia opuesta $\opp C$ tiene:
    \begin{itemize}
        \item como objetos $\obopp C$ los mismos objetos que $\ob C$,
        \item como morfismos los mismos que en $\cat C$ pero con el orden de las flechas invertidos, dado un morfismo $f\colon A\to B$ en $\cat C$, el morfismo correspondiente en $\opp C$ es $\invopp f \colon B\to A$. 
        \item como morfismos identidad los mismos morfismos identidad que en $\cat C$.
        \item la composici'on en $\opp C$ funciona igual que en $\cat C$. Dados morfismos $f\colon A\to B$, $g\colon B\to C$ su composici'on $gf\colon A\to C$ da lugar la composici'on $\invopp{(gf)} = \invopp f \invopp g\colon C\to A$.
    \end{itemize}
\end{defi}
Es evidente que los axiomas de identidad y asociatividad se siguen en $\opp C$ si y solo si la $\cat C$ es una categor'ia. Esta construcci'on y, en general, las construcciones duales que veremos m'as adelante ser'an de gran utilidad para el teorema del punto fijo de Lawvere. 



%Aqui finaliza la primera seccion 
%--------------------------------------------------------------------------------------------------

\section{Funtores y Transformaciones Naturales}

La noci'on de categor'ia nos invita a pensar en objetos matem'aticos y los morfismos que permiten preservar su estructura. Consideramos por ejemplo grupos. Los morfismos que preservan su estructura son los homomorfismos; pensamos en espacios topol'ogicos, cuyos morfismos que preservan estructura son funciones continuas. Una categor'ia es tambi'en un objeto matem'atico, por lo que podemos preguntarnos en qu'e morfismos preservan la estructura de una categor'ia. Los funtores son precisamente estos morfismos. 

\begin{defi}[Funtor]\label{funtor}
Sean $\cat C$, $\cat D$ categor'ias. Un \textbf{funtor} $\func {F}{C}{D}$ consta de 
\begin{itemize}
    \item objetos $\fun F C$ en $\ob D$ para cada $C$ en $\ob C$,
    \item morfismos $\fun F f\colon \fun F A\to \fun F B$ en $\cat D$ para cada morfismo $f\colon A\to B$ en $\cat C$. Es decir, para cada colecci'on $\homc{C}{A}{B}$ corresponde una colecci'on $\homc{D}{\fun FA}{\fun FB}$.
\end{itemize}
    Tales que:
    \begin{itemize}
        \item para cada $A$ en $ \ob C$, $\fun F(\id A) = \id{\fun F A}$,
        \item para cada $f\colon A\to B$, $g\colon B\to C$ en $\cat C$, $\fun F g\fun F f = \fun F(gf)$
    \end{itemize}
\end{defi}
\begin{ejems} \label{ejemplos funtores}
    \leavevmode
    \begin{enumerate}
        \item El funtor $\fun F\colon \grp\to \set$ llamado \emph{olvidadizo} pues \enquote{olvida} la estructura de grupo y deja tan solo al conjunto de elementos; olvida tambi'en que las funciones son homomorfismos y simplemente las convierte a funciones sin esa noci'on. 
        \item  $\fun P\colon \set \to \set$ es un funtor que manda a cada conjunto $X$ a su conjunto potencia $P(X)$ y que manda toda funci'on $f\colon X\to Y$ a $\fun Pf\colon P(X)\to P(Y)$, esta funci'on manda a cada $A\subset X$ a la imagen $f(A)\in P(Y)$.
        \item Dados 'ordenes parciales $(P,\leq_P)$, $(Q,\leq_Q$) un funtor $\fun F\colon P\to Q$ es una funci'on que preserva 'ordenes, es decir, manda objetos $p, q\in P$ a objetos $\fun F p, \fun F q \in Q$ de tal manera que $p\leq_P q$ si y s'olo si $\fun F p \leq_Q \fun F q$. Observemos que estos funtores son los morfismos en la categor'ia $\poset$. 
        \item El funtor $\pi_1 \colon \topo_*\to \grp$ que manda todo espacio topol'ogico $X$ con punto base $x$ a su grupo funamental $\pi_1(X,x)$ y a toda funci'on continua $f\colon (X,x)\to (Y,y)$ a un homomorfismo de grupos $\pi_1(f)\colon \pi_1(X,x)\to \pi_1(Y,y)$. 
        \item Sea $\cat C$ una categor'ia, el funtor identidad $\id {\cat C}\colon \cat C\to \cat C$ manda a objetos $C$ y morfismos $f\colon C\to D$ en s'i mismos.  
    \end{enumerate}
\end{ejems}

También denominamos a este tipo de funtor como \textbf{funtor covariante}, para distinguirlo de otro tipo de funtor. 

\begin{defi}[Funtor contravariante] \label{contravariante}
    Sean $\cat C$, $\cat D$ categor'ias. Un \textbf{funtor contravariante} de $\cat C$ a $\cat D$ es un funtor $\fun F\colon \opp C \to \cat D$. Siendo m'as expl'icitos, la diferencia principal de 'este y el funtor covariante de $\cat C$ en $\cat D$ es que morfismos $f\colon A\to B$ en $\cat C$ son mandados a morfismos $\fun F\colon \fun F B\to \fun F A$ en $\cat D$. Adem'as, si el codomonio del funtor es $\set$, decimos que el funtor es una \textbf{pregavilla}.
\end{defi}
\begin{ejem}\label{homfunctor} Sea $\cat C$ una categor'ia localmente pequeña y $C$ en $\ob C$. El funtor contravariante $\homc{C}{-}{C}\colon \opp C\to \set$ asigna objetos $A$ a conjuntos $\homc{C}{A}{C}$ y dado un morfismo $f\colon A\to B$ en $\homc{C}{A}{B}$ le asigna a 'este un morfismo $\homc{C}{-}{C} f\colon \homc{C}{B}{C}\to \homc{C}{A}{C}$ dado por $g\mapsto gf$.

Podemos, en una manera similar, definir un funtor covariante de $\cat C$ en $\set$, denotado $\homc{C}{C}{-}$, que asigna objetos $A$ en $\ob C$ a conjuntos $\homc{C}{C}{A}$ y morfismos $f\colon A\to B$ a morfismos $\homc{C}{C}{-}\colon \homc{C}{C}{A}\to \homc{C}{C}{B}$ dado por $g\mapsto fg$.
\end{ejem}

Como mencion'abamos al inicio de la secci'on, podemos entender a los funtores como morfismos entre categor'ias, por lo que naturalmente surge la pregunta de si existen nociones similares a epimorfismo, monomorfismo o isomorfismo para funtores. Tenemos nociones similares en categor'ias localmente pequeñas, que definimos a continuación.

\begin{defi}
    Sean $\cat C,\cat D$ categor'ias localmente pequeñas y $\func{F}{C}{D}$ un funtor entre ellas. Decimos que el funtor es:

    \begin{enumerate}
        \item \textbf{fiel} si para cada $A,B$ en $\ob C$, la funci'on $\homc{C}{A}{B}\to \homc{D}{\F A}{\F B}$ es inyectiva;
        \item \textbf{pleno} si para cada $A,B$ en $\ob C$, la funci'on $\homc{C}{A}{B}\to \homc{D}{\F A}{\F B}$ es suprayectiva;
        \item \textbf{esencialmente suprayectivo} si para cada $A$ en $\ob C$ existe una $D$ en $\ob D$ tal que $D$ es isom'orfica a $\F A$.
    \end{enumerate}
    \end{defi}
    Notemos aqu'i que estamos hablando de funciones, inyectividad y suprayectividad, por ello es que requerimos que las categor'ias en cuesti'on sean localmente pequeñas. Por otro lado, la noción de isomorfismo es idéntica a como vimos con morfismos. 
    \begin{defi} Sean $\cat C,\cat D$ categor'ias. Un funtor $\func{F}{C}{D}$ es un \textbf{isomorfismo} si existe un funtor $\func{G}{D}{C}$ tal que $\F \fun G = \id{\cat D}$ y $\fun G \F = \id{\cat C}$. En este caso decimos que $\F F$ y $\fun G$ son \textbf{isomorfos} y lo denotamos por $\F\cong \fun G$.
        
    \end{defi}

    Si los funtores los podemos entender como morfismos entre categor'ias, es igual de v'alido preguntarse sobre morfismos entre funtores. Hist'oricamente, este tipo de morfismos fueron los primeros en ser estudiados y que derivaron en la creaci'on de la teor'ia de categor'ias. Antes de definirlos propiamente, veamos un ejemplo en donde su presencia es 'util. 

    \begin{ejem}
        Sea $n \in \N$ y consideremos cualquier anillo conmutativo unitario $R$. Consideremos ahora las matrices $n\times n$ con entradas en $R$. Junto con la multiplicaci'on de matrices, este conjunto forma un monoide, que denotamos por $M_n(R)$. Por otro lado, si conservamos s'olo la multiplicaci'on del anillo $R$ y olvidamos la suma, obtenemos tambi'en un monoide que podemos denotar por $U(R)$. A partir de un anillo $R$ entonces tenemos dos maneras de llegar a un monoide, una definida por $M_n(R)$ y otra por $U(R)$.

        Recordemos ahora que cualquier matriz cuadrada $A$ en cualquier anillo $R$ posee un determinante $\det(A)$ con valores en $R$ y que, en particular, cumple las siguientes dos propiedades: 
        \[\det(A)\det(B) = \det(AB)\qquad \qquad \det(I_m) = 1_R\]
        donde $I_m$ es la matriz identidad de las matrices de $m\times m$ con $m\in \N$ y $1_R$ es la identidad bajo multiplicaci'on de $Q$. Observemos que estas mismas dos propiedades son las que determinan un homomorfismos de monoides y que $\det$ toma matrices cuadradas con entradas en $R$ y devuelve valores en $R$. Por lo tanto podemos considerar el homomorfismo $\det\colon M_n(R) \to U(R)$, para cada anillo $R$. 

       Consideremos un homomorfismo de anillos conmutativos $f\colon R\to Q$. Si preservamos 'unicamente la estructura de multiplicaci'on, la funci'on en cuesti'on es ahora un homomorfismo de monoides. Podemos denotar esta nueva funci'on como $U(f)\colon U(R)\to U(Q)$. Es decir, para cada anillo conmutativo $R$ tenemos un monoide $U(R)$ y para cada homomorfismo de anillos conmutativos $f\colon R\to Q$ tenemos un homomorfismo de monoides $U(f)\colon U(R)\to U(Q)$. Del mismo modo, todo homomorfismo de anillos $f\colon R\to Q$ induce un homomorfismos de monoides $M_n(f)\colon M_n(R)\to M_n(Q)$. Del estudio que llevamos al momento salta a la luz la noci'on de funtor. En efecto, podemos considerar la categor'ia $\cring$ cuyos objetos son anillos conmutativos unitarios y sus morfismos son homomorfismos de anillos conmutativos y la categor'ia $\mon$ cuyos objetos son monoides y cuyos morfismos son homomorfismos de monoides. Tenemos entonces dos funtores $M_n\colon \cring \to \mon$ y $U\colon \cring \to \mon$. 

       Con estas nociones en mente, consideremos el siguiente diagrama 
       % https\colon//q.uiver.app/#q=WzAsNCxbMCwwLCJNX24oUikiXSxbMywwLCJNX24oUSkiXSxbMCwzLCJNKFIpIl0sWzMsMywiTShRKSJdLFswLDEsIk1fbihmKSJdLFswLDIsIlxcZGV0X1IoTV9uKFIpKSIsMl0sWzIsMywiTShmKSIsMl0sWzEsMywiXFxkZXQoTV9uKFEpKSJdXQ==
\[\begin{tikzcd}
	{M_n(R)} &&& {M_n(Q)} \\
	\\
	\\
	{U(R)} &&& {U(Q)}
	\arrow["{M_n(f)}", from=1-1, to=1-4]
	\arrow["{\det(M_n(R))}"', from=1-1, to=4-1]
	\arrow["{\det(M_n(Q))}", from=1-4, to=4-4]
	\arrow["{U(f)}"', from=4-1, to=4-4]
\end{tikzcd}\]
Podemos observar que $\det$ act'ua como un morfismo entre estos dos funtores. La primera pregunta que podr'ia salirnos en mente es si el diagrama es conmutativo, i.e., tenemos que $U(f)\det(M_n(R)) = \det(M_n(Q))M_n(f)$. Podemos verificar r'apidamente que el diagrama conmuta considerando los elementros neutros de los monoides y recordando que todas estas funciones son homomorfismos de monoides. 

La elecci'on de los valores aqu'i es arbitraria, por lo que vemos que para la familia de morfismos $\pare{\det\colon M_n(R)\to U(R)}_{R\in \cring}$ cualquier diagrama como el mostrado arriba ser'a conmutativo. Si consideramos entonces a $\det\colon M_n\to M$, con $\det$ la famlilia mencionada $\pare{\det\colon M_n(R)\to U(R)}_{R\in \cring}$ vemos que para cada homomorfismos de anillos $f\colon R\to Q$, el diagrama mostrado arriba conmutar'a. Esto refleja que la noci'on de determinante es uniforme para todos los anillos, i.e., no es dependiente de c'omo est'a determinado un anillo en particular. Esta es la idea fundamental detr'as de una transformaci'on natural, la cual procedemos a definir formalmente. 
\end{ejem}

\begin{defi}[Transformaci'on Natural]
Sean $\C$ y $\cat D$ categor'ias, y \df{\C}{\cat D}{\F}{\fun G} funtores. Una \textbf{transformaci'on natural} $\alpha\colon \F \Rightarrow \fun G$ consiste en una colecci'on de morfismos $\alpha_c\colon \F c\to \fun G c$ para cada objeto $c$ en $\ob C$ tales que para cada morfismo $f\colon c\to d$ en $\C$, el siguiente diagrama de morfismos en $\cat D$ es conmutativo:

% https\colon//q.uiver.app/#q=WzAsNCxbMCwwLCJGYyJdLFsyLDAsIkZkIl0sWzAsMiwiR19jIl0sWzIsMiwiR2QiXSxbMCwxLCJGZiJdLFswLDIsIlxcYWxwaGFfYyIsMl0sWzIsMywiR2YiLDJdLFsxLDMsIlxcYWxwaGFfZCJdXQ==
\[\begin{tikzcd}
	Fc && Fd \\
	\\
	{G_c} && Gd
	\arrow["Ff", from=1-1, to=1-3]
	\arrow["{\alpha_c}"', from=1-1, to=3-1]
	\arrow["{\alpha_d}", from=1-3, to=3-3]
	\arrow["Gf"', from=3-1, to=3-3]
\end{tikzcd}\]
Si adem'as cada uno de los morfismos $\alpha_c$ es un isomorfismos, entonces decimos que es un \textbf{isomorfismo natural} y lo denotamos por $\alpha\colon \F \cong \fun G$. A los morfismos $\alpha_c$ tambi'en los denotamos como los \textbf{componentes} de la transformaci'on natural. 
\end{defi}
\begin{ejems} \leavevmode
\begin{enumerate}[label=\alph*)]
    \item Los funtores contravariantes $\fun O, \fun C\colon \begin{tikzcd}
	{\topo^{op}} & \set
	\arrow[shift left, from=1-1, to=1-2]
	\arrow[shift right, from=1-1, to=1-2]
\end{tikzcd}$, donde $\fun O$ es el funtor que lleva espacios topol'ogicos $X$ a la familia de conjuntos abiertos de $X$ y que a funciones continuas $f\colon X\to Y$ las lleva a $\fun O f\colon \fun O Y\to \fun O X$, con $\fun O f$ la funci'on que lleva a un subconjunto $U\subset Y$ a su preimagen bajo $f$, $\inv f(U)$ y $\fun C$ es el funtor an'alogo pero llevando los espacios a la familia de conjuntos cerrados de la topolog'ia son naturalmene isom'orficos, con la transformaci'on natural $\alpha\colon \fun O \to \fun C$, donde los morfismos $\alpha_X$ para cada $X\in \topo$ llevan a los conjuntos abiertos a su complemento, que es un conjunto cerrado. Para cada uno de estos, su inverso $\inv\alpha_X$ est'a dado como el morfismo que lleva los conjuntos cerrados de $X$ a su complemento, que es un conjunto abierto. 

\item Recordemos que dados 'ordenes parciales $(P,\leq_P)$, $(Q,\leq_Q)$ vistos como categor'ias, un funtor entre ellos es una funci'on que preserva 'ordenes. Consideremos funtores $\fun F, \fun G$ con dominio $P$ y codominio $Q$. Si para todo $p\in P$ tenemos que $\F p \leq_Q \fun G p$  podemos definir morfismos $\alpha_p\colon \F p \to \fun G p$ tales que el diagrama
% https\colon//q.uiver.app/#q=WzAsNCxbMCwwLCJcXGZ1bmYgcCJdLFsyLDAsIlxcZnVuZiBxIl0sWzAsMiwiXFxmdW4gRyBwIl0sWzIsMiwiXFxmdW4gRyBxIl0sWzAsMV0sWzAsMiwiXFxhbHBoYV9wIiwyXSxbMiwzXSxbMSwzLCJcXGFscGhhX3EiXV0=
\[\begin{tikzcd}
	{\F p} && {\F q} \\
	\\
	{\fun G p} && {\fun G q}
	\arrow[from=1-1, to=1-3]
	\arrow["{\alpha_p}"', from=1-1, to=3-1]
	\arrow["{\alpha_q}", from=1-3, to=3-3]
	\arrow[from=3-1, to=3-3]
\end{tikzcd}\]
conmuta, gracias a la transitividad del orden parcial $(Q,\leq_q)$. Por otro lado, la existencia de morfismos $\alpha_p\colon \F p \to \fun G p$ implica que $\F p\leq_Q \fun G p$, por lo que vemos que una transformaci'on natural entre los funtores $\F, \fun G$ existe si y solo si $\F p\leq_Q \fun G p$ para todo $p\in P$. 
\end{enumerate}
    
\end{ejems}

Las transformaciones naturales empiezan a resaltar la importancia de los diagramas para representar la informaci'on dada. Nos permiten entender de una manera m'as directa lo que est'a ocurriendo a trav'es de los objetos y los morfismos. Por ejemplo, dado el diagrama 
% https\colon//q.uiver.app/#q=WzAsMyxbMCwwLCJBIl0sWzAsMiwiQyJdLFsxLDEsIkIiXSxbMCwxLCJnZiIsMl0sWzAsMiwiZiJdLFsyLDEsImciXV0=
\[\begin{tikzcd}
	A \\
	& B \\
	C
	\arrow["f", from=1-1, to=2-2]
	\arrow["gf"', from=1-1, to=3-1]
	\arrow["g", from=2-2, to=3-1]
\end{tikzcd}\]
decir que es conmutativo representa la idea de la composici'on en una categor'ia. Ahora, un funtor lo podemos entender como el traslado o mapeo de la informaci'on de la categor'ia a otra dada y la propiedad $\F (gf) = \F g \F f$ implica que el mapeo del diagrama conserva la conmutatividad, es decir, el tri'angulo derecho del siguiente diagrama tambi'en es conmutativo.

% https\colon//q.uiver.app/#q=WzAsOCxbMCwwLCJBIl0sWzAsMiwiQyJdLFsxLDEsIkIiXSxbNCwwLCJGQSJdLFs0LDIsIkZDIl0sWzUsMSwiRkIiXSxbMiwxXSxbMywxXSxbMCwxLCJnZiIsMl0sWzAsMiwiZiJdLFsyLDEsImciXSxbMyw0LCJGZ2YiLDJdLFszLDUsIkZmIl0sWzUsNCwiRmciXSxbNiw3LCJGIl1d
\[\begin{tikzcd}
	A &&&& FA \\
	& B & {} & {} && FB \\
	C &&&& FC
	\arrow["f", from=1-1, to=2-2]
	\arrow["gf"', from=1-1, to=3-1]
	\arrow["Ff", from=1-5, to=2-6]
	\arrow["Fgf"', from=1-5, to=3-5]
	\arrow["g", from=2-2, to=3-1]
	\arrow["F", from=2-3, to=2-4]
	\arrow["Fg", from=2-6, to=3-5]
\end{tikzcd}\]

Finalmente podemos entender a una transformaci'on natural como un mapeo o traslado de la informaci'on de un funtor $\F$ a otro dado $\fun G$, de tal modo que tambi'en el diagrama 

% https\colon//q.uiver.app/#q=WzAsMTEsWzAsMCwiQSJdLFswLDIsIkMiXSxbMSwxLCJCIl0sWzQsMCwiRkEiXSxbNCwyLCJGQyJdLFs1LDEsIkZCIl0sWzIsMV0sWzMsMV0sWzcsMCwiR0EiXSxbNywyLCJHQyJdLFs5LDEsIkdCIl0sWzAsMSwiZ2YiLDJdLFswLDIsImYiXSxbMiwxLCJnIl0sWzMsNCwiRmdmIiwyXSxbMyw1LCJGZiJdLFs1LDQsIkZnIl0sWzYsNywiRiJdLFszLDgsIlxcYWxwaGFfQSJdLFs1LDEwLCJcXGFscGhhX0IiLDEseyJsYWJlbF9wb3NpdGlvbiI6MzB9XSxbOCw5LCJHZ2YiLDAseyJsYWJlbF9wb3NpdGlvbiI6MzB9XSxbOCwxMCwiR2YiXSxbMTAsOSwiR2ciXSxbNCw5LCJcXGFscGhhX0MiLDJdXQ==
\[\begin{tikzcd}
	A &&&& FA &&& GA \\
	& B & {} & {} && FB &&&& GB \\
	C &&&& FC &&& GC
	\arrow["f", from=1-1, to=2-2]
	\arrow["gf"', from=1-1, to=3-1]
	\arrow["{\alpha_A}", from=1-5, to=1-8]
	\arrow["Ff", from=1-5, to=2-6]
	\arrow["Fgf"', from=1-5, to=3-5]
	\arrow["Gf", from=1-8, to=2-10]
	\arrow["g", from=2-2, to=3-1]
	\arrow["F", from=2-3, to=2-4]
	\arrow["{\alpha_B}"{pos=0.3}, from=2-6, to=2-10]
	\arrow["Fg", from=2-6, to=3-5]
	\arrow["Gg", from=2-10, to=3-8]
	\arrow["{\alpha_C}"', from=3-5, to=3-8]
    \arrow["Ggf"{pos=0.3}, from=1-8, to=3-8, crossing over]
\end{tikzcd}\]

es conmutativo. Expresar toda esta información a través de igualdad de composiciones no sólo sería tedioso y tardado, sino también confuso en la entendimiento de las ideas expuestas. Hay una clara ventaja pedagógica en el apoyo de los diagramas para expresar nuestras ideas y, como veremos, nuestras demostraciones. La sección que sigue es un ejemplo perfecto de esto, y es también uno de los resultados centrales en la teoría de categorías. 

%Aquí finaliza la segunda sección
%---------------------------------------------------------------------------------------------------

\section{Lema de Yoneda}

El Lema de Yoneda es uno de los resultados m'as importantes de la teor'ia de categor'ias. La primera aparici'on del lema en la literatura es con Grothendieck en \cite{lemayoneda}, aunque el resultado se la atribuye a Nobuo Yoneda. Para entender adecuadamente el resultado, necesitamos introducir algunos resultados. Primeramente, recordemos la definici'on ~\ref{elemento global} de un elemento global. Esta noci'on es m'as tradicional a como entendemos la pertenencia y lo denotamos global porque no depende de ning'un punto de vista particular. Pero podemos extender la noci'on conjuntista de pertenencia y pensar en puntos de vista o de \enquote{referencia} desde los cuales podemos observar a un objeto. En la definici'on de elemento global usamos de punto de partida el objeto $\uno$, pero podemos observar al elemento desde otro punto de partida. Esto nos lleva a la siguiente definici'on 
\begin{defi}[Elemento local] \label{elemento local}
    Sea $A$ un objeto no isom'orfico a $\uno$. Decimos que un morfismo $A\to B$ es un \textbf{elemento local} de $B$ en la \textbf{etapa} $A$. 
\end{defi}

Esta noci'on de obsevar un elemento en cierta \enquote{etapa} ser'a 'util para entender lo que el lema de Yoneda dice. 

Recordemos ahora los funtores vistos en el ejemplo \ref{homfunctor}. Nos referimos a estos como \textbf{Hom-funtores}, en particular $\ff{\homo{A}{-}}{\C}{\set}$ es el Hom-funtor covariante y $\ff{\homo{-}{A}}{\opp C}{\set}$ es el Hom-funtor contravariante. Estos dos funtores son particularmente importantes porque nos permiten hablar de la noci'on de \enquote{representabilidad} que definimos a continuaci'on. 

\begin{defi}[Representable]
    Sea $\C$ una categor'ia. Un funtor $\func{F}{C}{\set}$ es (covariablemente) \textbf{representable} si es naturalmente isom'orfico a un Hom-funtor $\ff{\homo{A}{-}}{\C}{\set}$ para alg'un objeto $A$ en $\ob C$. Del mismo modo, un funtor $\ff{\fun G}{\opp C}{\set}$ es (contravariablemente) \textbf{representable} si es naturalmente isom'orfico a un Hom-funtor contravariante $\ff{\homo{-}{A}}{\opp C}{\set}$ para un objeto $A$ en $\ob C$. En ambos casos, tal objeto $A$ es llamado un \textbf{objeto representante} del funtor $\F$ o $\fun G$, seg'un corresponda. El objeto junto con la transformaci'on natural que induce el isomorfismo son llamados una \textbf{representaci'on del funtor} $\F$ o $\fun G$, seg'un corresponda.
\end{defi}

Ponemos entre par'entesis si es covariante o contravariante porque el dominio del Hom-funtor aclara inmediatamente eso. As'i, si no existe confusi'on decimos simplemnte que un funtor es representable. Antes de discutir su relaci'on con la definici'on \ref{elemento local}, veamos  algunos ejemplos. 

\begin{ejems} \leavevmode\begin{enumerate}[label=\alph*)]
    \item El funtor potencia $\ff{\fun P}{\set}{\set}$ visto en \ref{ejemplos funtores} no es representable, sin embargo, el funtor potencia contravariante $\ff{\invopp {\fun P}}{\opp \set}{\set}$ s'i es representable. Consideremos el objeto $\mathbf{2}= \{\varnothing, \{\varnothing\}\}=\{\mathbf{0},\uno\}$, y a $\ff{\homo{-}{\dos}}{\opp \set}{\set}$. La transformaci'on natural $\ff{\alpha}{\invopp {\fun P}}{\homo{-}{\dos}}$ est'a dada por la familia de morfismos $\ff{\alpha_A}{P(A)}{\homo{A}{\dos}}$ para cada conjunto $A$, que asigna $B\mapsto \chi_B$, donde $B\subseteq A$ y $\chi_B$ es la funci'on caracter'istica de $B$, 
    \[  \chi_B(x)=
    \begin{cases}
        \cero \quad \text{si}\, x\notin B\\
        \uno \quad \text{si}\, x\in B.
    \end{cases}\]
    Es claro ver por qu'e estos morfismos inducen una biyecci'on. Dado un subconjunto $B\subseteq A$, los elementos que pertenecen a 'este son mandados a $\uno$, es decir, $B$ no es otra cosa m'as que la preimagen bajo $\chi_B$ de $\uno$. Por otro lado, para ver la naturalidad necesitamos que dada una funci'on $f: A\to B$, el siguiente diagrama conmute

    % https://q.uiver.app/#q=WzAsNCxbMCwwLCJQKEIpIl0sWzIsMCwiUChBKSJdLFswLDIsIlxcaG9tb3tCfXtcXGRvc30iXSxbMiwyLCJcXGhvbW97QX17XFxkb3N9Il0sWzAsMSwiXFxpbnYgZiJdLFswLDIsIlxcYWxwaGFfQiIsMl0sWzEsMywiXFxhbHBoYV9BIl0sWzIsMywiZl4qIiwyXV0=
\[\begin{tikzcd}
	{P(B)} && {P(A)} \\
	\\
	{\homo{B}{\dos}} && {\homo{A}{\dos},}
	\arrow["{\inv f}", from=1-1, to=1-3]
	\arrow["{\alpha_B}"', from=1-1, to=3-1]
	\arrow["{\alpha_A}", from=1-3, to=3-3]
	\arrow["{f^*}"', from=3-1, to=3-3]
\end{tikzcd}\]

donde $f^*$ denota la precomposici'on. Notemos que el camino derecho nos dice que dado un conjunto $C\subseteq B$, nos da una funci'on clasificadora de la preimagen de $C$, es decir, nos da $\chi_{\inv f(C)}$. Por otro lado, el camino izquierdo nos dice que dado ese conjunto $C \subseteq B$, nos da a trav'es de la precomposici'on con $f^*$ una funci'on clasificadora dada por $\chi_C f$. Ver la igualdad entonces equivale a que $\chi_C f$ sea una funci'on clasificadora de $\inv f(C)\subseteq A$, lo cual es claramente el caso. Por lo tanto, el diagrama conmuta. 

De los argumentos anteriores vemos que $\alpha$ es un isomorfismo natural y por lo tanto $\invopp {\fun{P}}$ es representado por $\dos$. En realidad, es claro ver que cualquier conjunto con dos elementos representa a este funtor. 

Pensemos ahora 

\item Consideremos el funtor identidad de $\set$, $\ff{\id{\set}}{\set}{\set}$. Este funtor est'a representado por el elemento $\uno$. Es decir, $\homo{\uno}{-}\cong \id{\set}$. La transformaci'on natural es bastante directa, para cada $X$ en $\ob \set$, $\homo{\uno}{X}$ es el conjunto de todos los elementos de $X$, es decir, de todos los $\uno \xrightarrow{x} X$ que recordando la discusi'on en \ref{elemento global} no es m'as que los $x\in X$, por lo que $\homo{\uno}{X}$ es isom'orfico a $X$. Es inmediato ver por qu'e esta biyecci'on es natural en $\set$ tambi'en. Este funtor nos muestra de una manera clara que en $\set$, dado un conjunto $X$, $x\in X \iff \uno\xrightarrow{x} X$.  
    \end{enumerate}
 \end{ejems}

Antes de enunciar y demostrar el lema de Yoneda, veamos el siguiente ejemplo. 

\begin{ejem} \label{yonedaejem} Pensemos en un orden parcial $(P,\leq)$ como categor'ia y en un funtor $\F\to \set$. Para facilidad de consideraci'on, supongamos que la cardinalidad del conjunto $P$ es igual o menor a la de los naturales. Consideremos un objeto $p$. Tenemos que 
 \[\homo{p}{-}= \{q\in P\,\vert\, p\leq q\}\]
De existir una transformaci'on natural $\nat{\alpha}{\homo{p}{-}}{\F}$, debe ser tal que para cualquier $q\to r$, el diagrama
 % https://q.uiver.app/#q=WzAsNCxbMCwwLCJcXGhvbW97cH17cX0iXSxbMiwwLCJcXGhvbW97cH17cn0iXSxbMCwyLCJcXEYgcSJdLFsyLDIsIlxcRiByIl0sWzAsMV0sWzAsMiwiXFxhbHBoYSIsMl0sWzIsM10sWzEsMywiXFxhbHBoYSJdXQ==
\[\begin{tikzcd}
	{\homo{p}{q}} && {\homo{p}{r}} \\
	\\
	{\F q} && {\F r}
	\arrow[from=1-1, to=1-3]
	\arrow["\alpha_q"', from=1-1, to=3-1]
	\arrow["\alpha_r", from=1-3, to=3-3]
	\arrow[from=3-1, to=3-3]
\end{tikzcd}\]
conmuta. Si $p$  y $q$ no son comparables bajo $\leq$ el diagrama no aporta informaci'on alguna, del mismo modo que si $q\leq p$, pues ser'ia vac'io en su contenido. Si $p\leq q$, $\homo{p}{q}$ es simplemente la flecha $p\to q$ y del mismo modo $\homo{p}{r}$ es la flecha $p\to r$, es decir, ambos son unitarios. $\homo{p}{q}\to \homo{p}{r}$ es simplemente la transitividad del orden parcial en acci'on: $p\leq q \leq r$ implica $p\leq r$. De esto observamos que el diagrama determina elementos $\alpha_q\in \F q$ y $\alpha_r\in \F r$. Si denotamos por $f_{q,r}$ a la $\F q\to \F r$ del diagrama, vemos que la naturalidad implica $\alpha_r = f_{q,r}\alpha_q$. Si seguimos este proceso con una cadena que inicie en $p$, i.e., 
% https://q.uiver.app/#q=WzAsMTAsWzEsMCwiXFxob21ve3B9e3F9Il0sWzIsMCwiXFxob21ve3B9e3J9Il0sWzMsMCwiXFxob21ve3B9e3N9Il0sWzQsMCwiXFxkb3RzIl0sWzEsMSwiXFxGIHEiXSxbMiwxLCJcXEYgciJdLFszLDEsIlxcRiBzIl0sWzQsMSwiXFxkb3RzIl0sWzAsMCwiXFxob21ve3B9e3B9Il0sWzAsMSwiXFxGIHAiXSxbMCwxXSxbMSwyXSxbMiwzXSxbMCw0LCJcXGFscGhhIiwyXSxbMSw1LCJcXGFscGhhIiwyXSxbMiw2LCJcXGFscGhhIiwyXSxbNCw1LCJmX3txLHJ9IiwyXSxbNSw2LCJmX3tyLHN9IiwyXSxbNiw3LCJmX3tzLHR9IiwyXSxbOCwwXSxbOCw5LCJcXGFscGhhX3AiLDJdLFs5LDQsImZfe3AscX0iLDJdXQ==
\[\begin{tikzcd}
	{\homo{p}{p}} & {\homo{p}{q}} & {\homo{p}{r}} & {\homo{p}{s}} & \dots \\
	{\F p} & {\F q} & {\F r} & {\F s} & \dots
	\arrow[from=1-1, to=1-2]
	\arrow["{\alpha_p}"', from=1-1, to=2-1]
	\arrow[from=1-2, to=1-3]
	\arrow["\alpha_q"', from=1-2, to=2-2]
	\arrow[from=1-3, to=1-4]
	\arrow["\alpha_r"', from=1-3, to=2-3]
	\arrow[from=1-4, to=1-5]
	\arrow["\alpha_s"', from=1-4, to=2-4]
	\arrow["{f_{p,q}}"', from=2-1, to=2-2]
	\arrow["{f_{q,r}}"', from=2-2, to=2-3]
	\arrow["{f_{r,s}}"', from=2-3, to=2-4]
	\arrow[ from=2-4, to=2-5]
\end{tikzcd}\]
 se vuelve un ejercicio inductivo ver que para todos estos elementos son iguales a $\alpha_p$. Por supuesto, el orden parcial puede tener muchas ramificaciones en el proceso. Sin embargo, hay un componente $\alpha$ que \enquote{absorbe} a todos los dem'as valores, a saber, $\alpha_p: \homo{p}{p}\to \F p$. En principio, nuestro orden parcial visualizando a todos los elementos comparables con $p$ podr'ia verse de este estilo 
 % https://q.uiver.app/#q=WzAsMTksWzQsNSwicCJdLFszLDQsInEiXSxbMiwzLCJyIl0sWzEsMiwicyJdLFswLDEsInQiXSxbMiwxLCJcXGJ1bGxldCJdLFszLDIsIlxcYnVsbGV0Il0sWzQsMSwiXFxidWxsZXQiXSxbNCwzLCJcXGJ1bGxldCJdLFs1LDIsIlxcYnVsbGV0Il0sWzYsMSwiXFxidWxsZXQiXSxbNywwLCJcXGRvdHMiXSxbNSwwLCJcXGRvdHMiXSxbNSw0LCJcXGJ1bGxldCJdLFs2LDMsIlxcYnVsbGV0Il0sWzcsMSwiXFxidWxsZXQiXSxbNywyLCJcXGJ1bGxldCJdLFsyLDAsIlxcZG90cyJdLFsxLDAsIlxcZG90cyJdLFswLDFdLFsxLDJdLFsyLDNdLFszLDRdLFszLDVdLFsyLDZdLFs2LDVdLFs2LDddLFsxLDhdLFs4LDldLFs5LDEwXSxbMTAsMTFdLFsxMCwxMl0sWzAsMTNdLFsxMyw4XSxbMTMsMTRdLFsxNCw5XSxbOSwxNV0sWzE0LDE2XSxbMTYsMTVdLFs1LDE3XSxbNCwxOF1d
\[\begin{tikzcd}
	& \dots & \dots &&& \dots && \dots \\
	t && \bullet && \bullet && \bullet & \bullet \\
	& s && \bullet && \bullet && \bullet \\
	&& r && \bullet && \bullet \\
	&&& q && \bullet \\
	&&&& p
	\arrow[from=2-1, to=1-2]
	\arrow[from=2-3, to=1-3]
	\arrow[from=2-7, to=1-6]
	\arrow[from=2-7, to=1-8]
	\arrow[from=3-2, to=2-1]
	\arrow[from=3-2, to=2-3]
	\arrow[from=3-4, to=2-3]
	\arrow[from=3-4, to=2-5]
	\arrow[from=3-6, to=2-7]
	\arrow[from=3-6, to=2-8]
	\arrow[from=3-8, to=2-8]
	\arrow[from=4-3, to=3-2]
	\arrow[from=4-3, to=3-4]
	\arrow[from=4-5, to=3-6]
	\arrow[from=4-7, to=3-6]
	\arrow[from=4-7, to=3-8]
	\arrow[from=5-4, to=4-3]
	\arrow[from=5-4, to=4-5]
	\arrow[from=5-6, to=4-5]
	\arrow[from=5-6, to=4-7]
	\arrow[from=6-5, to=5-4]
	\arrow[from=6-5, to=5-6]
\end{tikzcd}\]
donde los $\bullet$ son otros elementos del orden parcial. Las flechas nos indican que los elementos estan bajo la relaci'on $\leq$. El conjunto pudiera ser denso tambi'en, pero como todos estos elementos son comparables con $p$, la naturalidad nos dar'a siempre que estos elementos son iguales a $\alpha_p\in \F p$. Por lo tanto, la transformaci'on natural $\nat{\alpha}{\homo{p}{-}}{\F}$ est'a completamente determinada por la elecci'on $\alpha_p\in \F p$. Podemos pensar en este elemento como la imagen del morfismo $\id p$ bajo la transformaci'on natural $\ff{\alpha_p}{\homo{p}{p}}{\F p}$. ¿Cu'ales son las posibles elecciones de elementos? Vemos que es precisamente los $x\in \F p$. 

De este an'alisis vemos que las transformaciones naturales de $\homo{p}{-}$ al funtor $\F$ est'an determinadas por los elementos de $\F p$. Esta idea, como veremos, no depende de nuestra elecci'on $p$ o del funtor en particular, ni de haber trabajado en un conjunto parcialmente ordenado.

Volveremos a este ejemplo en un momento, pero ahora enunciamos el teorema m'as importante en teor'ia de categor'ias.
\end{ejem}
\newpage

\begin{teo}[Lema de Yoneda] \label{lema yoneda}
    Para cualquier funtor $\ff{\F}{\C}{\set}$, con $\C$ una categor'ia localmente pequeña, y cualquier objeto $A$ en $\ob C$, la siguiente funci'on es una biyecci'on:
    \[ \varphi \colon \Nat(\homo{A}{-}),\F)\cong \F A,    \] 
    \[ \varphi(\alpha) = \alpha_A(\id A)\]

    donde $\Nat(\homo{A}{-}),\F)$ denota la colecci'on de transformaciones naturales entre $\homo{A}{-}$ y $\F$.
\end{teo}
\begin{proof}
    Queremos construir una $\ff{\phi}{\F A}{\Nat(\homo{A}{-}),\F)}$ que sea la inversa de $\varphi$. Para este fin, para cada elemento $x\in \F A$, $\phi(x)$ debe ser una transformaci'on natural, por lo que tenemos que definir una colecci'on de morfismos $\ff{\phi(x)_B}{\homo{A}{B}}{\F B}$ para cada objeto $B$ en $\ob C$ que definan a $\nat{\phi(x)}{\homo{A}{-}}{\F}$. Para este fin, definimos para cada $x\in \F A$ y cada objeto $B$ en $\ob C$,
    \[\phi(x)_B(f) = \F f(x)\]
    recordando que $f: A\to B$. Para corroborar que $\phi(x)$ es una transformaci'on natural, debemos verificar que para cualquier $\ff{g}{B}{C}$ con $C$ en $\ob C$, el diagrama 
    % https://q.uiver.app/#q=WzAsNCxbMCwwLCJcXGhvbW97QX17Qn0iXSxbMiwwLCJcXGhvbW97QX17Q30iXSxbMCwyLCJcXEYgQiJdLFsyLDIsIlxcRiBDIl0sWzAsMSwiZ18qIl0sWzAsMiwiXFxwaGkoeClfQiIsMl0sWzIsMywiRmciLDJdLFsxLDMsIlxccGhpKHgpX0MiXV0=
\[\begin{tikzcd}
	{\homo{A}{B}} && {\homo{A}{C}} \\
	\\
	{\F B} && {\F C}
	\arrow["{g_*}", from=1-1, to=1-3]
	\arrow["{\phi(x)_B}"', from=1-1, to=3-1]
	\arrow["{\phi(x)_C}", from=1-3, to=3-3]
	\arrow["Fg"', from=3-1, to=3-3]
\end{tikzcd}\]
conmuta, donde $g_*$ denota la post-composici'on. Evaluemos el camino izquierdo del diagrama.  

\begin{align*}
    \F g(\phi(x)_B(f)) &= \F g \F f(x) \\
                     &= \F gf(x) \\
                     &= \phi(x)_C (gf)\\
                     &= \phi(x)_C(g_*(f)).
\end{align*}
Por lo tanto el diagrama conmuta. Notemos que el paso de la primera igualdad a la segunda se cumple por los axiomas de funtorialidad vistos en \ref{funtor}.

Recordando que $\varphi(x)$ es una transformaci'on natural y por definici'on de $\phi$ tenemos que $\phi\varphi(x) = \phi(x)_A(\id A)$, por definici'on de $\phi(x)_A$ esto es igual a $\F \id A(x)$ y por axiomas de funtorialidad esto es $\id{\F A}(x)= x$. Por lo tanto, $\phi\varphi(x) = x$. 

Para probar la biyecci'on, nos resta probar que $\varphi\phi(\alpha) = \alpha$. Consideremos un morfismo $\ff{f}{A}{B}$. Evaluando, 
\begin{align*}
    \varphi(\phi(\alpha))_B (f) &= \varphi(\alpha_A(\id A))_B (f) \\
                                &= F(f)(\alpha_A(\id A))
\end{align*}
Ahora, $\alpha$ es una transformaci'on natural, por lo que el diagrama 
% https://q.uiver.app/#q=WzAsNCxbMCwwLCJcXGhvbW97QX17QX0iXSxbMiwwLCJcXGhvbW97QX17Qn0iXSxbMCwyLCJcXEYgQSJdLFsyLDIsIlxcRiBCIl0sWzAsMSwiZl8qIl0sWzAsMiwiXFxwaGkoeClfQSIsMl0sWzIsMywiRmYiLDJdLFsxLDMsIlxccGhpKHgpX0IiXV0=
\[\begin{tikzcd}
	{\homo{A}{A}} && {\homo{A}{B}} \\
	\\
	{\F A} && {\F B}
	\arrow["{f_*}", from=1-1, to=1-3]
	\arrow["{\alpha_A}"', from=1-1, to=3-1]
	\arrow["{\alpha_B}", from=1-3, to=3-3]
	\arrow["Ff"', from=3-1, to=3-3]
\end{tikzcd}\]
conmuta. La 'ultima igualdad, $F(f)(\alpha_A(\id A))$ es el camino izquierdo del diagrama, por lo que es igual a $\alpha_B(f_*(\id A))$. Tenemos, 
\begin{align*}
    \alpha_B(f_*(\id A)) & = \alpha_B(f(\id A)) \\
                         &= \alpha_B(f)
\end{align*}
Por lo tanto $\varphi(\phi(\alpha))_B (f) = \alpha_B (f)$. Como el morfismo fue elegido arbitrariamente concluimos que $\varphi(\phi(\alpha))_B = \alpha_B$, y como $B$ fue elegido arbitrariamente, concluimos que para todo morfismo de la colecci'on que conforman a $\alpha$ y a $\varphi(\phi(\alpha))$ se cumple la igualdad. Como coinciden en todos sus componentes, es el caso que son la misma transformaci'on natural. Por lo tanot, $\varphi(\phi(\alpha))=\alpha$. De esto concluimos que las funciones $\phi$ y $\varphi$ son inversas y por lo tanto $\phi$ es una biyecci'on. 
\end{proof}

Existe la versi'on del lema de Yoneda para pregavillas, es decir, tenemos 

\begin{coro}[Contravariante del lema de Yoneda] 
     Para cualquier funtor $\ff{\F}{\opp C}{\set}$, con $\opp C$ una categor'ia localmente pequeña, y cualquier objeto $A$ en $\ob C$, la siguiente funci'on es una biyecci'on:
    \[ \varphi \colon \Nat(\homo{-}{A}),\F)\cong \F A,    \] 
    \[ \varphi(\alpha) = \alpha_A(\invopp{\id A})\]
\end{coro}
\begin{proof}
    La demostraci'on de este teorema es completamente an'aloga a la demostraci'on del lema de Yoneda en su versi'on covariante. 
\end{proof}

Podemos extender el lema. La biyecci'on $\varphi$ de \ref{lema yoneda} es una correspondencia natural tanto en $A$ y $\F$. Pero expresar a qu'e se refiere esto en particular requiere la introducci'on de algunos conceptos. 

\begin{defi} [Categor'ia producto]
    Sean $\C$ y $\cat D$ categor'ias. Definimos la categor'ia producto $\C \times \cat D$ dada por 
    \begin{itemize}
        \item objetos que son parejas ordenadas $(c,d)$ donde $c$ es un objeto en $\ob C$ y $d$ es un objeto en $\ob D$, 
        \item morfismos que son parejas ordenadas $\ff{(f,g)}{(c,d)}{(e,f)}$ con $\ff{f}{c}{e}$ morfismo en $\C$ y $\ff{g}{d}{f}$ morfismo en $\cat D$
    \end{itemize}
    la composici'on y la identidad est'an definidas componentemente. 
\end{defi}
\newpage
\begin{defi} Dadas categor'ias $\C$ y $\cat D$, definimos la categor'ia $\cat D^{\C}$ donde 
\begin{itemize}
    \item los objetos son funtores $\ff{\F}{\cat C}{\cat D}$,
    \item los morfismos son transformaciones naturales entre funtores $\df{\C}{\cat D}{\F}{\fun G}$
    \item dado un funtor $\ff{\F}{\C}{\cat D}$ la transformaci'on natural identidad $\nat{\id{\F}}{\F}{\F}$ est'a definida por los morfismos $(\id{\F})_c = \id{\F_c}$.
    \item dadas transformaciones naturales $\nat{\alpha}{\F}{\fun G}$, $\nat{\beta}{\fun G}{\fun H}$ la composici'on $\nat{\beta\alpha}{\F}{\fun H}$ est'a definida en sus componentes $\pare{\beta\alpha}_A= \beta_A\alpha_A$.
\end{itemize}
    
\end{defi}

Una observaci'on importante de la definici'on anterior es que para garantizar que 'esta es efectivamente una categor'ia, la composici'on de los morfismos debe estar bien definida, y esto s'olo es el caso si en efecto $\beta\alpha$ es una transformaci'on natural. Consideremos $\ff{f}{A}{B}$ en $\C$, la naturalidad de $\alpha$ nos dice que los diagramas 
% https://q.uiver.app/#q=WzAsOCxbMCwwLCJcXEYgQSJdLFswLDIsIlxcRiBCIl0sWzIsMCwiXFxmdW4gRyBBIl0sWzIsMiwiXFxmdW4gRyBCIl0sWzQsMCwiXFxmdW4gRyBBIl0sWzQsMiwiXFxmdW4gRyBCIl0sWzYsMiwiXFxmdW4gSCBCIl0sWzYsMCwiXFxmdW4gSCBBIl0sWzAsMSwiXFxGIGYiLDJdLFswLDIsIlxcYWxwaGFfQSJdLFsyLDMsIlxcZnVuIEcgZiJdLFsxLDMsIlxcYWxwaGFfQiIsMl0sWzQsNSwiXFxmdW4gRyBmIiwyXSxbNSw2LCJcXGJldGFfQiIsMl0sWzQsNywiXFxiZXRhX0EiXSxbNyw2LCJcXGZ1biBIIGYiXV0=
\[\begin{tikzcd}
	{\F A} && {\fun G A} && {\fun G A} && {\fun H A} \\
	\\
	{\F B} && {\fun G B} && {\fun G B} && {\fun H B}
	\arrow["{\alpha_A}", from=1-1, to=1-3]
	\arrow["{\F f}"', from=1-1, to=3-1]
	\arrow["{\fun G f}", from=1-3, to=3-3]
	\arrow["{\beta_A}", from=1-5, to=1-7]
	\arrow["{\fun G f}"', from=1-5, to=3-5]
	\arrow["{\fun H f}", from=1-7, to=3-7]
	\arrow["{\alpha_B}"', from=3-1, to=3-3]
	\arrow["{\beta_B}"', from=3-5, to=3-7]
\end{tikzcd}\]

conmutan. Podemos unir estos dos diagramas y ver que por la naturalidad de cada componente, el rect'angulo exterior del diagrama
% https://q.uiver.app/#q=WzAsNixbMCwwLCJcXEYgQSJdLFswLDIsIlxcRiBCIl0sWzIsMCwiXFxmdW4gRyBBIl0sWzIsMiwiXFxmdW4gRyBCIl0sWzQsMCwiXFxmdW4gSCBBIl0sWzQsMiwiXFxmdW4gSCBCIl0sWzAsMSwiXFxGIGYiLDJdLFswLDIsIlxcYWxwaGFfQSJdLFsyLDMsIlxcZnVuIEcgZiJdLFsxLDMsIlxcYWxwaGFfQiIsMl0sWzIsNCwiXFxiZXRhX0EiXSxbNCw1LCJcXGZ1biBIIGYiXSxbMyw1LCJcXGJldGFfQiIsMl1d
\[\begin{tikzcd}
	{\F A} && {\fun G A} && {\fun H A} \\
	\\
	{\F B} && {\fun G B} && {\fun H B}
	\arrow["{\alpha_A}", from=1-1, to=1-3]
	\arrow["{\F f}"', from=1-1, to=3-1]
	\arrow["{\beta_A}", from=1-3, to=1-5]
	\arrow["{\fun G f}", from=1-3, to=3-3]
	\arrow["{\fun H f}", from=1-5, to=3-5]
	\arrow["{\alpha_B}"', from=3-1, to=3-3]
	\arrow["{\beta_B}"', from=3-3, to=3-5]
\end{tikzcd}\]
conmuta. De esto podemos ver que efectivamente $\beta\alpha$ es una transformaci'on natural y la composici'on est'a bien definida. 

Requerimos la construcci'on de estas dos categor'ias para hablar de una en particular. Podemos considerar dada una categor'ia $\C$ localmente pequeña, la categor'ia $\C \times \set^{\C}$. Aqu'i, podemos considerar un funtor $\ff{\Nat(\homo{-}{-},-)}{\C \times \set^{\C}}{\set}$ que recibe un objeto $(A,\F)$ y lo manda al conjunto $\Nat(\homo{A}{-}),\F)$. Es importante notar en este punto, que el hecho de que $\C$ sea localmente pequeña, no garantiza que $\set^{\C}$ sea localmente pequeña, por lo que no podr'iamos garantizar en principio que $\Nat(\homo{A}{-}),\F)$ sea un conjunto. Sin embargo, el lema de Yoneda precisamente garantiza esto, por lo que el codominio de este funtor est'a bien definido. Podemos considerar otro funtor de $\C\times \set^{\C}$ a $\set$, el funtor $\ff{\ev}{\C\times \set^{\C}}{\set}$ que manda objetos $(A,\F)$ al conjunto $\F A$. Con esta categor'ias y estos funtores en vista, podemos entender a la funci'on $\varphi$ dada en el lema de Yoneda como un isomorfismo natural $\nat{\varphi}{\Nat(\homo{-}{-},-)}{\ev}$. Es en este sentido que decimos que tanto $A$ como $\F$ son naturales en \ref{lema yoneda}. Como tal, esto es parte de la totalidad del lema de Yoneda, por lo que lo enunciamos como una \enquote{segunda parte}.

\begin{teo}[Segunda parte del Lema de Yoneda] El isomorfismo $\varphi$ visto en \ref{lema yoneda} es natural tanto en $A$ como en $\F$ cuando ambos lados son vistos como funtores de $\C\times \set^{\C}$ a $\set$. 
    
\end{teo}
\begin{proof}
    La demostraci'on consiste en dos partes. Primero, fijemos un funtor $\ff{\F}{\C}{\set}$ y consideremos un morfismo $f: A\to B$ en $\C$. Consideremos el diagrama $$\diag{\Nat(\homo{A}{-}),\F)}{\Nat(\homo{B}{-}),\F)}{\F A}{\F B}{\circ(f^*)}{\F f}{\varphi_{A,\F}}{\varphi_{B,\F}}$$
\end{proof}
Aqu'i, $\varphi_{A,\F}$ define la funci'on definida en \ref{lema yoneda} en los valores $A$ y $\F$ y del mismo modo para $\varphi_{B,\F}$; por otro lado, $\circ f^*$ es dada a trav'es de la composici'on con una transformaci'on natural, $\circ f^*(\alpha)=\nat{\alpha f^*}{\homo{A}{-}}{\homo{B}{-}}$. Evaluando el lado derecho del diagrama, 
\begin{align*}
    \varphi_{B,\F}\pare{\circ(f^*)(\alpha)} & =  \varphi_{B,\F}(\alpha f^*) \\
                                            & = (\alpha f^*)_B (\id B)\\
                                            & = \alpha_B (f)\\
                                            & \stackrel{\mathclap{\mbox{*}}}{=} \alpha_B(f_*(1_A))\\
                                            & \stackrel{\mathclap{\mbox{*}}}{=} \F (f)(\alpha_A(1_A)) \\ 
                                            & = \F(f)\varphi_{A,\F}(\alpha)\\
\end{align*}
La justificaci'on a las igualdades con un asterisco encima se encuentran en la demostraci'on de \ref{lema yoneda}. La 'ultima igualdad es precisamente el lado izquierdo del diagrama, de lo que concluimos que el diagrama conmuta. Por tanto, la funci'on es natural en $A$.

Ahora fijemos al objeto $A$ en $\ob C$ y consideremos una transformaci'on natural $\nat{\alpha}{\F}{\fun G}$ en $\set^{\C}$, es decir, $\alpha$ es un morfismo en dicha categor'ia. Consideremos el diagrama $$\diag{\Nat(\homo{A}{-},\F)}{\Nat(\homo{A}{-},\fun G}{\F A}{\fun G A}{\ev_A}{}{}{}$$


El lema de Yoneda nos dice que podemos entender cualquier objeto de una categor'ia a trav'es de c'omo se comportan todos los dem'as objetos. Por eso mismo a los morfismos tambi'en los llamamos elementos locales, como vimos en \ref{elemento local}: son puntos de vista de un objeto que queremos entender. Es decir, para saber todo lo que necesitamos de un objeto dado, basta con ver c'omo interaccionan los dem'as objetos de la categor'ia con 'este. 

Regresemos al ejemplo \ref{yonedaejem}. Entendemos ahora, por el lema de Yoneda, que 
\[\Nat(\homo{p}{-}, \F) \cong \F p\]
Nosotros sab'iamos que las transformaciones naturales estaban determinadas por los elementos de $\F p$, pero con el lema de yoneda entendemos tambi'en que los elementos de $\F p$ est'an determinados por las transformaciones naturales y que esta correspondencia es biyectiva.



\nocite{*}
\printbibliography
\end{document}
