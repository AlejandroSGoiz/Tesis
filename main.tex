\documentclass[12pt,twoside]{report}
\usepackage[activeacute,spanish]{babel}
\usepackage[width= 150mm,top=25mm,bottom=25mm]{geometry}
\usepackage{graphicx} % Required for inserting images

\usepackage{amssymb}
\usepackage{amsthm}
\usepackage{amsmath}
\usepackage{mathtools}
\usepackage{mathrsfs}
\usepackage{extpfeil}
\usepackage{enumitem}
\usepackage{microtype}
\usepackage[style=mexican]{csquotes} 

\usepackage[style= alphabetic]{biblatex}
\addbibresource{referencias.bib}


\usepackage{parskip}
\setlength{\headheight}{16pt}
\usepackage{fancyhdr}
\pagestyle{fancy}
\fancyhead{}
\fancyhead[RO,LE]{\nouppercase{\leftmark}}

\usepackage{tikz-cd}
\usetikzlibrary{decorations.pathmorphing} 
\usetikzlibrary{babel}
\usetikzlibrary{calc}

\usepackage[hidelinks]{hyperref}

\makeatletter
\renewenvironment{proof}[1][\proofname]{\par
\pushQED{\qed}%
\normalfont \topsep-1\p@\@plus6\p@\relax
\trivlist
\item\relax
{\itshape
#1\@addpunct{.}}\hspace\labelsep\ignorespaces
}{%
\popQED\endtrivlist\@endpefalse
}
\makeatother

\theoremstyle{definition}
\newtheorem{defi}{Definición}[section]
\newtheorem{ejem}[defi]{Ejemplo}
\newtheorem{ejems}[defi]{Ejemplos}

\theoremstyle{plain}
\newtheorem{prop}[defi]{Proposición}
\newtheorem{teo}[defi]{Teorema}
\newtheorem*{teo*}{Teorema}
\newtheorem{coro}[defi]{Corolario}

\newcommand{\N}[0]{\mathbb N}
\newcommand{\Q}[0]{\mathbb Q}
\newcommand{\ff}[3]{#1\colon #2\to #3}
\newcommand{\pare}[1]{\left(#1\right)}

\newcommand{\cat}[1]{\mathsf{#1}}
\newcommand{\id}[1]{1_{#1}}
\newcommand{\homc}[3]{\Hom_\cat #1\left(#2,#3\right)}
\newcommand{\homopp}[3]{\Hom_\opp #1\left(#2,#3\right)}
\newcommand{\homo}[2]{\Hom\left(#1,#2\right)}
\newcommand{\inv}[1]{#1^{-1}}
\newcommand{\invopp}[1]{#1^{\textbf{op}}}
\newcommand{\opp}[1]{\cat #1^{\textbf{op}}}
\newcommand{\fun}[1]{\mathcal{#1}}
\newcommand{\func}[3]{\fun{#1}\colon \cat #2\to \cat #3}
\newcommand{\nat}[3]{#1\colon #2\Rightarrow #3}

\newcommand\chap[1]{%
  \chapter*{#1}%
  \addcontentsline{toc}{chapter}{#1}}

\DeclareMathOperator{\sub}{Sub}
\DeclareMathOperator{\dems}{Dem}
\DeclareMathOperator{\Ob}{Ob}
\DeclareMathOperator{\set}{Set}
\DeclareMathOperator{\grp}{Grp}
\DeclareMathOperator{\topo}{Top}
\DeclareMathOperator{\mat}{Mat_{\mathbb K}}
\DeclareMathOperator{\cero}{\mathbf{0}}
\DeclareMathOperator{\uno}{\mathbf{1}}
\DeclareMathOperator{\dos}{\mathbf{2}}
\DeclareMathOperator{\poset}{Poset}
\DeclareMathOperator{\C}{\cat C}
\DeclareMathOperator{\F}{\fun F}
\DeclareMathOperator{\Hom}{Hom}
\DeclareMathOperator{\cring}{CRing}
\DeclareMathOperator{\mon}{Mon}
\DeclareMathOperator{\Nat}{Nat}
\DeclareMathOperator{\ev}{Ev}

\newcommand{\ob}[1]{\Ob(\cat{#1})}
\newcommand{\obopp}[1]{\Ob(\opp{#1})}


%Comandos en tikz
\newcommand{\df}[4]{\begin{tikzcd}[ampersand replacement=\&]
	{#1} \& {#2}
	\arrow["#3", shift left, from=1-1, to=1-2]
	\arrow["#4"', shift right, from=1-1, to=1-2]
\end{tikzcd}}
\newcommand{\diag}[8]{
\begin{tikzcd} [ampersand replacement=\&]
	#1 \&\& #2 \\
	\\
	#3 \&\& #4
	\arrow["#5", from=1-1, to=1-3]
	\arrow["#7"', from=1-1, to=3-1]
	\arrow["#8", from=1-3, to=3-3]
	\arrow["#6"', from=3-1, to=3-3]
\end{tikzcd}}


\title{Teorema del Punto Fijo de Lawvere}
\author{Alejandro Sánchez Goiz}
\date{\today}

\begin{document}

\maketitle

\tableofcontents


\chapter*{Introducci'on}
\addcontentsline{toc}{chapter}{Introducci'on}
\markboth{Introducci'on}{Introducci'on}

El desarrollo de la Teor'ia de Conjuntos de Cantor y, en general, de las matem'aticas en el siglo \textsc{xix} dio lugar a muchos argumentos que hoy conocemos como \enquote{diagonales}. Veamos, como ejemplo, el siguiente teorema.

\begin{teo*}
No existe una biyecci'on entre el conjunto de los n'umeros naturales, $\mathbb {N}$, y el intervalo real $(0,1)$.    
\end{teo*}
\begin{proof}
    Supongamos por el contrario que s'i existe una biyecci'on. Entonces hay una funci'on $\phi: 
    \N \to (0,1)$ que enumera a todos los n'umeros reales en el intervalo $(0,1)$. Es decir, podemos enlistar a todos los n'umeros del intervalo, 

    
    \begin{align*}
        &0.a_{00}a_{01}a_{02}a_{03}\dots\\
        &0.a_{10}a_{11}a_{12}a_{13}\dots\\
        &0.a_{20}a_{21}a_{22}a_{23}\dots\\
        &\vdots\\
        &0.a_{n0}a_{n0}a_{n2}a_{n3}\dots \\
        &\vdots
    \end{align*}
    donde $a_{ij}\in \{0,1,2,3,4,5,6,7,8,9\}$ para todo $i,j\in \N$. De esta forma tenemos una matriz infinita donde los renglones de ésta constituyen los elementos en el intervalo. Buscamos ahora construir un elemento en $(0,1)$ que no est'e en esta lista. Para ello, consideremos las entradas $a_{ij}$ tales que $i=j$. Si $a_{ii} = 2$, definimos $b_i = 1$. Si $a_{ii}\neq 2$, definimos $b_i=2$. Afirmamos que el n'umero $b=0.b_1b_2b_3b_4\dots$ no est'a en la lista, i.e., no es un elemento de la imagen  de $\phi$. Para ver eso, basta observar que si $b$ estuviese en la lista, existir'ia un $n
    \in \N$ tal que $\phi(n) = b$. Pero $b_n$ ser'ia un dígito distinto a $a_{nn}$ pues construimos a $b$ de tal forma que fuese distinto en dicha entrada. Como los dos n'umeros son diferentes en ese decimal, no pueden ser el mismo n'umero y por tanto $\phi(n)\neq b$. Hemos entonces construido un elemento que no est'a en la imagen de $\phi$ pero que s'i est'a en $(0,1)$. Pero supusimos que $\phi$ es biyectiva por lo que en particular es suprayectiva. Esta es una clara contradicci'on, por lo que concluimos que no puede existir una funci'on biyectiva entre $\N$ y $(0,1)$.      
\end{proof}
Podemos observar en la demostraci'on del teorema anterior que para construir al elemento $b$ usamos la diagonal principal de la matriz, i.e., nos fijamos en los elementos de la forma $a_{ii}$. 


\begin{center}
\tikzset{every picture/.style={line width=0.75pt}} %set default line width to 0.75pt        

\begin{tikzpicture}[x=0.75pt,y=0.75pt,yscale=-1,xscale=1]
%uncomment if require: \path (0,300); %set diagram left start at 0, and has height of 300


% Text Node
\draw (215,31.4) node [anchor=north west][inner sep=0.75pt]    {$ \begin{array}{l}
0.\textcolor[rgb]{0.82,0.01,0.11}{a_{00}} a_{01} a_{02} a_{03} a_{04} \cdots a_{0n} \cdots \\
0.a_{10}\textcolor[rgb]{0.82,0.01,0.11}{a_{11}} a_{12} a_{13} a_{14} \cdots a_{1n} \cdots \\
0.a_{20} a_{21}\textcolor[rgb]{0.82,0.01,0.11}{a_{22}} a_{23} a_{24} \cdots a_{2n} \cdots \\
0.a_{30} a_{31} a_{32}\textcolor[rgb]{0.82,0.01,0.11}{a_{33}} a_{34} \cdots a_{3n} \cdots \\
0.a_{40} a_{41} a_{42} a_{43}\textcolor[rgb]{0.82,0.01,0.11}{a_{44}} \cdots a_{4n} \cdots \\
\vdots \\
0.a_{n0} a_{n1} a_{n2} a_{n3} a_{n4} \cdots \textcolor[rgb]{0.82,0.01,0.11}{a_{nn}} \cdots \\
\vdots 
\end{array}$};


\end{tikzpicture}
\end{center}

El apelar a esta diagonal principal de la matriz para llegar a nuestra prueba es lo que conocemos como un \emph{argumento diagonal}. De hecho, este resultado es conocido como el \textit{argumento de la diagonal de Cantor}.

Los argumentos diagonales usualmente no son provistos de una manera tan expl'icita como en el argumento de Cantor. En realidad, muchos de estos son m'as conocidos por presentar en su desarrollo, visto metamatem'aticamente, la noci'on de autoreferencia. Para ver esto, consideremos la siguiente,

%\textbf{Paradoja de Russel:}
\subsubsection*{Paradoja de Russell}

La paradoja de Russell concierne a la teor'ia de conjuntos informal, en donde cualquier propiedad define un conjunto. Para ver la paradoja, es necesario considerar a los conjuntos que se pertenecen a s'i mismos, i.e., pensemos en todos los conjuntos $X$ tales que $X\in X$. Consideremos ahora el siguiente conjunto: 
\[R = \{X\vert X\notin X\}\]
Es decir, $R$ es el conjunto de todos los conjuntos que no se pertenecen a s'i mismos. Surge entonces la pregunta, ¿$R$ se pertenece a s'i mismo? Si $R\in R$ entonces tenemos $R\notin R$ por definici'on de $R$. Por otro lado, si $R\notin R$ entonces es un conjunto que no se pertenece a s'i mismo y por tanto $R\in R$. Podemos observar entonces que en ambos casos llegamos a 
\[R\in R \iff R\notin R\]
Esta es una clara contradicci'on, por lo que llegamos a una paradoja. 

La Paradoja de Russell nos deja ver la necesidad de formalizar la teor'ia de conjuntos y que no puede ser el caso de que cualquier propiedad defina a un conjunto. De hacerlo, nuestra teor'ia se derrumbar'ia. 

En el coraz'on de la paradoja de Russell est'a la autoferencia. Construimos un conjunto $R$ tal que al considerarse a s'i mismo lleva a una paradoja. Esta paradoja es otro ejemplo de un argumento diagonal. Dicho esto, podemos ver que en ning'un momento construimos una matriz ni apelamos a una diagonal. ¿Cómo es entonces que éste es un argumento diagonal? 

Pensemos en todos los conjuntos existentes. Organiz'emoslos tanto de manera vertical como horizontal, i.e., 

\begin{align*}
    &\quad A\quad B \quad C \quad D \quad \cdots\\
    &A\\
    &B\\
    &C\\
    &D\\
    &\vdots
\end{align*}
Asignemos valores como sigue: si el elemento horizontal pertenece al elemento vertical, ponemos un $1$. En caso contrario, ponemos un $0$. As'i por ejemplo supongamos que $A\in B$ y que $A\notin A$, tendr'iamos entonces 

\begin{align*}
    &\,\,\quad A\quad B \quad C \quad D \quad \cdots\\
    &A\quad 0\\
    &B\quad 1\\
    &C\\
    &D\\
    &\vdots
\end{align*}

Nuestra matriz entonces podr'ia lucir algo como el siguiente ejemplo, 
\[
\begin{matrix}
    & A & B & C & D & E & F & \dots\\
    A &  0 & 0 & 0 &1 & 0 &1 & \dots\\
    B & 1 & 1 & 0 &1 &  0 & 0 & \dots\\
    C&  1 & 0 & 0 & 1 & 1 & 1 & \dots\\
    D& 0 & 0 & 1 & 0 & 1 & 1 & \dots \\
    E& 1 & 0 & 1 & 0 &1 & 0 & \dots\\
    F& 1 & 1 & 0 & 0 & 1 & 0 & \dots\\
    \vdots & \vdots & \vdots & \vdots & \vdots & \vdots & \vdots & \ddots\\
\end{matrix}\]
Aquí, nos interesan los valores de la diagonal principal. Las entradas $0$ son los $X$ tales que $X\notin X$ y la teor'ia informal de conjuntos nos dice que podemos crear el conjunto de todos los conjuntos cuyo valor en esta diagonal principal es $0$. Siendo este un conjunto, ocupa un lugar en esta matriz. ¿Qué valor tiene $R$ en esta diagonal principal? Como vimos, vale $0$ si y sólo si vale $1$. La existencia de este conjunto, entonces, nos lleva a un absurdo, por lo que concluimos que tal conjunto no puede existir. Como la teoría informal de conjuntos permite la construcción de dicho conjunto, podemos ver que existe un problema fundamental con ésta.

Esta construcci'on nos permite ver que en un problema de autoreferencia permite ser representado en forma de un elemento de la diagonal principal. En s'i, entendemos que la $n-$'esima columna est'a \emph{hablando algo} de la $n-$'esima fila, es decir, en cierto modo pensamos que est'a \emph{hablando algo de s'i mismo} al compartir este valor $n$ en com'un. 

De estos casos de autoreferencia uno de los m'as famosos y el m'as rodeado de inter'es filos'ofico es el primer teorema de incompletitud de Gödel. Un planteamiento y demostraci'on formal de 'este teorema ocupar'ia demasiado espacio y saldr'ia mucho del inter'es de 'esta introducci'on, por lo que nos limitaremos dar una explicaci'on breve del teorema, en un esbozo que sigue al dado por Gödel en \parencite{gdl}.

\textbf{Primer Teorema de Incompletitud de Gödel}

Los sistemas formales que proveen los Axiomas de Peano o los de Zermelo-Fraenkel, que son capaces de construir aritmética básica de los naturales, tienen limitaciones internas respecto a qué pueden demostrar. Es decir, existen proposiciones en dichos sistemas que no pueden ser demostradas ni refutadas. Añadir estas proposiciones como axiomas no resolvería el problema: en este nuevo sistema existir'ian tambi'en proposiciones indecidibles, i.e., ni demostrables ni refutables. Este problema no es exclusivo a los dos sistemas mencionados, como veremos en el cap'itulo 3, una cantidad extensiva de sistemas formales de este tipo son incompletos, en el sentido de que tendr'an proposicones indecidibles. Es importante observar que estas proposiciones indecidibles ocurren tambi'en porque suponemos que estos sistemas son consistentes, una hipótesis necesaria para más adelante.

El argumento heurístico que provee Gödel, y que nos permitirá exhibir la diagonalidad del argumento, nace con la idea de que en un sistema formal, es inconsecuente qué signos son ocupados para describir el sistema. Es decir, la elección de símbolos como paréntesis \enquote{$()$}, conectivos l'ogicos \enquote{$\lor$}, \enquote{$\land$}, etc. es arbitraria y, por tanto, es igual de plausible usar números naturales en su lugar. De esta forma, fórmulas en nuestro sistema se convierten en sucesiones finitas de números naturales y demostraciones se vuelven sucesiones finitas de sucesiones finitas de números naturales. 

Lo increíble de esta idea es que vuelve problemas metamatemáticos, es decir, problemas que nosotros observamos respecto al sistema, en problemas de números naturales. Por ejemplo, si nosotros usamos el número $0$ en lugar del conectivo lógico $\lor$, entonces una fórmula como \enquote{$x = 0$} en el sistema se puede entender tambi'en a nivel metamatem'atico como \enquote{$x= \lor$}. Es entonces posible construir dentro del sistema una fórmula $F(v)$, donde $v$ es una sucesión finita de números, que a nivel metamatemáticos nos diga \enquote{$v$ es una f'ormula demostrable}. En su demostraci'on, Gödel construye $40$ funciones recursivas para llegar a ésta fórmula. Esta fórmula es el interés principal de hacer estos cambios de simbología. 

Ahora, para el objetivo en mano, nos concentraremos en fórmulas bien formadas con una sola variable libre, por ejemplo, como la provista arriba \enquote{$x=0$}. Podemos imaginar estas f'ormulas con una sola variable libre ordenadas en una sucesi'on, de modo que sean fácilmente identificables. En esta sucesión, denotamos a la $n-$'esima entrada por $R_n$. 

Representamos por $\sub(R_n, m)$ a la f'ormula derivada de sustitur todas las instancias de la 'unica variable libre en $R_n$ por el n'umero natural $m$. A su vez, si una fórmula es demostrable, lo denotamos por $\dems (F)$. Observemos que esto 'ultimo es la f'ormula mencionada arriba por la cual surge esta necesidad de una numeraci'on de los s'imbolos, f'ormulas, etc.  Con estas notaciones en mano, definimos el siguiente conjunto $K$: 

\[K = \{n\in \N \vert \lnot \dems(\sub(R_n,n))\}\]

Es decir, los elementos en $K$ son los naturales $n$ tales que la f'ormula resultante de substituir todas las entradas de la variable libre en la f'ormula $R_n$ (que es una f'ormula con una sola variable libre) no es demostrable en el sistema formal. 

Las nociones que conforman a este conjunto ($R_n$, $\dems$, $\sub$) son todas definibles en el sistema, por lo que la noci'on de pertenencia al conjunto $K$ es tambi'en definible dentro del sistema. Es decir, existe una f'ormula  de una sola variable libre, $S$, tal que 
\[\dems(\sub(S,n))\iff n\in K\]

Al ser una f'ormula de una sola variable libre, tiene un lugar en la secuencia, por lo que hay un $g\in \N$ para el cual $S=R_g$. Llegados a este punto tenemos ya en nuestras manos una proposici'on que no es ni demostrable ni refutable en el sistema, a saber 
\[\sub(R_g,g)\]

Antes de ver por qu'e no es demostrable ni refutable esta proposici'on, observemos del esbozo dado la presencia de un argumento diagonal. Por un lado, podemos acomodar en columnas a todas las f'ormulas de una sola variable libre y por otro, en filas a todos los n'umeros $n$. De ah'i podemos asignar $f$ a la entrada con f'ormula $R_n$ y n'umero $m$ si la f'ormula $\sub(R_n, m)$ no es demostrable en el sistema y $v$ si, por el contrario, es demostrable. Nuestra matriz entonces podr'ia lucir algo como 

\[
\begin{matrix}
    & R_1 & R_2 & R_3 & \dots & R_{n-1} & R_n & \dots\\
    1 &  f & f & f &f & v &v & \dots\\
    2 & f & v & v &v &  f & v & \dots\\
    3 &  v & v & f & v & f & f & \dots\\
    \vdots& \vdots & \vdots & \vdots & \vdots & \vdots & \vdots & \ddots \\
    n-1& v & v & v & v &v & v & \dots\\
    n& f & v & f & v & v & f & \dots\\
    \vdots & \vdots & \vdots & \vdots & \vdots & \vdots & \vdots & \ddots\\
\end{matrix}\]

Nuestro conjunto $K$ ser'ian los $n\in \N$ para los cuales el valor de la $n-$'esima fila en la diagonal principal son $f$. Entonces, ¿cuál será el valor que presente el número $g$ en la diagonal principal? Sale a la luz la similitud con la paradoja de Russell a través de esta visualización de la matriz. Por supuesto, aqu'i no lidiamos con una paradoja, sino con una limitaci'on de nuestro sistema. 

Retomando el esbozo de la demostraci'on, veamos por que $\sub(R_g,g)$ no es demostrable ni refutable. 

Primero, de ser demostrable, tendr'iamos que $\dems(\sub(R_g,g))$ y, por tanto, $g\notin K$. Pero $S= R_g$ y $S$ es tal que $\dems(\sub(S, g))\iff g\in K$, por lo que tendr'iamos $g\in K \iff g\notin K$. Esto es una clara contradicci'on, pues suponemos que nuestro sistema es consistente, por lo que no puede ser demostrable.

Segundo, de ser refutable, tendr'iamos que $\lnot \dems(\sub(R_g,g))$ por lo que por definici'on de $K$ implicar'ia que $g\in K$. Como $g\in K$, tenemos $\dems(\sub(S, g))$, pero $S$ es $R_g$, por lo que $\dems(\sub(R_g, g)$, de lo que llegamos a que ocurre tanto $\lnot \dems(\sub(R_g,g))$ como $\dems(\sub(R_g,g))$ lo cual es absurdo, porque suponemos que nuestro sistema es consistente y, por tanto, concluimos que no puede ser refutable.

Como tal, hemos construido entonces una proposici'on que no es ni demostrable ni refutable en nuestro sistema. Podemos ver en la construcci'on de 'esta la noci'on de autoreferencia. En cierto sentido, la proposici'on dice a un nivel metametam'etico \enquote{No soy demostrable en el sistema formal} 
a trav'es del sentido sem'antico que le damos a los s'imbolos. 

Por supuesto, el esbozo dado de este Teorema no es bajo ningún motivo una demostración del mismo. Muchos detalles son omitidos y la visualización de la diagonalidad en el argumento de la demostración es mucho más complicado de ver explícitamente. Sin embargo, este esbozo nos permite observar la relación que guarda con los dos ejemplos anteriores y la noción de autoreferencia que lo rodea. 

Los tres ejemplos expuestos son considerablemente distintos entre s'i en el contenido de sus proposiciones. El primero habla de la diferencia en cardinalidad entre el conjunto de los n'umeros naturales y el de los reales; el segundo, de los problemas irremediables en la formulaci'on informal de la teor'ia de conjuntos; el tercero, de las limitaciones de algunos sistemas formales consistentes que puedan desarrollar aritm'etica básica. Sin embargo, llegar a vislumbrar estos tres resultados ocupa un m'etodo muy similar y, en dos de estos casos, la motivaci'on principal surge de una noci'on de autoreferencia siendo \enquote{transmitida} a la demostraci'on. Este 'ultimo detalle es el que rodea de mucho misticismo y consideraciones filos'oficas a los Teorema de Incompletitud de Gödel y, en general, el que rodea a muchos de estos resultados obtenidos mediante argumentos diagonales de consideraciones más allá de las netamente matemáticas. Si el germen de la autoreferencia puede ser incrustado en el corazón de las matemáticas por un método tan simple, ¿acaso hay una noción más allá de la científica que las limite? 

Esta noción fue muy popular en el siglo \textsc{xx}, tras los varios avances que ocurrieron en el desarrollo de fundamentos de la matemática. En 1969, se publicaron notas de William F. Lawvere que silenciosamente derrumbarían todo este misticismo que rodeaba a estos resultados. En su artículo, llamado simplemente \emph{Diagonal Arguments and Cartesian Closed Categories}, observar'ia que todos estos argumentos son simples resultados de trabajar en tipos particulares de categor'ias. Citandolo, 

\begin{quote}
    The original aim of this article was to demystify the incompleteness theorem of Gödel and the truth-definition theory of Tarski by showing that both are consequences of some very simple algebra in the cartesian-closed setting.
\end{quote}

El logro de Lawvere con este trabajo es importante: logra unificar en un solo resultado algunos de los resultados matem'aticos m'as importantes del siglo \textsc{xx}. M'as adelante, este resultado ser'ia conocido como el \emph{Teorema del punto fijo de Lawvere}, uni'endose a una larga lista de teoremas de puntos fijos, de los cuales abundan en diferentes are'as de la matem'atica y que suelen ser de mucha importancia en el 'area respectiva. 

A pesar de esto, este teorema es muy poco conocido. Una conjetura al respecto de por qu'e esto puede ser es debido a que la teor'ia en donde toma lugar este resultado, la teor'ia de categor'ias, es relativamente nueva y no muy conocida incluso entre matem'aticos. Por ello, a pesar de la significancia del resultado, no hay mucho estudio al respecto de 'este. 

El presente trabajo tiene por intenci'on presentar este teorema de forma autocontenida, por lo que no se presupone ning'un conocimiento previo de teor'ia de categor'ias. Todos los resultados y conceptos relevantes para entender el teorema ser'an presentados en el siguiente cap'itulo. Este cap'itulo no es un curso extensivo de la teor'ia de categor'ias, solo de las herramientas necesarias para llegar al resultado deseado; el lector interesado en averiguar m'as a fondo esta teor'ia puede consultar \parencite{riehl}, \parencite{leinster}, \parencite{maclane}.

El cap'itulo 2 lidiar'a con el Teorema del Punto Fijo de Lawvere. Presentamos una versi'on modernizada y una demostraci'on m'as detallada que la presentada por Lawvere en \parencite{lawverefixed}. Introducimos aqu'i, como Lawvere hace, el concepto de categor'ias cartesianas cerradas para hacer 'enfasis en la necesidad de estar trabajando en 'estas para la aplicaci'on del teorema. Hacemos también una explicación m'as detallada de la contrapositiva al teorema, pues este resultado será muy usado para las aplicaciones que veremos m'as adelante. 

El cap'itulo 3 abarca aplicaciones del teorema a trav'es de diferentes 'areas de la matem'atica. Este cap'itulo es el m'as extenso de todos porque nuestro principal inter'es es mostrar la utilidad, relevancia e importancia que tiene el Teorema de Lawvere como una forma de unificar distintos resultados en uno solo. Aunque todas estas aplicaciones son tratadas desde una argumentaci'on categ'orica, las 'areas donde nacen estos son de distinta naturaleza y por lo tanto conocimientos previos de 'estas son 'utiles para una comprensi'on m'as adecuada. Cada problema es explicado y las demostraciones son lo m'as detalladas posibles sin que entremos mucho en la teor'ia necesaria, para permitir a los lectores la comprensi'on m'as amplia de todos los resultados. Bibliograf'ia adecuada para entender a fondo cada problema ser'a provista para cada uno de ellos para los interesados. 

En el cap'itulo 4 presentamos las conclusiones de nuestro trabajo. Al final, la lista de la bibliograf'ia en detalle es provista. 





\chapter{Teoría de Categorías}
\section{Categor'ias y tipos de morfismos}

En el estudio moderno de las matemáticas, la mayoría de áreas empiezan a través de la teoría de conjuntos. Así, por ejemplo, un grupo es un conjunto con cierta estructura añadida; una topología es un conjunto y un subconjunto del conjunto potencia con ciertos axiomas a cumplir; incluso un sistema formal parte de conjuntos de símbolos, axiomas, reglas de formación y de inferencia. El lenguaje matemático es entonces, por lo general, uno que nace de la visión conjuntista. Sin embargo, el teorema del punto fijo de Lawvere sólo puede aplicarse en categorías cartesianas cerradas. Por tanto, todos los ejemplos que veremos en el capítulo 3 serán tratados desde la visión categórica, no la conjuntista. Es necesario entonces introducir los conceptos básicos de la teoría de categorías para poder llegar al teorema deseado y poder visualizar los ejemplos a mostrar desde una nueva percepción, i.e., la categórica. El punto natural para iniciar es con la noci'on de categor'ia, d'andonos nuestra primera


\begin{defi}[Categoría]
    Una \emph{categoría} $\cat C$ consta de:
    \begin{itemize}
        \item Una colección de objetos, simbolizada por $\ob C$, 
        \item Para cada $A, B$ en $\ob C$, una colección de morfismos $f\colon A\to B$, donde $A$ es denominado el \textbf{dominio} y $B$ el \textbf{codominio}.
        \item Para cada $A$ en $\ob C$, un \textbf{morfismo identidad}, denotado por $\id A$, con dominio y codominio $A$.  
        \item Para cada par de morfismos $f\colon A\to B$, $g\colon B\to C$ un \textbf{morfismo composici'on}, $gf\colon A\to C$, tambi'en denominado el \textbf{compuesto de $f$ con $g$}.

    \end{itemize}

    tales que cumplen los siguientes dos axiomas:

    \begin{itemize}
        \item \textbf{Identidad:} Para cada $f\colon A\to B$, se cumple que $\id A f = f = f \id B$.
        \item \textbf{Asociatividad:} Para cualesquiera $f\colon A\to B$, $g\colon B\to C$, $h\colon C\to D$, tenemos que $h(gf) = (hg)f$.
    \end{itemize}

\end{defi}

    Para denotar objetos usamos intercambiablemente letras may'usculas o letras min'usculas, seg'un sea m'as conveniente en el contexto que se usen. 

    A los morfismos $f\colon A \to B$ tambi'en los denominamos como \textbf{flechas} o \textbf{mapas} entre objetos. A la colecci'on de morfismos entre dos objetos $A$ y $B$ de una categor'ia $\cat C$ la simbolizamos por $\homc{C}{A}{B}$ o bien simplemente $\homo{A}{B}$ si no existe confusi'on. Adicionalmente, los morfismos $f\colon A\to B$ tambi'en los podemos representar como $A \xlongrightarrow{f} B$. 

    En el estudio de la teor'ia de categor'ias nos veremos constantemente usando \emph{diagramas}. Someramente, un diagrama es una representaci'on visual de objetos y flechas entre estos objetos. Pensado como una gr'afica, los objetos actuan como v'ertices y las flechas como aristas. Existe, como veremos a lo largo de este cap'itulo, una ventaja en representar cuestiones de la teor'ia a trav'es de diagramas. Usualmente, nos permitir'a una nueva percepci'on del problema y puede facilitar nuestro entendimiento de 'este. As'i, por ejemplo, la composici'on $gf\colon A\to B$ puede ser representado como la afirmaci'on de que el siguiente diagrama conmuta:

% https\colon//q.uiver.app/#q=WzAsMyxbMSwwLCJCIl0sWzAsMSwiQSJdLFsyLDEsIkMiXSxbMSwyLCJnZiIsMl0sWzEsMCwiZiJdLFswLDIsImciXV0=
\[\begin{tikzcd}
	& B \\
	A && C.
	\arrow["g", from=1-2, to=2-3]
	\arrow["f", from=2-1, to=1-2]
	\arrow["gf"', from=2-1, to=2-3]
\end{tikzcd}\]

    Cuando decimos que el diagrama conmuta, nos referimos a que es lo mismo tomar el camino superior que el inferior, es decir, del objeto $A$ pasar al objeto $B$ a trav'es de $f$ y del objeto $B$ pasar a trav'es de $g$ al objeto $C$, que simplemente pasar del objeto $A$ al objeto $C$ a trav'es de $gf$.

    Para ejemplificar mejor, mostramos los axiomas de identidad y asociatividad a trav'es de diagramas. 

    \textbf{Axioma de identidad} 

    Para toda $f\colon A\to B$, el siguiente diagrama es conmutativo 

    % https\colon//q.uiver.app/#q=WzAsNCxbMCwwLCJBIl0sWzIsMCwiQSJdLFsyLDIsIkIiXSxbNCwyLCJCIl0sWzAsMSwiXFxpZCBBIl0sWzEsMiwiZiJdLFsyLDMsIlxcaWQgQiJdLFswLDIsImYiLDJdLFsxLDMsImYiXV0=
\[\begin{tikzcd}
	A && A \\
	\\
	&& B && B.
	\arrow["{\id A}", from=1-1, to=1-3]
	\arrow["f"', from=1-1, to=3-3]
	\arrow["f", from=1-3, to=3-3]
	\arrow["f", from=1-3, to=3-5]
	\arrow["{\id B}", from=3-3, to=3-5]
\end{tikzcd}\]
\newpage
\textbf{Axioma de asociatividad}

Para cualesquiera, $f\colon A\to B$, $g\colon B\to C$ y $C\to D$, el siguiente diagrama es conmutativo. 

% https\colon//q.uiver.app/#q=WzAsNCxbMCwwLCJBIl0sWzMsMCwiQiJdLFszLDIsIkMiXSxbMCwyLCJEIl0sWzAsMSwiZiJdLFsxLDIsImciXSxbMiwzLCJoIl0sWzAsMywiaChnZik9KGhnKWYiLDEseyJzdHlsZSI6eyJib2R5Ijp7Im5hbWUiOiJkb3R0ZWQifX19XSxbMCwyLCJnZiIsMSx7ImxhYmVsX3Bvc2l0aW9uIjozMCwic3R5bGUiOnsiYm9keSI6eyJuYW1lIjoiZGFzaGVkIn19fV0sWzEsMywiaGciLDEseyJsYWJlbF9wb3NpdGlvbiI6MzAsInN0eWxlIjp7ImJvZHkiOnsibmFtZSI6ImRhc2hlZCJ9fX1dXQ==
\[\begin{tikzcd}
	A &&& B \\
	\\
	D &&& C.
	\arrow["f", from=1-1, to=1-4]
	\arrow["{h(gf)=(hg)f}"{description}, dotted, from=1-1, to=3-1]
	\arrow["gf"{description, pos=0.3}, dashed, from=1-1, to=3-4]
	\arrow["hg"{description, pos=0.3}, dashed, from=1-4, to=3-1]
	\arrow["g", from=1-4, to=3-4]
	\arrow["h", from=3-4, to=3-1]
\end{tikzcd}\]

Dos observaciones son importantes en la definici'on de categor'ia. La primera, la visi'on conjuntista habitual nos har'ia pensar que los objetos son en realidad conjuntos y los morfismos, funciones. Es importante observar que esta visi'on puede privar a uno de una visi'on m'as amplia que ofrece 'esta teor'ia y que, en lo general, no es cierto que se pueda hacer dicha comparaci'on. 

La segunda observación, se hace menci'on de \emph{colecciones}, no de conjuntos en la definición. Esto es porque en principio las categor'ias que se manejan pueden recolectar una cantidad tan grande de objetos y morfismos que no pueden ser conjuntos. Sin embargo, hay casos en que podemos hablar de conjuntos en vez de colecciones para los morfismos de una categor'ia, y la distinci'on es lo suficientemente importante para valer una definici'on:

\begin{defi} Sea $\cat C$ una categor'ia. Decimos que es 
\begin{enumerate}
    \item \textbf{Localmente pequeña} si para cada par de objetos $A,B$ en $\ob C$, $\homc{C}{A}{B}$ es un conjunto. 
    \item \textbf{Pequeña} si tanto la colección de sus objetos como la colecci'on de todas sus flechas son conjuntos. 
\end{enumerate}   
\end{defi}
 
Exhibimos algunos ejemplos de categor'ias. 

\begin{ejems} \leavevmode 
    \begin{enumerate}[label=\alph*)]
        \item $\set$ es la categor'ia cuyos objetos son conjuntos y los morfismos son funciones entre conjuntos. 
        \item $\grp$ es la categor'ia cuyos objetos son grupos y los morfismos son homomorfismos entre grupos. Del mismo modo podemos definir categor'ias para grupos abelianos, Ab, para anillos unitarios, Ring, etc.
        \item Un s'olo grupo $G$ puede ser visto como una categor'ia $\cat G$, que consta de un 'unico objeto $*$ y cuyos morfismos son todos los elementos del grupo. As'i, la composici'on est'a dada por la operaci'on del grupo y el morfismo identidad de $\cat G$ es el elemento neutro del grupo. 
        \item $\topo$ es la categor'ia cuyos objetos son espacios topol'ogicos y los morfismos son las funciones continuas entre espacios topol'ogicos. 
        \item $\topo_*$ es la categor'ia cuyos objetos son espacios topol'ogicos con un punto base y los morfismos son funciones continuas que preservan puntos base. 
        \item $\mat$ es la categor'ia cuyos objetos son n'umeros naturales y cuyos morfismos son matrices con entradas elementos de un campo $\mathbb K$, es decir, dados n'umeros naturales $m,n$ un morfismo $A\colon m\to n$ es una matriz de $m\times n$ con entradas en $\mathbb K$. La composici'on est'a dada por multiplicaci'on de matrices, y los morfismos identidad son las matrices identidad. 
        \item Dado un orden parcial $(P, \leq)$ podemos tratar 'este como una categor'ia $\cat P$. Sus objetos son los elementos del orden parcial y dados dos objetos $a,b\in \ob P$, decimos que existe un morfismo $a\to b$ si y solo si $a\leq b$. Por la propiedad reflexiva del orden parcial, $a\leq a$ para todo $a\in \ob P$ por lo que existen morfismos identidad para cada objeto. La propiedad transitiva del orden define la composici'on de morfismos. 
        \item La categoría $\poset$ tiene como objetos 'ordenes parciales $(P,\leq)$ y como morfismos funciones que preservan 'ordenes, es decir, tales que si $(P, \leq_P)$ y $(Q,\leq_Q)$ son 'ordenes parciales, $f\colon (P,\leq_P)\to (Q,\leq_q)$ es una funci'on tal que para todo $p, q\in P$, $fp\leq_Q fq$ si y s'olo si $p\leq_P q$.

    \end{enumerate}
    \end{ejems}

En contraste a la teor'ia de conjuntos, aqu'i no hablamos de elementos. La pertenencia como tal no es propia en el manejo de una categor'ia. Es decir, si consideramos un objeto $X$ en la categor'ia $\set$, no podemos hablar de un elemento $a\in X$. Existe, en este sentido, una forma de hablar de elementos y pertenencia a trav'es de morfismos. Para ello consideramos las siguientes definiciones:

\begin{defi}
    Decimos que un objeto $A$ en una categor'ia $\cat C$ es 
    \begin{enumerate}
        \item \textbf{terminal}, si para cualquier objeto $X$ en $\cat C$ existe un 'unico morfismo con dominio $X$ y codominio $A$,
        \item \textbf{inicial}, si para cualquier objeto $X$ en $\cat C$ existe un 'unico morfismo con dominio $A$ y codominio $X$.
    \end{enumerate}
\end{defi}

\begin{ejems} \leavevmode 
\begin{enumerate}[label=\alph*)]
    \item En la categor'ia $\set$, cualquier unitario, i.e., conjunto con un solo elemento $\{x\}$ es un objeto terminal. En particular, el conjunto $\mathbf{1}$ que definimos como $\{\varnothing\}$ es un objeto terminal. En contraste, el 'unico objeto incial es el conjunto vac'io, $\varnothing$, que definimos como $\mathbf{0}$, por vacuidad s'olo existe una funci'on con dominio $\varnothing$ y codominio $X$ para cualquier conjunto $X$. 

    \item  En un orden parcial $(P,\leq)$ un m'inimo $a$ es tal que para todo $b\in P$, $a\leq b$. Por otro lado, un m'aximo $c$ es tal que para todo $b\in P$, $b\leq c$. Con estas nociones en mente, podemos observar que un objeto inicial en un orden parcial es un m'inimo y, por el contrario, un objeto terminal es un m'aximo.
\end{enumerate}
\end{ejems}

Con esta noci'on en mente, podemos definir un elemento y la pertenencia. Pensemos, por ejemplo, en un elemento $a$ en un conjunto $A$. En t'erminos categ'oricos, podemos definir a $a$ como el morfismo $\uno \xlongrightarrow{a} A$. Tenemos entonces la siguiente definici'on. 

\begin{defi}\label{elemento global}
    En la categor'ia $\set$, un \textbf{elemento global} de un objeto $A$ es un morfismo $\uno \xlongrightarrow{a} A$. 
\end{defi}

En particular, podemos entender as'i la noci'on de pertenencia, $\in$, en t'erminos categ'oricos. Cuando expresamos $a\in A$ es lo mismo que expresar, en forma categ'orica, que existe un morfismo $a\colon\uno\to A$.

Pensando ahora en los morfismos, hay propiedades interesantes en estos que merecen ser distinguidas. Tenemos las siguientes definiciones. 

\begin{defi}
    Decimos que un morfismo $f\colon A\to B$ es 
    \begin{enumerate}
        \item \textbf{monomorfismo} o \textbf{mono} si para cada par de morfismos $h,g \colon C\to A$ tales que $fg=fh$ esto implica $g=h$. Es decir, si el siguiente diagrama conmuta 
        % https\colon//q.uiver.app/#q=WzAsMyxbMCwwLCJDIl0sWzIsMCwiQSJdLFs0LDAsIkIiXSxbMCwxLCJnIiwwLHsib2Zmc2V0IjotMn1dLFsxLDIsImYiXSxbMCwxLCJoIiwyLHsib2Zmc2V0IjoyfV1d
\[\begin{tikzcd}
	C && A && B
	\arrow["g", shift left=2, from=1-1, to=1-3]
	\arrow["h"', shift right=2, from=1-1, to=1-3]
	\arrow["f", from=1-3, to=1-5]
\end{tikzcd}\]
entonces, $g=h$.
        \item \textbf{epimorfismo} o $\textbf{epi}$ si para cada par de morfismos $h,g\colon B\to C$ tales que $gf=hf$ esto implica $g=h$. Es decir, si el siguiente diagrama conmuta, 
        % https\colon//q.uiver.app/#q=WzAsMyxbMCwwLCJDIl0sWzIsMCwiQiJdLFs0LDAsIkEiXSxbMSwwLCJnIiwyLHsib2Zmc2V0IjoyfV0sWzIsMSwiZiIsMl0sWzEsMCwiaCIsMCx7Im9mZnNldCI6LTJ9XV0=
\[\begin{tikzcd}
	C && B && A
	\arrow["g"', shift right=2, from=1-3, to=1-1]
	\arrow["h", shift left=2, from=1-3, to=1-1]
	\arrow["f"', from=1-5, to=1-3]
\end{tikzcd}\]
entonces $g=h$.
        \item \textbf{invertible} si existe $g\colon B\to A$ tal que $fg= \id B$ y $gf= \id A$. En este caso decimos que $f$ es un \textbf{isomorfismo} y que $A$ y $B$ son \textbf{isomorfos}. Es decir, existe $g$ tal que el siguiente diagrama conmuta 
        % https\colon//q.uiver.app/#q=WzAsNSxbMCwwLCJBIl0sWzIsMCwiQiJdLFswLDIsIkEiXSxbMiwyLCJCIl0sWzIsMywiQiJdLFswLDEsImYiXSxbMCwyLCJcXGlkIEEiXSxbMSwzLCJcXGlkIEIiLDJdLFsyLDMsImYiLDJdLFsxLDIsImciLDFdXQ==
\[\begin{tikzcd}
	A && B \\
	\\
	A && B. \\
	\arrow["f", from=1-1, to=1-3]
	\arrow["{\id A}"', from=1-1, to=3-1]
	\arrow["g"{description}, from=1-3, to=3-1]
	\arrow["{\id B}", from=1-3, to=3-3]
	\arrow["f"', from=3-1, to=3-3]
\end{tikzcd}\]
    \end{enumerate}
\end{defi}
\begin{ejems}  \label{iso} \leavevmode
    
    \begin{enumerate}[label=\alph*)]
        \item En $\set$ los monomorfismos son las funciones inyectivas, los epimorfismos son las funciones suprayectivas y los isomorfismos son las funciones biyectivas. 
        \item En $\topo$ los isomorfismos son los homeomorfismos. 
        \item  En un orden parcial $(P, \leq)$, todas las flechas son tanto monomorfismos como epimorfismos. Como las flechas $a\to b$ se dan si y s'olo si $a\leq b$, cada flecha $a\to b$ es 'unica. Por contraste, los 'unicos isomorfismos son las flechas identidades. En efecto, si tenemos $a\to b$ invertible,  esto implica que existe una flecha $b\to a$. Tenemos entonces $a\leq b$ y $b\leq a$, por lo que $a=b$ por antisimetr'ia. 
        \item En un grupo $G$ considerado como categor'ia, todos los morfismos son isomorfismos. Recordemos que un grupo visto como categor'ia consta de un 'unico objeto $*$ y los elementos del grupo act'uan como morfismos, mientras que el elemento neutro actua como el morfismo identidad del objeto $G$. Recordemos que para cada elemento $g$ del grupo $G$, existe un inverso $h$ bajo la operaci'on del grupo tal que $g\cdot h = e$, con $\cdot$ la operaci'on del grupo y $e$ el elemento neutro. De esto vemos que cada morfismo tiene un inverso que da el morfismo identidad, por lo que to1546546564641064158 b ndos son isomorfismos. 
    \end{enumerate}
\end{ejems}

El ejemplo anterior ayuda a exhibir una noci'on importante que pudiese confundir. En la teor'ia de conjuntos, una funci'on biyectiva es tanto suprayectiva como inyectiva. Conversamente, si una funci'on es tanto suprayectiva como inyectiva, es biyectiva. En teor'ia de categor'ias el an'alogo es cierto para la primera parte: si un morfismo es isomorfismo es tambi'en epimorfismo y monomorfismo. Pero como el inciso c) del ejemplo~\ref{iso} nos muestra, el converso no es cierto: un morfismo que es tanto epi como mono no implica que sea isomorfismo. 

Procedemos a demostrar la afirmaci'on hecha.

\begin{prop} Si $f\colon A\to B$ es un isomorfismo, entonces es tambi'en un epimorfismo y un monomorfismo. 
\end{prop}
\begin{proof}
    Demostremos primero que es un monomorfismo. Sean entonces $g,h\colon C\to A$ morfismos tales que $fg=fh$. Como $f$ es isomorfismo, existe $\inv f\colon B\to A$ tal que $\inv f f = \id A$. Tenemos entonces 
    \[
        g = \id A g = (\inv f f) g=  \inv f(fg) = \inv f (fh) = (\inv f f) h = \id A h = h 
   \]
   por lo tanto $g=h$. Probar que $f$ es epimorfismo es completamente an'alogo a como procedimos. 
\end{proof}
En el uso de la prueba anterior usamos la notaci'on $\inv f$ para hablar de un morfismo inverso para $f$. Esto se debe a que dicho morfismo inverso, de existir, es 'unico. Demostramos esta afirmaci'on. 

\begin{prop}
    Si $f$ es un isomorfismo, entonces existe un 'unico morfismo inverso. 
\end{prop}
\begin{proof}
    Sea $f\colon A\to B$ isomorfismo y $g,h\colon B\to A$ inversas de $f$. Tenemos entonces 
    \[g = g\id B = g(fh) = (gf)h = \id A h = h \]
    Por lo tanto $g=h$.
\end{proof}
Por tanto nuestra notaci'on $\inv f$ esta debidamente justificada. 

Concluimos esta secci'on construyendo una categor'ia a partir de una dada $\cat C$. Dada una flecha $a\to b$, podemos pensar en el proceso de invertir la direcci'on de 'esta. En este caso, tendr'iamos $b\to a$. El proceso de \enquote{revertir todas las flechas} produce lo que conocemos como un proceso \emph{dual}. Este procedimiento nos permite definir la siguiente categor'ia: 

\begin{defi} [Categor'ia Opuesta]
    Sea $\cat C$ una categor'ia. La categor'ia opuesta $\opp C$ tiene:
    \begin{itemize}
        \item como objetos $\obopp C$ los mismos objetos que $\ob C$,
        \item como morfismos los mismos que en $\cat C$ pero con el orden de las flechas invertidos, dado un morfismo $f\colon A\to B$ en $\cat C$, el morfismo correspondiente en $\opp C$ es $\invopp f \colon B\to A$. 
        \item como morfismos identidad los mismos morfismos identidad que en $\cat C$.
        \item la composici'on en $\opp C$ funciona igual que en $\cat C$. Dados morfismos $f\colon A\to B$, $g\colon B\to C$ su composici'on $gf\colon A\to C$ da lugar la composici'on $\invopp{(gf)} = \invopp f \invopp g\colon C\to A$.
    \end{itemize}
\end{defi}
Es evidente que los axiomas de identidad y asociatividad se siguen en $\opp C$ si y solo si la $\cat C$ es una categor'ia. Esta construcci'on y, en general, las construcciones duales que veremos m'as adelante ser'an de gran utilidad para el teorema del punto fijo de Lawvere. 



%Aqui finaliza la primera seccion 
%--------------------------------------------------------------------------------------------------

\section{Funtores y Transformaciones Naturales}

La noci'on de categor'ia nos invita a pensar en objetos matem'aticos y los morfismos que permiten preservar su estructura. Consideramos por ejemplo grupos. Los morfismos que preservan su estructura son los homomorfismos; pensamos en espacios topol'ogicos, cuyos morfismos que preservan estructura son funciones continuas. Una categor'ia es tambi'en un objeto matem'atico, por lo que podemos preguntarnos en qu'e morfismos preservan la estructura de una categor'ia. Los funtores son precisamente estos morfismos. 

\begin{defi}[Funtor]\label{funtor}
Sean $\cat C$, $\cat D$ categor'ias. Un \textbf{funtor} $\func {F}{C}{D}$ consta de 
\begin{itemize}
    \item objetos $\fun F C$ en $\ob D$ para cada $C$ en $\ob C$,
    \item morfismos $\fun F f\colon \fun F A\to \fun F B$ en $\cat D$ para cada morfismo $f\colon A\to B$ en $\cat C$. Es decir, para cada colecci'on $\homc{C}{A}{B}$ corresponde una colecci'on $\homc{D}{\fun FA}{\fun FB}$.
\end{itemize}
    Tales que:
    \begin{itemize}
        \item para cada $A$ en $ \ob C$, $\fun F(\id A) = \id{\fun F A}$,
        \item para cada $f\colon A\to B$, $g\colon B\to C$ en $\cat C$, $\fun F g\fun F f = \fun F(gf)$
    \end{itemize}
\end{defi}
\begin{ejems} \label{ejemplos funtores}
    \leavevmode
    \begin{enumerate}
        \item El funtor $\fun F\colon \grp\to \set$ llamado \emph{olvidadizo} pues \enquote{olvida} la estructura de grupo y deja tan solo al conjunto de elementos; olvida tambi'en que las funciones son homomorfismos y simplemente las convierte a funciones sin esa noci'on. 
        \item  $\fun P\colon \set \to \set$ es un funtor que manda a cada conjunto $X$ a su conjunto potencia $P(X)$ y que manda toda funci'on $f\colon X\to Y$ a $\fun Pf\colon P(X)\to P(Y)$, esta funci'on manda a cada $A\subset X$ a la imagen $f(A)\in P(Y)$.
        \item Dados 'ordenes parciales $(P,\leq_P)$, $(Q,\leq_Q$) un funtor $\fun F\colon P\to Q$ es una funci'on que preserva 'ordenes, es decir, manda objetos $p, q\in P$ a objetos $\fun F p, \fun F q \in Q$ de tal manera que $p\leq_P q$ si y s'olo si $\fun F p \leq_Q \fun F q$. Observemos que estos funtores son los morfismos en la categor'ia $\poset$. 
        \item El funtor $\pi_1 \colon \topo_*\to \grp$ que manda todo espacio topol'ogico $X$ con punto base $x$ a su grupo funamental $\pi_1(X,x)$ y a toda funci'on continua $f\colon (X,x)\to (Y,y)$ a un homomorfismo de grupos $\pi_1(f)\colon \pi_1(X,x)\to \pi_1(Y,y)$. 
        \item Sea $\cat C$ una categor'ia, el funtor identidad $\id {\cat C}\colon \cat C\to \cat C$ manda a objetos $C$ y morfismos $f\colon C\to D$ en s'i mismos.  
    \end{enumerate}
\end{ejems}

También denominamos a este tipo de funtor como \textbf{funtor covariante}, para distinguirlo de otro tipo de funtor. 

\begin{defi}[Funtor contravariante] \label{contravariante}
    Sean $\cat C$, $\cat D$ categor'ias. Un \textbf{funtor contravariante} de $\cat C$ a $\cat D$ es un funtor $\fun F\colon \opp C \to \cat D$. Siendo m'as expl'icitos, la diferencia principal de 'este y el funtor covariante de $\cat C$ en $\cat D$ es que morfismos $f\colon A\to B$ en $\cat C$ son mandados a morfismos $\fun F\colon \fun F B\to \fun F A$ en $\cat D$. Adem'as, si el codomonio del funtor es $\set$, decimos que el funtor es una \textbf{pregavilla}.
\end{defi}
\begin{ejem}\label{homfunctor} Sea $\cat C$ una categor'ia localmente pequeña y $C$ en $\ob C$. El funtor contravariante $\homc{C}{-}{C}\colon \opp C\to \set$ asigna objetos $A$ a conjuntos $\homc{C}{A}{C}$ y dado un morfismo $f\colon A\to B$ en $\homc{C}{A}{B}$ le asigna a 'este un morfismo $\homc{C}{-}{C} f\colon \homc{C}{B}{C}\to \homc{C}{A}{C}$ dado por $g\mapsto gf$.

Podemos, en una manera similar, definir un funtor covariante de $\cat C$ en $\set$, denotado $\homc{C}{C}{-}$, que asigna objetos $A$ en $\ob C$ a conjuntos $\homc{C}{C}{A}$ y morfismos $f\colon A\to B$ a morfismos $\homc{C}{C}{-}\colon \homc{C}{C}{A}\to \homc{C}{C}{B}$ dado por $g\mapsto fg$.
\end{ejem}

Como mencion'abamos al inicio de la secci'on, podemos entender a los funtores como morfismos entre categor'ias, por lo que naturalmente surge la pregunta de si existen nociones similares a epimorfismo, monomorfismo o isomorfismo para funtores. Tenemos nociones similares en categor'ias localmente pequeñas, que definimos a continuación.

\begin{defi}
    Sean $\cat C,\cat D$ categor'ias localmente pequeñas y $\func{F}{C}{D}$ un funtor entre ellas. Decimos que el funtor es:

    \begin{enumerate}
        \item \textbf{fiel} si para cada $A,B$ en $\ob C$, la funci'on $\homc{C}{A}{B}\to \homc{D}{\F A}{\F B}$ es inyectiva;
        \item \textbf{pleno} si para cada $A,B$ en $\ob C$, la funci'on $\homc{C}{A}{B}\to \homc{D}{\F A}{\F B}$ es suprayectiva;
        \item \textbf{esencialmente suprayectivo} si para cada $A$ en $\ob C$ existe una $D$ en $\ob D$ tal que $D$ es isom'orfica a $\F A$.
    \end{enumerate}
    \end{defi}
    Notemos aqu'i que estamos hablando de funciones, inyectividad y suprayectividad, por ello es que requerimos que las categor'ias en cuesti'on sean localmente pequeñas. Por otro lado, la noción de isomorfismo es idéntica a como vimos con morfismos. 
    \begin{defi} Sean $\cat C,\cat D$ categor'ias. Un funtor $\func{F}{C}{D}$ es un \textbf{isomorfismo} si existe un funtor $\func{G}{D}{C}$ tal que $\F \fun G = \id{\cat D}$ y $\fun G \F = \id{\cat C}$. En este caso decimos que $\F F$ y $\fun G$ son \textbf{isomorfos} y lo denotamos por $\F\cong \fun G$.
        
    \end{defi}

    Si los funtores los podemos entender como morfismos entre categor'ias, es igual de v'alido preguntarse sobre morfismos entre funtores. Hist'oricamente, este tipo de morfismos fueron los primeros en ser estudiados y que derivaron en la creaci'on de la teor'ia de categor'ias. Antes de definirlos propiamente, veamos un ejemplo en donde su presencia es 'util. 

    \begin{ejem}
        Sea $n \in \N$ y consideremos cualquier anillo conmutativo unitario $R$. Consideremos ahora las matrices $n\times n$ con entradas en $R$. Junto con la multiplicaci'on de matrices, este conjunto forma un monoide, que denotamos por $M_n(R)$. Por otro lado, si conservamos s'olo la multiplicaci'on del anillo $R$ y olvidamos la suma, obtenemos tambi'en un monoide que podemos denotar por $U(R)$. A partir de un anillo $R$ entonces tenemos dos maneras de llegar a un monoide, una definida por $M_n(R)$ y otra por $U(R)$.

        Recordemos ahora que cualquier matriz cuadrada $A$ en cualquier anillo $R$ posee un determinante $\det(A)$ con valores en $R$ y que, en particular, cumple las siguientes dos propiedades: 
        \[\det(A)\det(B) = \det(AB)\qquad \qquad \det(I_m) = 1_R\]
        donde $I_m$ es la matriz identidad de las matrices de $m\times m$ con $m\in \N$ y $1_R$ es la identidad bajo multiplicaci'on de $Q$. Observemos que estas mismas dos propiedades son las que determinan un homomorfismos de monoides y que $\det$ toma matrices cuadradas con entradas en $R$ y devuelve valores en $R$. Por lo tanto podemos considerar el homomorfismo $\det\colon M_n(R) \to U(R)$, para cada anillo $R$. 

       Consideremos un homomorfismo de anillos conmutativos $f\colon R\to Q$. Si preservamos 'unicamente la estructura de multiplicaci'on, la funci'on en cuesti'on es ahora un homomorfismo de monoides. Podemos denotar esta nueva funci'on como $U(f)\colon U(R)\to U(Q)$. Es decir, para cada anillo conmutativo $R$ tenemos un monoide $U(R)$ y para cada homomorfismo de anillos conmutativos $f\colon R\to Q$ tenemos un homomorfismo de monoides $U(f)\colon U(R)\to U(Q)$. Del mismo modo, todo homomorfismo de anillos $f\colon R\to Q$ induce un homomorfismos de monoides $M_n(f)\colon M_n(R)\to M_n(Q)$. Del estudio que llevamos al momento salta a la luz la noci'on de funtor. En efecto, podemos considerar la categor'ia $\cring$ cuyos objetos son anillos conmutativos unitarios y sus morfismos son homomorfismos de anillos conmutativos y la categor'ia $\mon$ cuyos objetos son monoides y cuyos morfismos son homomorfismos de monoides. Tenemos entonces dos funtores $M_n\colon \cring \to \mon$ y $U\colon \cring \to \mon$. 

       Con estas nociones en mente, consideremos el siguiente diagrama 
       % https\colon//q.uiver.app/#q=WzAsNCxbMCwwLCJNX24oUikiXSxbMywwLCJNX24oUSkiXSxbMCwzLCJNKFIpIl0sWzMsMywiTShRKSJdLFswLDEsIk1fbihmKSJdLFswLDIsIlxcZGV0X1IoTV9uKFIpKSIsMl0sWzIsMywiTShmKSIsMl0sWzEsMywiXFxkZXQoTV9uKFEpKSJdXQ==
\[\begin{tikzcd}
	{M_n(R)} &&& {M_n(Q)} \\
	\\
	\\
	{U(R)} &&& {U(Q)}
	\arrow["{M_n(f)}", from=1-1, to=1-4]
	\arrow["{\det(M_n(R))}"', from=1-1, to=4-1]
	\arrow["{\det(M_n(Q))}", from=1-4, to=4-4]
	\arrow["{U(f)}"', from=4-1, to=4-4]
\end{tikzcd}\]
Podemos observar que $\det$ act'ua como un morfismo entre estos dos funtores. La primera pregunta que podr'ia salirnos en mente es si el diagrama es conmutativo, i.e., tenemos que $U(f)\det(M_n(R)) = \det(M_n(Q))M_n(f)$. Podemos verificar r'apidamente que el diagrama conmuta considerando los elementros neutros de los monoides y recordando que todas estas funciones son homomorfismos de monoides. 

La elecci'on de los valores aqu'i es arbitraria, por lo que vemos que para la familia de morfismos $\pare{\det\colon M_n(R)\to U(R)}_{R\in \cring}$ cualquier diagrama como el mostrado arriba ser'a conmutativo. Si consideramos entonces a $\det\colon M_n\to M$, con $\det$ la famlilia mencionada $\pare{\det\colon M_n(R)\to U(R)}_{R\in \cring}$ vemos que para cada homomorfismos de anillos $f\colon R\to Q$, el diagrama mostrado arriba conmutar'a. Esto refleja que la noci'on de determinante es uniforme para todos los anillos, i.e., no es dependiente de c'omo est'a determinado un anillo en particular. Esta es la idea fundamental detr'as de una transformaci'on natural, la cual procedemos a definir formalmente. 
\end{ejem}

\begin{defi}[Transformaci'on Natural]
Sean $\C$ y $\cat D$ categor'ias, y \df{\C}{\cat D}{\F}{\fun G} funtores. Una \textbf{transformaci'on natural} $\alpha\colon \F \Rightarrow \fun G$ consiste en una colecci'on de morfismos $\alpha_c\colon \F c\to \fun G c$ para cada objeto $c$ en $\ob C$ tales que para cada morfismo $f\colon c\to d$ en $\C$, el siguiente diagrama de morfismos en $\cat D$ es conmutativo:

% https\colon//q.uiver.app/#q=WzAsNCxbMCwwLCJGYyJdLFsyLDAsIkZkIl0sWzAsMiwiR19jIl0sWzIsMiwiR2QiXSxbMCwxLCJGZiJdLFswLDIsIlxcYWxwaGFfYyIsMl0sWzIsMywiR2YiLDJdLFsxLDMsIlxcYWxwaGFfZCJdXQ==
\[\begin{tikzcd}
	Fc && Fd \\
	\\
	{G_c} && Gd
	\arrow["Ff", from=1-1, to=1-3]
	\arrow["{\alpha_c}"', from=1-1, to=3-1]
	\arrow["{\alpha_d}", from=1-3, to=3-3]
	\arrow["Gf"', from=3-1, to=3-3]
\end{tikzcd}\]
Si adem'as cada uno de los morfismos $\alpha_c$ es un isomorfismos, entonces decimos que es un \textbf{isomorfismo natural} y lo denotamos por $\alpha\colon \F \cong \fun G$. A los morfismos $\alpha_c$ tambi'en los denotamos como los \textbf{componentes} de la transformaci'on natural. 
\end{defi}
\begin{ejems} \leavevmode
\begin{enumerate}[label=\alph*)]
    \item Los funtores contravariantes $\fun O, \fun C\colon \begin{tikzcd}
	{\topo^{op}} & \set
	\arrow[shift left, from=1-1, to=1-2]
	\arrow[shift right, from=1-1, to=1-2]
\end{tikzcd}$, donde $\fun O$ es el funtor que lleva espacios topol'ogicos $X$ a la familia de conjuntos abiertos de $X$ y que a funciones continuas $f\colon X\to Y$ las lleva a $\fun O f\colon \fun O Y\to \fun O X$, con $\fun O f$ la funci'on que lleva a un subconjunto $U\subset Y$ a su preimagen bajo $f$, $\inv f(U)$ y $\fun C$ es el funtor an'alogo pero llevando los espacios a la familia de conjuntos cerrados de la topolog'ia son naturalmene isom'orficos, con la transformaci'on natural $\alpha\colon \fun O \to \fun C$, donde los morfismos $\alpha_X$ para cada $X\in \topo$ llevan a los conjuntos abiertos a su complemento, que es un conjunto cerrado. Para cada uno de estos, su inverso $\inv\alpha_X$ est'a dado como el morfismo que lleva los conjuntos cerrados de $X$ a su complemento, que es un conjunto abierto. 

\item Recordemos que dados 'ordenes parciales $(P,\leq_P)$, $(Q,\leq_Q)$ vistos como categor'ias, un funtor entre ellos es una funci'on que preserva 'ordenes. Consideremos funtores $\fun F, \fun G$ con dominio $P$ y codominio $Q$. Si para todo $p\in P$ tenemos que $\F p \leq_Q \fun G p$  podemos definir morfismos $\alpha_p\colon \F p \to \fun G p$ tales que el diagrama
% https\colon//q.uiver.app/#q=WzAsNCxbMCwwLCJcXGZ1bmYgcCJdLFsyLDAsIlxcZnVuZiBxIl0sWzAsMiwiXFxmdW4gRyBwIl0sWzIsMiwiXFxmdW4gRyBxIl0sWzAsMV0sWzAsMiwiXFxhbHBoYV9wIiwyXSxbMiwzXSxbMSwzLCJcXGFscGhhX3EiXV0=
\[\begin{tikzcd}
	{\F p} && {\F q} \\
	\\
	{\fun G p} && {\fun G q}
	\arrow[from=1-1, to=1-3]
	\arrow["{\alpha_p}"', from=1-1, to=3-1]
	\arrow["{\alpha_q}", from=1-3, to=3-3]
	\arrow[from=3-1, to=3-3]
\end{tikzcd}\]
conmuta, gracias a la transitividad del orden parcial $(Q,\leq_q)$. Por otro lado, la existencia de morfismos $\alpha_p\colon \F p \to \fun G p$ implica que $\F p\leq_Q \fun G p$, por lo que vemos que una transformaci'on natural entre los funtores $\F, \fun G$ existe si y solo si $\F p\leq_Q \fun G p$ para todo $p\in P$. 
\end{enumerate}
    
\end{ejems}

Las transformaciones naturales empiezan a resaltar la importancia de los diagramas para representar la informaci'on dada. Nos permiten entender de una manera m'as directa lo que est'a ocurriendo a trav'es de los objetos y los morfismos. Por ejemplo, dado el diagrama 
% https\colon//q.uiver.app/#q=WzAsMyxbMCwwLCJBIl0sWzAsMiwiQyJdLFsxLDEsIkIiXSxbMCwxLCJnZiIsMl0sWzAsMiwiZiJdLFsyLDEsImciXV0=
\[\begin{tikzcd}
	A \\
	& B \\
	C
	\arrow["f", from=1-1, to=2-2]
	\arrow["gf"', from=1-1, to=3-1]
	\arrow["g", from=2-2, to=3-1]
\end{tikzcd}\]
decir que es conmutativo representa la idea de la composici'on en una categor'ia. Ahora, un funtor lo podemos entender como el traslado o mapeo de la informaci'on de la categor'ia a otra dada y la propiedad $\F (gf) = \F g \F f$ implica que el mapeo del diagrama conserva la conmutatividad, es decir, el tri'angulo derecho del siguiente diagrama tambi'en es conmutativo.

% https\colon//q.uiver.app/#q=WzAsOCxbMCwwLCJBIl0sWzAsMiwiQyJdLFsxLDEsIkIiXSxbNCwwLCJGQSJdLFs0LDIsIkZDIl0sWzUsMSwiRkIiXSxbMiwxXSxbMywxXSxbMCwxLCJnZiIsMl0sWzAsMiwiZiJdLFsyLDEsImciXSxbMyw0LCJGZ2YiLDJdLFszLDUsIkZmIl0sWzUsNCwiRmciXSxbNiw3LCJGIl1d
\[\begin{tikzcd}
	A &&&& FA \\
	& B & {} & {} && FB \\
	C &&&& FC
	\arrow["f", from=1-1, to=2-2]
	\arrow["gf"', from=1-1, to=3-1]
	\arrow["Ff", from=1-5, to=2-6]
	\arrow["Fgf"', from=1-5, to=3-5]
	\arrow["g", from=2-2, to=3-1]
	\arrow["F", from=2-3, to=2-4]
	\arrow["Fg", from=2-6, to=3-5]
\end{tikzcd}\]

Finalmente podemos entender a una transformaci'on natural como un mapeo o traslado de la informaci'on de un funtor $\F$ a otro dado $\fun G$, de tal modo que tambi'en el diagrama 

% https\colon//q.uiver.app/#q=WzAsMTEsWzAsMCwiQSJdLFswLDIsIkMiXSxbMSwxLCJCIl0sWzQsMCwiRkEiXSxbNCwyLCJGQyJdLFs1LDEsIkZCIl0sWzIsMV0sWzMsMV0sWzcsMCwiR0EiXSxbNywyLCJHQyJdLFs5LDEsIkdCIl0sWzAsMSwiZ2YiLDJdLFswLDIsImYiXSxbMiwxLCJnIl0sWzMsNCwiRmdmIiwyXSxbMyw1LCJGZiJdLFs1LDQsIkZnIl0sWzYsNywiRiJdLFszLDgsIlxcYWxwaGFfQSJdLFs1LDEwLCJcXGFscGhhX0IiLDEseyJsYWJlbF9wb3NpdGlvbiI6MzB9XSxbOCw5LCJHZ2YiLDAseyJsYWJlbF9wb3NpdGlvbiI6MzB9XSxbOCwxMCwiR2YiXSxbMTAsOSwiR2ciXSxbNCw5LCJcXGFscGhhX0MiLDJdXQ==
\[\begin{tikzcd}
	A &&&& FA &&& GA \\
	& B & {} & {} && FB &&&& GB \\
	C &&&& FC &&& GC
	\arrow["f", from=1-1, to=2-2]
	\arrow["gf"', from=1-1, to=3-1]
	\arrow["{\alpha_A}", from=1-5, to=1-8]
	\arrow["Ff", from=1-5, to=2-6]
	\arrow["Fgf"', from=1-5, to=3-5]
	\arrow["Gf", from=1-8, to=2-10]
	\arrow["g", from=2-2, to=3-1]
	\arrow["F", from=2-3, to=2-4]
	\arrow["{\alpha_B}"{pos=0.3}, from=2-6, to=2-10]
	\arrow["Fg", from=2-6, to=3-5]
	\arrow["Gg", from=2-10, to=3-8]
	\arrow["{\alpha_C}"', from=3-5, to=3-8]
    \arrow["Ggf"{pos=0.3}, from=1-8, to=3-8, crossing over]
\end{tikzcd}\]

es conmutativo. Expresar toda esta información a través de igualdad de composiciones no sólo sería tedioso y tardado, sino también confuso en la entendimiento de las ideas expuestas. Hay una clara ventaja pedagógica en el apoyo de los diagramas para expresar nuestras ideas y, como veremos, nuestras demostraciones. La sección que sigue es un ejemplo perfecto de esto, y es también uno de los resultados centrales en la teoría de categorías. 

%Aquí finaliza la segunda sección
%---------------------------------------------------------------------------------------------------

\section{Lema de Yoneda}

El Lema de Yoneda es uno de los resultados m'as importantes de la teor'ia de categor'ias. La primera aparici'on del lema en la literatura es con Grothendieck en \cite{lemayoneda}, aunque el resultado se la atribuye a Nobuo Yoneda. Para entender adecuadamente el resultado, necesitamos introducir algunos resultados. Primeramente, recordemos la definici'on \ref{elemento global} de un elemento global. Esta noci'on es m'as tradicional a como entendemos la pertenencia y lo denotamos global porque no depende de ning'un punto de vista particular. Pero podemos extender la noci'on conjuntista de pertenencia y pensar en puntos de vista o de \enquote{referencia} desde los cuales podemos observar a un objeto. En la definici'on de elemento global usamos de punto de partida el objeto $\uno$, pero podemos observar al elemento desde otro punto de partida. Esto nos lleva a la siguiente definici'on 
\begin{defi}[Elemento local] \label{elemento local}
    Sea $A$ un objeto no isom'orfico a $\uno$. Decimos que un morfismo $A\to B$ es un \textbf{elemento local} de $B$ en la \textbf{etapa} $A$. 
\end{defi}

Esta noci'on de obsevar un elemento en cierta \enquote{etapa} ser'a 'util para entender lo que el lema de Yoneda dice. 

Recordemos ahora los funtores vistos en el ejemplo \ref{homfunctor}. Nos referimos a estos como \textbf{Hom-funtores}, en particular $\ff{\homo{A}{-}}{\C}{\set}$ es el Hom-funtor covariante y $\ff{\homo{-}{A}}{\opp C}{\set}$ es el Hom-funtor contravariante. Estos dos funtores son particularmente importantes porque nos permiten hablar de la noci'on de \enquote{representabilidad} que definimos a continuaci'on. 

\begin{defi}[Representable]
    Sea $\C$ una categor'ia. Un funtor $\func{F}{C}{\set}$ es (covariablemente) \textbf{representable} si es naturalmente isom'orfico a un Hom-funtor $\ff{\homo{A}{-}}{\C}{\set}$ para alg'un objeto $A$ en $\ob C$. Del mismo modo, un funtor $\ff{\fun G}{\opp C}{\set}$ es (contravariablemente) \textbf{representable} si es naturalmente isom'orfico a un Hom-funtor contravariante $\ff{\homo{-}{A}}{\opp C}{\set}$ para un objeto $A$ en $\ob C$. En ambos casos, tal objeto $A$ es llamado un \textbf{objeto representante} del funtor $\F$ o $\fun G$, seg'un corresponda. El objeto junto con la transformaci'on natural que induce el isomorfismo son llamados una \textbf{representaci'on del funtor} $\F$ o $\fun G$, seg'un corresponda.
\end{defi}

Ponemos entre par'entesis si es covariante o contravariante porque el dominio del Hom-funtor aclara inmediatamente eso. As'i, si no existe confusi'on decimos simplemnte que un funtor es representable. Antes de discutir su relaci'on con la definici'on \ref{elemento local}, veamos  algunos ejemplos. 

\begin{ejems} \leavevmode\begin{enumerate}[label=\alph*)]
    \item El funtor potencia $\ff{\fun P}{\set}{\set}$ visto en \ref{ejemplos funtores} no es representable, sin embargo, el funtor potencia contravariante $\ff{\invopp {\fun P}}{\opp \set}{\set}$ s'i es representable. Consideremos el objeto $\mathbf{2}= \{\varnothing, \{\varnothing\}\}=\{\mathbf{0},\uno\}$, y a $\ff{\homo{-}{\dos}}{\opp \set}{\set}$. La transformaci'on natural $\ff{\alpha}{\invopp {\fun P}}{\homo{-}{\dos}}$ est'a dada por la familia de morfismos $\ff{\alpha_A}{P(A)}{\homo{A}{\dos}}$ para cada conjunto $A$, que asigna $B\mapsto \chi_B$, donde $B\subseteq A$ y $\chi_B$ es la funci'on caracter'istica de $B$, 
    \[  \chi_B(x)=
    \begin{cases}
        \cero \quad \text{si}\, x\notin B\\
        \uno \quad \text{si}\, x\in B.
    \end{cases}\]
    Es claro ver por qu'e estos morfismos inducen una biyecci'on. Dado un subconjunto $B\subseteq A$, los elementos que pertenecen a 'este son mandados a $\uno$, es decir, $B$ no es otra cosa m'as que la preimagen bajo $\chi_B$ de $\uno$. Por otro lado, para ver la naturalidad necesitamos que dada una funci'on $f: A\to B$, el siguiente diagrama conmute

    % https://q.uiver.app/#q=WzAsNCxbMCwwLCJQKEIpIl0sWzIsMCwiUChBKSJdLFswLDIsIlxcaG9tb3tCfXtcXGRvc30iXSxbMiwyLCJcXGhvbW97QX17XFxkb3N9Il0sWzAsMSwiXFxpbnYgZiJdLFswLDIsIlxcYWxwaGFfQiIsMl0sWzEsMywiXFxhbHBoYV9BIl0sWzIsMywiZl4qIiwyXV0=
\[\begin{tikzcd}
	{P(B)} && {P(A)} \\
	\\
	{\homo{B}{\dos}} && {\homo{A}{\dos},}
	\arrow["{\inv f}", from=1-1, to=1-3]
	\arrow["{\alpha_B}"', from=1-1, to=3-1]
	\arrow["{\alpha_A}", from=1-3, to=3-3]
	\arrow["{f^*}"', from=3-1, to=3-3]
\end{tikzcd}\]

donde $f^*$ denota la precomposici'on. Notemos que el camino derecho nos dice que dado un conjunto $C\subseteq B$, nos da una funci'on clasificadora de la preimagen de $C$, es decir, nos da $\chi_{\inv f(C)}$. Por otro lado, el camino izquierdo nos dice que dado ese conjunto $C \subseteq B$, nos da a trav'es de la precomposici'on con $f^*$ una funci'on clasificadora dada por $\chi_C f$. Ver la igualdad entonces equivale a que $\chi_C f$ sea una funci'on clasificadora de $\inv f(C)\subseteq A$, lo cual es claramente el caso. Por lo tanto, el diagrama conmuta. 

De los argumentos anteriores vemos que $\alpha$ es un isomorfismo natural y por lo tanto $\invopp {\fun{P}}$ es representado por $\dos$. En realidad, es claro ver que cualquier conjunto con dos elementos representa a este funtor. 

Pensemos ahora 

\item Consideremos el funtor identidad de $\set$, $\ff{\id{\set}}{\set}{\set}$. Este funtor est'a representado por el elemento $\uno$. Es decir, $\homo{\uno}{-}\cong \id{\set}$. La transformaci'on natural es bastante directa, para cada $X$ en $\ob \set$, $\homo{\uno}{X}$ es el conjunto de todos los elementos de $X$, es decir, de todos los $\uno \xrightarrow{x} X$ que recordando la discusi'on en \ref{elemento global} no es m'as que los $x\in X$, por lo que $\homo{\uno}{X}$ es isom'orfico a $X$. Es inmediato ver por qu'e esta biyecci'on es natural en $\set$ tambi'en. Este funtor nos muestra de una manera clara que en $\set$, dado un conjunto $X$, $x\in X \iff \uno\xrightarrow{x} X$.  
    \end{enumerate}
 \end{ejems}

Antes de enunciar y demostrar el lema de Yoneda, veamos el siguiente ejemplo. 

\begin{ejem} \label{yonedaejem} Pensemos en un orden parcial $(P,\leq)$ como categor'ia y en un funtor $\F\to \set$. Para facilidad de consideraci'on, supongamos que la cardinalidad del conjunto $P$ es igual o menor a la de los naturales. Consideremos un objeto $p$. Tenemos que 
 \[\homo{p}{-}= \{q\in P\,\vert\, p\leq q\}\]
De existir una transformaci'on natural $\nat{\alpha}{\homo{p}{-}}{\F}$, debe ser tal que para cualquier $q\to r$, el diagrama
 % https://q.uiver.app/#q=WzAsNCxbMCwwLCJcXGhvbW97cH17cX0iXSxbMiwwLCJcXGhvbW97cH17cn0iXSxbMCwyLCJcXEYgcSJdLFsyLDIsIlxcRiByIl0sWzAsMV0sWzAsMiwiXFxhbHBoYSIsMl0sWzIsM10sWzEsMywiXFxhbHBoYSJdXQ==
\[\begin{tikzcd}
	{\homo{p}{q}} && {\homo{p}{r}} \\
	\\
	{\F q} && {\F r}
	\arrow[from=1-1, to=1-3]
	\arrow["\alpha_q"', from=1-1, to=3-1]
	\arrow["\alpha_r", from=1-3, to=3-3]
	\arrow[from=3-1, to=3-3]
\end{tikzcd}\]
conmuta. Si $p$  y $q$ no son comparables bajo $\leq$ el diagrama no aporta informaci'on alguna, del mismo modo que si $q\leq p$, pues ser'ia vac'io en su contenido. Si $p\leq q$, $\homo{p}{q}$ es simplemente la flecha $p\to q$ y del mismo modo $\homo{p}{r}$ es la flecha $p\to r$, es decir, ambos son unitarios. $\homo{p}{q}\to \homo{p}{r}$ es simplemente la transitividad del orden parcial en acci'on: $p\leq q \leq r$ implica $p\leq r$. De esto observamos que el diagrama determina elementos $\alpha_q\in \F q$ y $\alpha_r\in \F r$. Si denotamos por $f_{q,r}$ a la $\F q\to \F r$ del diagrama, vemos que la naturalidad implica $\alpha_r = f_{q,r}\alpha_q$. Si seguimos este proceso con una cadena que inicie en $p$, i.e., 
% https://q.uiver.app/#q=WzAsMTAsWzEsMCwiXFxob21ve3B9e3F9Il0sWzIsMCwiXFxob21ve3B9e3J9Il0sWzMsMCwiXFxob21ve3B9e3N9Il0sWzQsMCwiXFxkb3RzIl0sWzEsMSwiXFxGIHEiXSxbMiwxLCJcXEYgciJdLFszLDEsIlxcRiBzIl0sWzQsMSwiXFxkb3RzIl0sWzAsMCwiXFxob21ve3B9e3B9Il0sWzAsMSwiXFxGIHAiXSxbMCwxXSxbMSwyXSxbMiwzXSxbMCw0LCJcXGFscGhhIiwyXSxbMSw1LCJcXGFscGhhIiwyXSxbMiw2LCJcXGFscGhhIiwyXSxbNCw1LCJmX3txLHJ9IiwyXSxbNSw2LCJmX3tyLHN9IiwyXSxbNiw3LCJmX3tzLHR9IiwyXSxbOCwwXSxbOCw5LCJcXGFscGhhX3AiLDJdLFs5LDQsImZfe3AscX0iLDJdXQ==
\[\begin{tikzcd}
	{\homo{p}{p}} & {\homo{p}{q}} & {\homo{p}{r}} & {\homo{p}{s}} & \dots \\
	{\F p} & {\F q} & {\F r} & {\F s} & \dots
	\arrow[from=1-1, to=1-2]
	\arrow["{\alpha_p}"', from=1-1, to=2-1]
	\arrow[from=1-2, to=1-3]
	\arrow["\alpha_q"', from=1-2, to=2-2]
	\arrow[from=1-3, to=1-4]
	\arrow["\alpha_r"', from=1-3, to=2-3]
	\arrow[from=1-4, to=1-5]
	\arrow["\alpha_s"', from=1-4, to=2-4]
	\arrow["{f_{p,q}}"', from=2-1, to=2-2]
	\arrow["{f_{q,r}}"', from=2-2, to=2-3]
	\arrow["{f_{r,s}}"', from=2-3, to=2-4]
	\arrow[ from=2-4, to=2-5]
\end{tikzcd}\]
 se vuelve un ejercicio inductivo ver que para todos estos elementos son iguales a $\alpha_p$. Por supuesto, el orden parcial puede tener muchas ramificaciones en el proceso. Sin embargo, hay un componente $\alpha$ que \enquote{absorbe} a todos los dem'as valores, a saber, $\alpha_p: \homo{p}{p}\to \F p$. En principio, nuestro orden parcial visualizando a todos los elementos comparables con $p$ podr'ia verse de este estilo 
 % https://q.uiver.app/#q=WzAsMTksWzQsNSwicCJdLFszLDQsInEiXSxbMiwzLCJyIl0sWzEsMiwicyJdLFswLDEsInQiXSxbMiwxLCJcXGJ1bGxldCJdLFszLDIsIlxcYnVsbGV0Il0sWzQsMSwiXFxidWxsZXQiXSxbNCwzLCJcXGJ1bGxldCJdLFs1LDIsIlxcYnVsbGV0Il0sWzYsMSwiXFxidWxsZXQiXSxbNywwLCJcXGRvdHMiXSxbNSwwLCJcXGRvdHMiXSxbNSw0LCJcXGJ1bGxldCJdLFs2LDMsIlxcYnVsbGV0Il0sWzcsMSwiXFxidWxsZXQiXSxbNywyLCJcXGJ1bGxldCJdLFsyLDAsIlxcZG90cyJdLFsxLDAsIlxcZG90cyJdLFswLDFdLFsxLDJdLFsyLDNdLFszLDRdLFszLDVdLFsyLDZdLFs2LDVdLFs2LDddLFsxLDhdLFs4LDldLFs5LDEwXSxbMTAsMTFdLFsxMCwxMl0sWzAsMTNdLFsxMyw4XSxbMTMsMTRdLFsxNCw5XSxbOSwxNV0sWzE0LDE2XSxbMTYsMTVdLFs1LDE3XSxbNCwxOF1d
\[\begin{tikzcd}
	& \dots & \dots &&& \dots && \dots \\
	t && \bullet && \bullet && \bullet & \bullet \\
	& s && \bullet && \bullet && \bullet \\
	&& r && \bullet && \bullet \\
	&&& q && \bullet \\
	&&&& p
	\arrow[from=2-1, to=1-2]
	\arrow[from=2-3, to=1-3]
	\arrow[from=2-7, to=1-6]
	\arrow[from=2-7, to=1-8]
	\arrow[from=3-2, to=2-1]
	\arrow[from=3-2, to=2-3]
	\arrow[from=3-4, to=2-3]
	\arrow[from=3-4, to=2-5]
	\arrow[from=3-6, to=2-7]
	\arrow[from=3-6, to=2-8]
	\arrow[from=3-8, to=2-8]
	\arrow[from=4-3, to=3-2]
	\arrow[from=4-3, to=3-4]
	\arrow[from=4-5, to=3-6]
	\arrow[from=4-7, to=3-6]
	\arrow[from=4-7, to=3-8]
	\arrow[from=5-4, to=4-3]
	\arrow[from=5-4, to=4-5]
	\arrow[from=5-6, to=4-5]
	\arrow[from=5-6, to=4-7]
	\arrow[from=6-5, to=5-4]
	\arrow[from=6-5, to=5-6]
\end{tikzcd}\]
donde los $\bullet$ son otros elementos del orden parcial. Las flechas nos indican que los elementos estan bajo la relaci'on $\leq$. El conjunto pudiera ser denso tambi'en, pero como todos estos elementos son comparables con $p$, la naturalidad nos dar'a siempre que estos elementos son iguales a $\alpha_p\in \F p$. Por lo tanto, la transformaci'on natural $\nat{\alpha}{\homo{p}{-}}{\F}$ est'a completamente determinada por la elecci'on $\alpha_p\in \F p$. Podemos pensar en este elemento como la imagen del morfismo $\id p$ bajo la transformaci'on natural $\ff{\alpha_p}{\homo{p}{p}}{\F p}$. ¿Cu'ales son las posibles elecciones de elementos? Vemos que es precisamente los $x\in \F p$. 

De este an'alisis vemos que las transformaciones naturales de $\homo{p}{-}$ al funtor $\F$ est'an determinadas por los elementos de $\F p$. Esta idea, como veremos, no depende de nuestra elecci'on $p$ o del funtor en particular, ni de haber trabajado en un conjunto parcialmente ordenado.

Volveremos a este ejemplo en un momento, pero ahora enunciamos el teorema m'as importante en teor'ia de categor'ias.
\end{ejem}
\newpage

\begin{teo}[Lema de Yoneda] \label{lema yoneda}
    Para cualquier funtor $\ff{\F}{\C}{\set}$, con $\C$ una categor'ia localmente pequeña, y cualquier objeto $A$ en $\ob C$, la siguiente funci'on es una biyecci'on:
    \[ \varphi \colon \Nat(\homo{A}{-}),\F)\cong \F A,    \] 
    \[ \varphi(\alpha) = \alpha_A(\id A)\]

    donde $\Nat(\homo{A}{-}),\F)$ denota la colecci'on de transformaciones naturales entre $\homo{A}{-}$ y $\F$.
\end{teo}
\begin{proof}
    Queremos construir una $\ff{\phi}{\F A}{\Nat(\homo{A}{-}),\F)}$ que sea la inversa de $\varphi$. Para este fin, para cada elemento $x\in \F A$, $\phi(x)$ debe ser una transformaci'on natural, por lo que tenemos que definir una colecci'on de morfismos $\ff{\phi(x)_B}{\homo{A}{B}}{\F B}$ para cada objeto $B$ en $\ob C$ que definan a $\nat{\phi(x)}{\homo{A}{-}}{\F}$. Para este fin, definimos para cada $x\in \F A$ y cada objeto $B$ en $\ob C$,
    \[\phi(x)_B(f) = \F f(x)\]
    recordando que $f: A\to B$. Para corroborar que $\phi(x)$ es una transformaci'on natural, debemos verificar que para cualquier $\ff{g}{B}{C}$ con $C$ en $\ob C$, el diagrama 
    % https://q.uiver.app/#q=WzAsNCxbMCwwLCJcXGhvbW97QX17Qn0iXSxbMiwwLCJcXGhvbW97QX17Q30iXSxbMCwyLCJcXEYgQiJdLFsyLDIsIlxcRiBDIl0sWzAsMSwiZ18qIl0sWzAsMiwiXFxwaGkoeClfQiIsMl0sWzIsMywiRmciLDJdLFsxLDMsIlxccGhpKHgpX0MiXV0=
\[\begin{tikzcd}
	{\homo{A}{B}} && {\homo{A}{C}} \\
	\\
	{\F B} && {\F C}
	\arrow["{g_*}", from=1-1, to=1-3]
	\arrow["{\phi(x)_B}"', from=1-1, to=3-1]
	\arrow["{\phi(x)_C}", from=1-3, to=3-3]
	\arrow["Fg"', from=3-1, to=3-3]
\end{tikzcd}\]
conmuta, donde $g_*$ denota la post-composici'on. Evaluemos el camino izquierdo del diagrama.  

\begin{align*}
    \F g(\phi(x)_B(f)) &= \F g \F f(x) \\
                     &= \F gf(x) \\
                     &= \phi(x)_C (gf)\\
                     &= \phi(x)_C(g_*(f)).
\end{align*}
Por lo tanto el diagrama conmuta. Notemos que el paso de la primera igualdad a la segunda se cumple por los axiomas de funtorialidad vistos en \ref{funtor}.

Recordando que $\varphi(x)$ es una transformaci'on natural y por definici'on de $\phi$ tenemos que $\phi\varphi(x) = \phi(x)_A(\id A)$, por definici'on de $\phi(x)_A$ esto es igual a $\F \id A(x)$ y por axiomas de funtorialidad esto es $\id{\F A}(x)= x$. Por lo tanto, $\phi\varphi(x) = x$. 

Para probar la biyecci'on, nos resta probar que $\varphi\phi(\alpha) = \alpha$. Consideremos un morfismo $\ff{f}{A}{B}$. Evaluando, 
\begin{align*}
    \varphi(\phi(\alpha))_B (f) &= \varphi(\alpha_A(\id A))_B (f) \\
                                &= F(f)(\alpha_A(\id A))
\end{align*}
Ahora, $\alpha$ es una transformaci'on natural, por lo que el diagrama 
% https://q.uiver.app/#q=WzAsNCxbMCwwLCJcXGhvbW97QX17QX0iXSxbMiwwLCJcXGhvbW97QX17Qn0iXSxbMCwyLCJcXEYgQSJdLFsyLDIsIlxcRiBCIl0sWzAsMSwiZl8qIl0sWzAsMiwiXFxwaGkoeClfQSIsMl0sWzIsMywiRmYiLDJdLFsxLDMsIlxccGhpKHgpX0IiXV0=
\[\begin{tikzcd}
	{\homo{A}{A}} && {\homo{A}{B}} \\
	\\
	{\F A} && {\F B}
	\arrow["{f_*}", from=1-1, to=1-3]
	\arrow["{\alpha_A}"', from=1-1, to=3-1]
	\arrow["{\alpha_B}", from=1-3, to=3-3]
	\arrow["Ff"', from=3-1, to=3-3]
\end{tikzcd}\]
conmuta. La 'ultima igualdad, $F(f)(\alpha_A(\id A))$ es el camino izquierdo del diagrama, por lo que es igual a $\alpha_B(f_*(\id A))$. Tenemos, 
\begin{align*}
    \alpha_B(f_*(\id A)) & = \alpha_B(f(\id A)) \\
                         &= \alpha_B(f)
\end{align*}
Por lo tanto $\varphi(\phi(\alpha))_B (f) = \alpha_B (f)$. Como el morfismo fue elegido arbitrariamente concluimos que $\varphi(\phi(\alpha))_B = \alpha_B$, y como $B$ fue elegido arbitrariamente, concluimos que para todo morfismo de la colecci'on que conforman a $\alpha$ y a $\varphi(\phi(\alpha))$ se cumple la igualdad. Como coinciden en todos sus componentes, es el caso que son la misma transformaci'on natural. Por lo tanot, $\varphi(\phi(\alpha))=\alpha$. De esto concluimos que las funciones $\phi$ y $\varphi$ son inversas y por lo tanto $\phi$ es una biyecci'on. 
\end{proof}

Existe la versi'on del lema de Yoneda para pregavillas, es decir, tenemos 

\begin{coro}[Contravariante del lema de Yoneda] 
     Para cualquier funtor $\ff{\F}{\opp C}{\set}$, con $\opp C$ una categor'ia localmente pequeña, y cualquier objeto $A$ en $\ob C$, la siguiente funci'on es una biyecci'on:
    \[ \varphi \colon \Nat(\homo{-}{A}),\F)\cong \F A,    \] 
    \[ \varphi(\alpha) = \alpha_A(\invopp{\id A})\]
\end{coro}
\begin{proof}
    La demostraci'on de este teorema es completamente an'aloga a la demostraci'on del lema de Yoneda en su versi'on covariante. 
\end{proof}

Podemos extender el lema. La biyecci'on $\varphi$ de \ref{lema yoneda} es una correspondencia natural tanto en $A$ y $\F$. Pero expresar a qu'e se refiere esto en particular requiere la introducci'on de algunos conceptos. 

\begin{defi} [Categor'ia producto]
    Sean $\C$ y $\cat D$ categor'ias. Definimos la categor'ia producto $\C \times \cat D$ dada por 
    \begin{itemize}
        \item objetos que son parejas ordenadas $(c,d)$ donde $c$ es un objeto en $\ob C$ y $d$ es un objeto en $\ob D$, 
        \item morfismos que son parejas ordenadas $\ff{(f,g)}{(c,d)}{(e,f)}$ con $\ff{f}{c}{e}$ morfismo en $\C$ y $\ff{g}{d}{f}$ morfismo en $\cat D$
    \end{itemize}
    la composici'on y la identidad est'an definidas componentemente. 
\end{defi}
\newpage
\begin{defi} Dadas categor'ias $\C$ y $\cat D$, definimos la categor'ia $\cat D^{\C}$ donde 
\begin{itemize}
    \item los objetos son funtores $\ff{\F}{\cat C}{\cat D}$,
    \item los morfismos son transformaciones naturales entre funtores $\df{\C}{\cat D}{\F}{\fun G}$
    \item dado un funtor $\ff{\F}{\C}{\cat D}$ la transformaci'on natural identidad $\nat{\id{\F}}{\F}{\F}$ est'a definida por los morfismos $(\id{\F})_c = \id{\F_c}$.
    \item dadas transformaciones naturales $\nat{\alpha}{\F}{\fun G}$, $\nat{\beta}{\fun G}{\fun H}$ la composici'on $\nat{\beta\alpha}{\F}{\fun H}$ est'a definida en sus componentes $\pare{\beta\alpha}_A= \beta_A\alpha_A$.
\end{itemize}
    
\end{defi}

Una observaci'on importante de la definici'on anterior es que para garantizar que 'esta es efectivamente una categor'ia, la composici'on de los morfismos debe estar bien definida, y esto s'olo es el caso si en efecto $\beta\alpha$ es una transformaci'on natural. Consideremos $\ff{f}{A}{B}$ en $\C$, la naturalidad de $\alpha$ nos dice que los diagramas 
% https://q.uiver.app/#q=WzAsOCxbMCwwLCJcXEYgQSJdLFswLDIsIlxcRiBCIl0sWzIsMCwiXFxmdW4gRyBBIl0sWzIsMiwiXFxmdW4gRyBCIl0sWzQsMCwiXFxmdW4gRyBBIl0sWzQsMiwiXFxmdW4gRyBCIl0sWzYsMiwiXFxmdW4gSCBCIl0sWzYsMCwiXFxmdW4gSCBBIl0sWzAsMSwiXFxGIGYiLDJdLFswLDIsIlxcYWxwaGFfQSJdLFsyLDMsIlxcZnVuIEcgZiJdLFsxLDMsIlxcYWxwaGFfQiIsMl0sWzQsNSwiXFxmdW4gRyBmIiwyXSxbNSw2LCJcXGJldGFfQiIsMl0sWzQsNywiXFxiZXRhX0EiXSxbNyw2LCJcXGZ1biBIIGYiXV0=
\[\begin{tikzcd}
	{\F A} && {\fun G A} && {\fun G A} && {\fun H A} \\
	\\
	{\F B} && {\fun G B} && {\fun G B} && {\fun H B}
	\arrow["{\alpha_A}", from=1-1, to=1-3]
	\arrow["{\F f}"', from=1-1, to=3-1]
	\arrow["{\fun G f}", from=1-3, to=3-3]
	\arrow["{\beta_A}", from=1-5, to=1-7]
	\arrow["{\fun G f}"', from=1-5, to=3-5]
	\arrow["{\fun H f}", from=1-7, to=3-7]
	\arrow["{\alpha_B}"', from=3-1, to=3-3]
	\arrow["{\beta_B}"', from=3-5, to=3-7]
\end{tikzcd}\]

conmutan. Podemos unir estos dos diagramas y ver que por la naturalidad de cada componente, el rect'angulo exterior del diagrama
% https://q.uiver.app/#q=WzAsNixbMCwwLCJcXEYgQSJdLFswLDIsIlxcRiBCIl0sWzIsMCwiXFxmdW4gRyBBIl0sWzIsMiwiXFxmdW4gRyBCIl0sWzQsMCwiXFxmdW4gSCBBIl0sWzQsMiwiXFxmdW4gSCBCIl0sWzAsMSwiXFxGIGYiLDJdLFswLDIsIlxcYWxwaGFfQSJdLFsyLDMsIlxcZnVuIEcgZiJdLFsxLDMsIlxcYWxwaGFfQiIsMl0sWzIsNCwiXFxiZXRhX0EiXSxbNCw1LCJcXGZ1biBIIGYiXSxbMyw1LCJcXGJldGFfQiIsMl1d
\[\begin{tikzcd}
	{\F A} && {\fun G A} && {\fun H A} \\
	\\
	{\F B} && {\fun G B} && {\fun H B}
	\arrow["{\alpha_A}", from=1-1, to=1-3]
	\arrow["{\F f}"', from=1-1, to=3-1]
	\arrow["{\beta_A}", from=1-3, to=1-5]
	\arrow["{\fun G f}", from=1-3, to=3-3]
	\arrow["{\fun H f}", from=1-5, to=3-5]
	\arrow["{\alpha_B}"', from=3-1, to=3-3]
	\arrow["{\beta_B}"', from=3-3, to=3-5]
\end{tikzcd}\]
conmuta. De esto podemos ver que efectivamente $\beta\alpha$ es una transformaci'on natural y la composici'on est'a bien definida. 

Requerimos la construcci'on de estas dos categor'ias para hablar de una en particular. Podemos considerar dada una categor'ia $\C$ localmente pequeña, la categor'ia $\C \times \set^{\C}$. Aqu'i, podemos considerar un funtor $\ff{\Nat(\homo{-}{-},-)}{\C \times \set^{\C}}{\set}$ que recibe un objeto $(A,\F)$ y lo manda al conjunto $\Nat(\homo{A}{-}),\F)$. Es importante notar en este punto, que el hecho de que $\C$ sea localmente pequeña, no garantiza que $\set^{\C}$ sea localmente pequeña, por lo que no podr'iamos garantizar en principio que $\Nat(\homo{A}{-}),\F)$ sea un conjunto. Sin embargo, el lema de Yoneda precisamente garantiza esto, por lo que el codominio de este funtor est'a bien definido. Podemos considerar otro funtor de $\C\times \set^{\C}$ a $\set$, el funtor $\ff{\ev}{\C\times \set^{\C}}{\set}$ que manda objetos $(A,\F)$ al conjunto $\F A$. Con esta categor'ias y estos funtores en vista, podemos entender a la funci'on $\varphi$ dada en el lema de Yoneda como un isomorfismo natural $\nat{\varphi}{\Nat(\homo{-}{-},-)}{\ev}$. Es en este sentido que decimos que tanto $A$ como $\F$ son naturales en \ref{lema yoneda}. Como tal, esto es parte de la totalidad del lema de Yoneda, por lo que lo enunciamos como una \enquote{segunda parte}.

\begin{teo}[Segunda parte del Lema de Yoneda] El isomorfismo $\varphi$ visto en \ref{lema yoneda} es natural tanto en $A$ como en $\F$ cuando ambos lados son vistos como funtores de $\C\times \set^{\C}$ a $\set$. 
    
\end{teo}
\begin{proof}
    La demostraci'on consiste en dos partes. Primero, fijemos un funtor $\ff{\F}{\C}{\set}$ y consideremos un morfismo $f: A\to B$ en $\C$. Consideremos el diagrama $$\diag{\Nat(\homo{A}{-}),\F)}{\Nat(\homo{B}{-}),\F)}{\F A}{\F B}{\circ(f^*)}{\F f}{\varphi_{A,\F}}{\varphi_{B,\F}}$$
\end{proof}
Aqu'i, $\varphi_{A,\F}$ define la funci'on definida en \ref{lema yoneda} en los valores $A$ y $\F$ y del mismo modo para $\varphi_{B,\F}$; por otro lado, $\circ f^*$ es dada a trav'es de la composici'on con una transformaci'on natural, $\circ f^*(\alpha)=\nat{\alpha f^*}{\homo{A}{-}}{\homo{B}{-}}$. Evaluando el lado derecho del diagrama, 
\begin{align*}
    \varphi_{B,\F}\pare{\circ(f^*)(\alpha)} & =  \varphi_{B,\F}(\alpha f^*) \\
                                            & = (\alpha f^*)_B (\id B)\\
                                            & = \alpha_B (f)\\
                                            & \stackrel{\mathclap{\mbox{*}}}{=} \alpha_B(f_*(1_A))\\
                                            & \stackrel{\mathclap{\mbox{*}}}{=} \F (f)(\alpha_A(1_A)) \\ 
                                            & = \F(f)\varphi_{A,\F}(\alpha)\\
\end{align*}
La justificaci'on a las igualdades con un asterisco encima se encuentran en la demostraci'on de \ref{lema yoneda}. La 'ultima igualdad es precisamente el lado izquierdo del diagrama, de lo que concluimos que el diagrama conmuta. Por tanto, la funci'on es natural en $A$.

Ahora fijemos al objeto $A$ en $\ob C$ y consideremos una transformaci'on natural $\nat{\alpha}{\F}{\fun G}$ en $\set^{\C}$, es decir, $\alpha$ es un morfismo en dicha categor'ia. Consideremos el diagrama $$\diag{\Nat(\homo{A}{-},\F)}{\Nat(\homo{A}{-},\fun G}{\F A}{\fun G A}{\ev_A}{}{}{}$$


El lema de Yoneda nos dice que podemos entender cualquier objeto de una categor'ia a trav'es de c'omo se comportan todos los dem'as objetos. Por eso mismo a los morfismos tambi'en los llamamos elementos locales, como vimos en \ref{elemento local}: son puntos de vista de un objeto que queremos entender. Es decir, para saber todo lo que necesitamos de un objeto dado, basta con ver c'omo interaccionan los dem'as objetos de la categor'ia con 'este. 

Regresemos al ejemplo \ref{yonedaejem}. Entendemos ahora, por el lema de Yoneda, que 
\[\Nat(\homo{p}{-}, \F) \cong \F p\]
Nosotros sab'iamos que las transformaciones naturales estaban determinadas por los elementos de $\F p$, pero con el lema de yoneda entendemos tambi'en que los elementos de $\F p$ est'an determinados por las transformaciones naturales y que esta correspondencia es biyectiva. 

\nocite{*}
\printbibliography
\end{document}
